\documentclass[fontsize=11pt, paper=a4, twoside=true, BCOR=5mm, DIV=10]{scrbook}

% -------------------------------------------------------------------
% PAQUETES Y OPCIONES
% -------------------------------------------------------------------
\RequirePackage[utf8]{inputenc} 
\RequirePackage[english, spanish, es-nodecimaldot, es-noindentfirst, es-tabla]{babel}

\usepackage{listings}
\usepackage{dcolumn}
\usepackage{enumitem}
\usepackage[chapter]{algorithm}
\usepackage{algpseudocode}
\RequirePackage{verbatim}
%\RequirePackage[Glenn]{fncychap}
\usepackage{fancyhdr}
\usepackage{graphicx}
%\usepackage{doxygen/doxygen}
\usepackage{pdfpages}
\usepackage{url}
\usepackage{colortbl,longtable}
\usepackage[stable]{footmisc}
\usepackage{index}
\usepackage{subfigure}
\usepackage{multicol}
\usepackage{booktabs}
\selectlanguage{spanish} 
\floatname{algorithm}{Algoritmo}


\newcolumntype{.}{D{.}{\esperiod}{-1}}
\makeatletter
\addto\shorthandsspanish{\let\esperiod\es@period@code}
\makeatother

\DeclareUnicodeCharacter{200B}{}

\usepackage[pdfborder={000}]{hyperref} %referencia

% -------------------------------------------------------------------
% INFORMACIÓN DEL TFG Y EL AUTOR
% -------------------------------------------------------------------
\newcommand{\myTitle}{Título del proyecto\xspace}
\newcommand{\myDegree}{Grado en Ingeniería informática}
\newcommand{\myName}{Cristhian Moya Mota (alumno)\xspace}
\newcommand{\myProf}{Diego Jesús García Gil (tutor1)\xspace}
\newcommand{\myOtherProf}{Julián Luengo Martín (tutor2)\xspace}
%\newcommand{\mySupervisor}{Put name here\xspace}
\newcommand{\myFaculty}{Escuela Técnica Superior de Ingenierías Informática y de
Telecomunicación\xspace}
\newcommand{\myFacultyShort}{E.T.S. de Ingenierías Informática y de
Telecomunicación\xspace}
\newcommand{\myDepartment}{Departamento de ...\xspace}
\newcommand{\myUni}{\protect{Universidad de Granada}\xspace}
\newcommand{\myLocation}{Granada\xspace}
\newcommand{\myTime}{\today\xspace}
\newcommand{\myVersion}{Version 0.1\xspace}


% -------------------------------------------------------------------
% ESTILOS DE LA CLASE KOMA
% -------------------------------------------------------------------
% Selecciona el tipo de fuente para los títulos (capítulo, sección, subsección) del documento.
\setkomafont{disposition}{\sffamily\bfseries}

% Cambia el ancho de la cita. Al inicio de un capítulo podemos usar el comando \dictum[autor]{cita} para añadir una cita famosa de un autor.
\renewcommand{\dictumwidth}{0.45\textwidth} 

\recalctypearea % Necesario tras definir la tipografía a usar.

\hypersetup{
	    pdfauthor = {\myName (email (en) ugr (punto) es)},
		% hidelinks,            % Enlaces sin color ni borde. El borde no se imprime
		linkbordercolor=0.8 0 0,
		citebordercolor=0 0.8 0,
		citebordercolor=0 0.8 0,
		colorlinks = true,            % Color en texto de los enlaces. Comentar esta línea o cambiar `true` por `false` para imprimir el documento.
		linkcolor = [rgb]{0.5, 0, 0}, % Color de los enlaces internos
		urlcolor = [rgb]{0, 0, 0.5},  % Color de los hipervínculos
		citecolor = [rgb]{0, 0.5, 0}, % Color de las referencias bibliográficas
		pdfsubject={Trabajo de fin de Grado},%
		pdfkeywords = {palabra_clave1, palabra_clave2, palabra_clave3, ...},
		pdfcreator={pdfLaTeX}%
	}


\newlist{multienum}{enumerate}{1}
\setlist[multienum]{
	label=\alph*),
	before=\begin{multicols}{2},
		after=\end{multicols}
}

\newlist{multiitem}{itemize}{1}
\setlist[multiitem]{
	label=\textbullet,
	before=\begin{multicols}{2},
		after=\end{multicols}
}

% Normal font: 			URW Palladio typeface. 
% Sans-serif font: 	Gill Sans (paquete cabin)
% Monospace font: 	Inconsolata
\RequirePackage[T1]{fontenc}
\RequirePackage[sc, osf]{mathpazo} \linespread{1.05}         
\IfFileExists{cabin.sty}{
	\RequirePackage[scaled=.95,type1]{cabin} 
} %else
{
	% Si cabin da ERROR usar el siguiente comando en su lugar
	\renewcommand{\sfdefault}{iwona} 
}

%\hyphenation{}


%\makeindex
%\usepackage[style=long, cols=2,border=plain,toc=true,number=none]{glossary}
% \makeglossary

% Definición de comandos que me son tiles:
%\renewcommand{\indexname}{Índice alfabético}
%\renewcommand{\glossaryname}{Glosario}

\pagestyle{fancy}
\fancyhf{}
\fancyhead[LO]{\leftmark}
\fancyhead[RE]{\rightmark}
\fancyhead[RO,LE]{\textbf{\thepage}}
\renewcommand{\chaptermark}[1]{\markboth{\textbf{#1}}{}}
\renewcommand{\sectionmark}[1]{\markright{\textbf{\thesection. #1}}}

\setlength{\headheight}{1.5\headheight}

\newcommand{\HRule}{\rule{\linewidth}{0.5mm}}
%Definimos los tipos teorema, ejemplo y definición podremos usar estos tipos
%simplemente poniendo \begin{teorema} \end{teorema} ...
%\newtheorem{teorema}{Teorema}[chapter]
%\newtheorem{ejemplo}{Ejemplo}[chapter]
%\newtheorem{definicion}{Definición}[chapter]

\definecolor{gray97}{gray}{.97}
\definecolor{gray75}{gray}{.75}
\definecolor{gray45}{gray}{.45}
\definecolor{gray30}{gray}{.94}

\lstset{ frame=Ltb,
     framerule=0.5pt,
     aboveskip=0.5cm,
     framextopmargin=3pt,
     framexbottommargin=3pt,
     framexleftmargin=0.1cm,
     framesep=0pt,
     rulesep=.4pt,
     backgroundcolor=\color{gray97},
     rulesepcolor=\color{black},
     %
     stringstyle=\ttfamily,
     showstringspaces = false,
     basicstyle=\scriptsize\ttfamily,
     commentstyle=\color{gray45},
     keywordstyle=\bfseries,
     %
     numbers=left,
     numbersep=6pt,
     numberstyle=\tiny,
     numberfirstline = false,
     breaklines=true,
   }
 
% minimizar fragmentado de listados
\lstnewenvironment{listing}[1][]
   {\lstset{#1}\pagebreak[0]}{\pagebreak[0]}

\lstdefinestyle{CodigoC}
   {
	basicstyle=\scriptsize,
	frame=single,
	language=C,
	numbers=left
   }
\lstdefinestyle{CodigoC++}
   {
	basicstyle=\small,
	frame=single,
	backgroundcolor=\color{gray30},
	language=C++,
	numbers=left
   }

 
\lstdefinestyle{Consola}
   {basicstyle=\scriptsize\bf\ttfamily,
    backgroundcolor=\color{gray30},
    frame=single,
    numbers=none
   }


\newcommand{\bigrule}{\titlerule[0.5mm]}

\algnewcommand\algorithmicforeach{\textbf{for each}}
\algdef{S}[FOR]{ForEach}[1]{\algorithmicforeach\ #1\ \algorithmicdo}
%Para conseguir que en las páginas en blanco no ponga cabecerass
\makeatletter
\def\clearpage{%
  \ifvmode
    \ifnum \@dbltopnum =\m@ne
      \ifdim \pagetotal <\topskip
        \hbox{}
      \fi
    \fi
  \fi
  \newpage
  \thispagestyle{empty}
  \write\m@ne{}
  \vbox{}
  \penalty -\@Mi
}
\makeatother

% -------------------------------------------------------------------
% DESARROLLO DE LOS CAPÍTULOS DEL TRABAJO
% -------------------------------------------------------------------

\begin{document}
	% Plantilla portada UGR
	\input{portada/portada}
	
	% Plantilla prefacio UGR
	\chapter*{}
%\thispagestyle{empty}
%\cleardoublepage

%\thispagestyle{empty}

\begin{titlepage}
 
 
\setlength{\centeroffset}{-0.5\oddsidemargin}
\addtolength{\centeroffset}{0.5\evensidemargin}
\thispagestyle{empty}

\noindent\hspace*{\centeroffset}\begin{minipage}{\textwidth}

\centering
%\includegraphics[width=0.9\textwidth]{imagenes/logo_ugr.jpg}\\[1.4cm]

%\textsc{ \Large PROYECTO FIN DE CARRERA\\[0.2cm]}
%\textsc{ INGENIERÍA EN INFORMÁTICA}\\[1cm]
% Upper part of the page
% 

 \vspace{3.3cm}

%si el proyecto tiene logo poner aquí
\includegraphics[scale=0.35]{imagenes/logo.png} 
 \vspace{0.5cm}

% Title

{\Huge\bfseries Aplicación móvil para la prognosis y detección del cáncer de piel usando deep learning\\
}
\noindent\rule[-1ex]{\textwidth}{3pt}\\[3.5ex]
%{\large\bfseries Subtítulo del proyecto.\\[4cm]}
\end{minipage}

\vspace{2.5cm}
\noindent\hspace*{\centeroffset}\begin{minipage}{\textwidth}
\centering

\textbf{Autor}\\ {Cristhian Moya Mota}\\[2.5ex]
\textbf{Directores}\\
{Diego Jesús García Gil\\
	Julián Luengo Martín}\\[2cm]
%\includegraphics[width=0.15\textwidth]{imagenes/tstc.png}\\[0.1cm]
%\textsc{Departamento de Teoría de la Señal, Telemática y Comunicaciones}\\
%\textsc{---}\\
%Granada, mes de 201
\end{minipage}
%\addtolength{\textwidth}{\centeroffset}
\vspace{\stretch{2}}

 
\end{titlepage}






\cleardoublepage
\thispagestyle{empty}

\begin{center}
{\large\bfseries Título del Proyecto: Subtítulo del proyecto}\\
\end{center}
\begin{center}
Cristhian Moya Mota\\
\end{center}

%\vspace{0.7cm}
\noindent{\textbf{Palabras clave}: palabra\_clave1, palabra\_clave2, palabra\_clave3, ......}\\

\vspace{0.7cm}
\noindent{\textbf{Resumen}}\\

Poner aquí el resumen.
\cleardoublepage


\thispagestyle{empty}


\begin{center}
{\large\bfseries Project Title: Project Subtitle}\\
\end{center}
\begin{center}
First name, Family name (student)\\
\end{center}

%\vspace{0.7cm}
\noindent{\textbf{Keywords}: Keyword1, Keyword2, Keyword3, ....}\\

\vspace{0.7cm}
\noindent{\textbf{Abstract}}\\

Write here the abstract in English.

\chapter*{}
\thispagestyle{empty}

\noindent\rule[-1ex]{\textwidth}{2pt}\\[4.5ex]

Yo, \textbf{Cristhian Moya Mota}, alumno de la titulación Grado en Ingeniería informática de la \textbf{Escuela Técnica Superior
de Ingenierías Informática y de Telecomunicación de la Universidad de Granada}, con DNI 20596451C, autorizo la
ubicación de la siguiente copia de mi Trabajo Fin de Grado en la biblioteca del centro para que pueda ser
consultada por las personas que lo deseen.

\vspace{6cm}

\noindent Fdo: Cristhian Moya Mota

\vspace{2cm}

\begin{flushright}
Granada a X de junio de 2024 .
\end{flushright}


\chapter*{}
\thispagestyle{empty}

\noindent\rule[-1ex]{\textwidth}{2pt}\\[4.5ex]

D. \textbf{Diego Jesús García Gil}, Profesor del Área de XXXX del Departamento YYYY de la Universidad de Granada.

\vspace{0.5cm}

D. \textbf{Julián Luengo Martín}, Profesor del Área de XXXX del Departamento YYYY de la Universidad de Granada.


\vspace{0.5cm}

\textbf{Informan:}

\vspace{0.5cm}

Que el presente trabajo, titulado \textit{\textbf{Aplicación móvil para la prognosis y detección del cáncer de piel usando deep learning}},
ha sido realizado bajo su supervisión por \textbf{Cristhian Moya Mota}, y autorizamos la defensa de dicho trabajo ante el tribunal
que corresponda.

\vspace{0.5cm}

Y para que conste, expiden y firman el presente informe en Granada a X de junio de 2024 .

\vspace{1cm}

\textbf{Los directores:}

\vspace{5cm}

\noindent \textbf{Diego Jesús García Gil \ \ \ \ \ \ Julián Luengo Martín}

\chapter*{Agradecimientos}
\thispagestyle{empty}

       \vspace{1cm}


Poner aquí agradecimientos...


	
	% Índice de contenidos
	\newpage
	\tableofcontents
	
	% Índice de imágenes y tablas
	\newpage
	\listoffigures
	
	% Si hay suficientes se incluirá dicho índice
	\listoftables 
	\newpage
	
	\chapter{Introducción}

\section{Motivación}

El cáncer es una de las causas de muerte principales en el mundo. Su gran agresividad, así como su dificultad de diagnóstico, debido a su gran variedad de ubicaciones y manifestaciones, provoca que un alto porcentaje de casos no sean diagnosticados a tiempo correctamente. Tan solo en 2023, aproximadamente se registraron 20 millones de nuevos casos de cáncer a nivel mundial, y produciéndose algo menos de 10 millones de defunciones.\\

Estos registros provocan una gran inquietud en la población y entre los expertos de la materia; debido al aumento que se produce cada año, se espera que para el año 2050, el número de nuevos casos sea un 70\% mayor.  Por desgracia, no existen formas de prevención claras para este tipo de enfermedad, ni un tratamiento efectivo que permita al paciente recuperarse fácilmente. \\

La única opción probada es la realización de pruebas rutinarias a colectivos de riesgo, para así acelerar la detección de posibles tumores, y aumentar la esperanza de supervivencia. Esto se ve reflejado en las cifras de los dos tipos de cáncer más frecuentes: el cáncer de mama, y el cáncer colorrectal. Si se detectan en fases iniciales, la correcta recuperación del colon podría aumentarse hasta el 90\%, mientras que en el cáncer de mama, podría reducirse su mortalidad entre un 25-31\%. Gracias a la existencia de pruebas rutinarias programadas por servicio de salud, se puede reducir la mortalidad.\\

El gran problema de estos tipos de cánceres son la escasa visibilidad y síntomas de los mismos; cuando muestran señales, es probable que la tasa de supervivencia sea mucho menor, sobre todo en el colon. Pero existe otro tipo de cáncer que sí se manifiesta de forma más visible y que puede alarmar al paciente de forma más temprana: el cáncer de piel.\\

Este tipo de tumores puede manifestarse en las diferentes capas de la dermis, y su origen se atribuye a la exposición prolongada a la luz solar sin hacer uso de protección. Debido a los daños que sufre la capa de ozono, y otros factores ambientales, la cantidad de rayos ultravioleta que llegan hasta la superficie ascendió desde que se tienen registros. Si bien la capa de ozono parece recuperarse, debemos ser cautos, y tener cuidado de nuestra piel; los rayos ultravioleta pueden dañar células de la misma, y provocar alteraciones en su material genéctico. Son las que dan lugar al crecimiento incontrolado de células, son las que forman los tumores cancerígenos en la piel.\\

Se estima que en el mundo, los tumores de la piel representan un tercio de los casos de cáncer diagnosticados. Esta distribución sigue valores parecidos en España, y al igual que las cifras de otros tipos de tumores, los casos diagnosticados aumentan año tras año. Las muertes debidas a esta enfermedad son principalmente, por ser indentificadas en fases tardías de su evolución. Debido a que la piel es el órgano más grande del cuerpo humano, y que está en contacto con todos los capilares sanguineos y el sistema linfático, las células cancerosas se pueden extender por ellos hacia otros lugares del cuerpo.\\

Aunque este cáncer puede ser identificado de forma más sencilla por su portador, la escasa información acerca del tema, y la confusión con otras lesiones benignas de la piel como verrugas o lunares, provoca una disminución en las posibilidades de supervivencia. Por ello, el objetivo de este trabajo es aportar una nueva forma de diagnóstico que permita a los usuarios obtener una orientación acerca de qué posible lesión están experimentando en la piel, y sirvan como complemento del experto. O bien, ayudar a los expertos a tomar la decisión, acortando los tiempos de diagnóstico para aumentar las posibilidades de supervivencia. Esta tarea será realizada gracias al uso de uno de las herramientas en auge en la actualidad: la inteligencia articificial, y concretamente, el uso de DeepLearning para visión por computador.\\
 
 Mediante una nueva arquitectura, el propósito es conseguir un buen modelo, capaz de segmentar las manchas de interés en la piel que estén recogidas en una fotografía. Dicha fotografía será capturada con el telefóno movil del usuario, retirando así la necesidad de disponer de dispositivos especializados. Y posteriormente, clasificar dichas manchas para ofrecer al usuario final una respuesta sólida acerca del posible tipo de lesión de piel que sufre.\\

\newpage
\section{El cáncer de piel}

Prosiguiendo con el cáncer de piel, su diagnóstico si dificulta, sobre todo, por su amplia variedad de formas, tamaños, texturas y manifestaciones. Aunque su visibilidad pueda parecer evidente, (ya que es observable a nivel macroscópico) puede ser confundido fácilmente con lesiones benignas. Normalmente, suele dividirse entre dos tipos diferentes:
\begin{itemize}
	\item \textbf{Melanomas de la piel}. Son la variante más peligrosa. Su origen se encuentra en los melanocitos, las células encargadas de dar el color bronceado a la piel.  Éstas pueden comenzar a crecer sin control originando tumores, los cuales crecen y se diseminan rápidamente hacia otras regiones del organismo, provocando la metástasis, una extensión a nivel total del organismo. Es el más grave de los diagnósticos. Puede identificarse como una mancha oscura en la piel, formando tumores de color café oscuro. Sin embargo, debido a la gran variedad de reacciones, pueden darse de color rosado si dejan de producir melanina. Este aspecto dificulta su diagnóstico, por lo que el papel de las herramientas de visión por computador pueden ayudar a su identificación.
	\item \textbf{Cánceres no melanomas}. Este tipo de cánceres no se ubican en los melanocitos, y pueden ser tratados mediante otras ténicas menos agresivas debido a su rara probabilidad de expansión. Los más comunes, son los tumores de células basales y los de células escamosas:
	\begin{itemize}
		\item Células basales. Componen la capa inferior de la piel, y son las células encargadas de sustituir aquellas que componen la capa más externa de la piel. Se encuentran , por tanto, en constante reproducción para cubrir aquellas que mueren en la superificie. Si experimentan alguna mutación, producen tumores de color similar al de piel del paciente, con la posibilidad de aparecer en colores como negro brillante en las pieles más oscuras.
		
		\item Células escamosas. Son las células externas de la piel, con forma plana. Se regeneran constantemente gracias a las células basales, que producen estas células las cuales se aplanan a medida que ascienden hacia la capa externa. Es frecuente, de nuevo, en zonas expuestas al sol, sobre todo la cara. Normalmente, se encuentran bien localizados, y puede procederse a su extirpación. En casos en los que se haya extendido, se hace uso de radioterapia.
		
	\end{itemize}
\end{itemize}

Aunque en base a su descripción parezcan distinguibles, son fácilmente confudidos por su variedad con otros tumores benignos de la piel, como:

\begin{itemize}
	\item \textbf{Lunares(nevus)}: hiperpigmentación benigna en la piel.
	\item \textbf{Verrugas}: tumores benignos de piel, frecuente debido a virus como el del papiloma humano.
	\item \textbf{Lesiones vasculares}: varices, derrames, y otro tipo de problemas circulatorios.
	\item \textbf{Lipomas}: tumores de tacto blando, debido a su contenido en lípidos (grasa).
	\item \textbf{Queratosis seborreica}: son manchas cerosas, comúnmente desarrolladas en la espalda. De aspecto oscuro y gran relieve, no suponen ninguna amenaza más allá de posible incomodidad al roce o estética.
\end{itemize}

El uso de aprendizaje profundo para este fin resulta interesante como forma de mejora del diagnóstico ante casos malignos y benignos de gran similitud, los cuales pueden confunir y dificultar la labor incluso a expertos dermatólogos.

	
	\chapter{Tendencias y Estado del arte}

Antes de adentrarnos en el análisis del problema, debemos de tener en cuenta de que este problema es una temática en constante evolución, y por tanto, podemos encontrar diferentes conceptos y procedimientos seguidos en la literatura que pueden servirnos de inspiración para abordar el problema sin cometer los errores ya cometidos en el pasado, y ser capaces de encontrar un nuevo enfoque que nos ofrezca ventajas.\\

Generalmente, este problema ha sido abordado empleando hardware de computador de escritorio, por lo que la mayoría de modelos se centran en el aprovechamiento de los recursos hardware alojados en un servidor para realizar la clasificación y evaluación de las imágenes potencialmente cancerosas tomadas. Por tanto, nos adentraremos en sus conceptos, pero teniendo en cuenta que el proyecto propuesto hará uso de dispositivos móviles durante el tiempo de inferencia.

\section{Aprendizaje profundo en dispositivos móviles}

Con la creciente tendencia de la potencia de cálculo en los dispositivos actuales, prácticamente todos los aparatos electrónicos que nos rodean han crecido en cuanto a potencia y complejidad de cálculo. Los smartphones son precisamente uno de ellos, y nos acompañan cada día, por lo que es el dispositivo ideal para tareas de uso cotidiano y portabilidad.\\

Estos dispositivos, a diferencia de los computadores tradicionales, normalmente basado en la arquitectura x86 o AMD64, se basan en ARM, siguiendo como concepto de diseño ofrecer el máximo rendimiento posible dentro de unos consumos contenidos,mejorando el ahorro de energía y la pérdida de la misma mediante calor. El entrenamiento de modelos que encontramos habitualmente en la literatura, como ResNet, Inception o similares, es prácticamente inviable de forma nativa.\\

Sin embargo, en lo que respecta a la inferencia, éstos son capaces de ofrecer muy buenos resultados, gracias a la incorporación de hardware dedicado capaz de ofrecer estas características. Como prueba, podemos observar infinidad de aplicaciones que hace uso de ello, como Google Lens, que si bien se ayuda del uso de servidores de búsqueda especializados, es capaz de realizar procedimientos locales en los dispositivos de gama alta. \\

El objetivo del proyecto es aprovechar dicho vacío en la existencia de apliaciones de inferencia local para ofrecer una app que no necesite de conexión de red permanente para ofrecer resultados acerca de las manchas de piel identificadas.

Aprovechando el auge de los smartphones, grandes empresas, como Google y Meta, centran sus esfuerzos en la creación de arquitecturas basadas en redes convolucionales capaces en realizar detección de imágenes en tiempo real, para la clasificación de distintos objetos que podemos encontrar en nuestra vida cotidiana, y servir así como una herramienta de apoyo para diferentes necesidades. Sin embargo, esto es una tarea completa, ya que suelen carecer de complejas operaciones, o sacrificar en profundidad para lograr un rendimiento aceptable de unas decenas de milisegundos por inferencia.\\

A continuación, evaluaremos algunos de los modelos más conocidos y efectivos de propósito general, como SqueezeNet\cite{iandola2016squeezenet}, MobileNet \cite{howard2017mobilenets}\cite{sandler2019mobilenetv2}\cite{howard2019searching}, ShuffleNet \cite{zhang2017shufflenet} y  EfficientNet\cite{tan2020efficientnet}\cite{eflite}.

\subsection{Squeeze-Net (2016)}

Squeeze-Net\cite{iandola2016squeezenet} se centra en la reducción de complejidad de la arquitectura, sin pérdida de capacidad predictiva y evitando aplicar técnicas de compresión y cuantización\cite{kuzmin2024fp8} de modelos. Se autodefine como ``una red al nivel de AlexNet\cite{NIPS2012_c399862d} pero con una quincuagésima parte de los parámetros'', haciendo alusión a disponer de una capacidad de cálculo similar a AlexNet, pero recortando en cuanto a número de parámetros necesarios.\\

Aunque el motivo de su creación no es de forma directa el uso de la arquitectura en dispositivos móviles, ha sido ampliamente utilizada en ellos al formar parte de la tendencia actual de reducción de coste computacional para reducir las necesidades de potencia de cálculo. De esta forma, se puede facilitar la implementación de redes convolucionales en sistemas empotrados con escasa capacidad de memoria y cómputo, haciendo uso de FPGAs \cite{fpga}.\\

Siguiendo este punto de vista, fue capaz de igualar e incluso superar levemente el rendimiento de AlexNet\cite{NIPS2012_c399862d}, empleando las siguientes técnicas:
\begin{itemize}
    \item Uso de módulos \textbf{Fire}. Se trata de un nuevo tipo de estructura convolucional modular que puede ser apilado en capas al estilo de los módulos Inception \cite{szegedy2014going} de Google. Consiste en una unidad modular ajustable en función de 3 parámetros: el número de convoluciones 1x1, y el número de filtros 1x1 y 3x3 de ``expasión'' a aplicar. El objetivo de añadir las convoluciones 1x1 es, por un lado, la reducción de dimensionalidad del volumen a convolucionar, y por otro lado, la simplificación en número de parámetros. En arquitecturas de gran profundidad, como VGGNet o ResNet, quedó demostrado que este tipo de convoluciones permitían llegar más allá sin perder información relevante para el aprendizaje. \ref{figsqueeze}
    \item Desplazamiento de los métodos de \textbf{reducción} de dimensionalidad hacia las capas más \textbf{profundas} de la arquitectura. En lugar de realizar pooling o aplicar stride a la hora de aplicar el filtro para reducir el volumen de salida en las primeras capas de la red, este tipo de transformaciones se reparten en capas más profundas para evitar que las capas cercanas al Head, de forma que se reduce la pérdida de características si retrasamos el subsampling del filtro.
    \item Eliminación de capas totalmente conectadas. Estas capas son, normalmente, las que mayor complejidad añaden al modelo por su gran cantidad de parámetros. Gracias al uso de Average Pooling en su última capa, podemos tener una red completamente independiente del tamaño de la entrada sin gran cantidad de parámetros ni necesidad de capas adicionales.
\end{itemize}

\begin{figure}[H]
	\label{figsqueeze}
	\centering
	\includegraphics[scale = 0.2]{imagenes/squeezenet.png}
	\caption{Arquitectura de SqueezeNet}
\end{figure}


Esta red ha sido empleada en multitud de aplicaciones: detección de objetos en tiempo real, clasificación semántica, y modelos preliminares de conducción autónoma. Entre todas sus aplicaciones, podríamos destacar su utilización en imagen médica, concretamente MRI (Resonancias magnéticas). Ha permitido facilitar el diagnóstico de ciertas enfermedades y lesiones cerebrales en un espacio de memoria y recursos contenido.\\

Sin embargo, a pesar de las mejoras recibidas en sus versiones sucesivas, como los módulos Fire de doble nivel para reducir dimensionalidad, o la introducción de más reducciones mediante pooling, sacrifica resultados a nivel de accuracy respecto a la competencia, y no ha sido aplicada de forma firme y exitosa sobre imágenes de enfermedades cutáneas.



\subsection{MobileNet}

MobileNet es el fruto del proyecto de investigación de Google Research para la implementación de redes convolucionales en dispositivos móviles. El objetivo era encontrar un modelo eficiente que pueda ser incluso utilizado en tareas de segmentación en tiempo real, pero reduciendo el número de parámetros del red así como el número de operaciones de producto necesarias, para poder ejecutarlas de forma nativa en dispositivos móviles como smartphones y tablets.\\

Esta arquitectura consta de 3 versiones diferentes, siendo cada una más sofisticada que la anterior. Disponemos de MobileNet V1, MobileNet V2 y MobileNetV3.

\subsubsection{MobileNet V1 (2017)}

La versión original de la arquitectura convolucional MobileNet  \cite{howard2017mobilenets} fue publicada en 2017. En esta publicación, se busca reducir el número de operaciones realizadas para conseguir un menor impacto de las operaciones en punto flotante sobre el rendimiento.\\
El punto clave de esta arquitectura reside en las llamadas "pointwise convolutions", haciendo uso del concepto de separabilidad, ampliamente estudiado desde el año 2012 por la literatura.

Las nuevas convoluciones descomponibles se pueden separar en dos pasos bien delimitados: la convolución en profundidad y la convolución puntual.

\begin{itemize}
    \item Las convoluciones en profundidad realizan el producto del filtro con el volumen de entrada capa a capa. Es decir, no se tiene en cuenta la dimensionalidad total de la imagen, sino que se realiza por cada nivel de profundidad el mismo producto. Esto reduce considerablemente el número de parámetros, ya que la dimensionalidad del problema es mucho menor.
    \item La segunda fase es la convolución puntual, cuyo objetivo no es más que acumular el producto de todas las capas calculadas independientemente mediante una simple combinación lineal, la cual es de coste computacional muy bajo.
    
    $$G_{k,l,m} =  \sum_{i,j}^{} K^{i,j,m} * F_{k+i-1,l+j-1,m}$$
\end{itemize}

Este producto es calculable eficientemente por las técnica de álgebra lineal GEMM, que permite aplicar propiedades de la suma y la multiplicación para el producto matricial de forma eficiente mediante Tensor cores.\\

En resumen, gracias a la separabilidad convolucional, se adquieren varias ventajas:
\begin{itemize}
    \item El número de productos se reduce considerablemente. Como se puede verificar en \cite{howard2017mobilenets} se traduce en una reducción de entre 8 y 9 veces el número de operaciones con respecto a las arquitectura tradicional de convolución
    \item Se reduce el espacio necesario en memoria.
    \item Se puede aprovechar el hardware específico.
    \item No se pierde precisión de cálculo gracias a que la separabilidad de convoluciones no afecta al resultado.
\end{itemize}

\begin{figure}[H]
	\label{depthwise}
	\centering
	\includegraphics[scale = 0.225]{imagenes/depthwise.png}
	\caption{Producto depthwise}
\end{figure}

Adicionalmente, también incorporaron dos hiperparámetros: $\alpha$ y $\rho$. Estos parámetros controlan la anchura y la resolución de entrada, respectivamente.\\
El hiperparámetro $\alpha$ hace referencia a la anchura de cada capa convolucional que compone la red, adquiriendo un valor de 1 cuando la arquitectura no se ve reducida, y gradualmente podrá ser reducida en el intervalo (0,1]. Así se conseguirán modelos más simples en anchura para dispositivos con menores recursos.\\

En cuanto a $\rho$, este se encuentra implícito en la resolución de la imagen de entrada. Por defecto, la red acepta imágenes de hasta 224x224, pero en función de dicho valor $\rho$, podremos reducir su resolución también dentro del intervalo (0,1].\\

Ambos parámetros sacrificarán bondad y ajuste en los resultados a favor de una mayor eficiencia.

\subsubsection{MobileNet V2 (2018)}

Dos años más tarde de la publicación de MobileNets, da a luz su versión V2  \cite{sandler2019mobilenetv2}. Esta conserva los hiperparámetros  de la versión anterior, así como el producto punto a punto. Sin embargo, añade tres nuevas características, algunas de ellas no triviales y que requieren experimentación:
\begin{itemize}
    \item Se introduce el concepto de ``residuo invertido''. Ésta mejora reside en la utilización de los bloques residuales, propuesto por la arquitectura de ResNet. Su objetivo es evitar la degradación del gradiente, y que se frene el aprendizaje en modelos de gran profundidad. Normalmente, estas conexiones se realizan entre capas de gran profundidad, siendo las capas intermedias bloques estrechos. Sin embargo, en MobileNet V2, se propone la composición inversa, de forma que sean los bloques intermedios entre los residuales aquellos que poseen una mayor anchura, y así reducir el número de parámetros sin perder expresividad en el modelo \cite{invertedresidualsv2}.
    \begin{figure}[H]
    	\label{invert}
    	\centering
    	\includegraphics[scale = 0.3]{imagenes/invertedres.png}
    	\caption{Bloque residual invertido}
    \end{figure}
    
     \item En las capas donde el volumen de entrada es estrecho, al hacer uso de bloques residuales invertidos, la eliminación de la no linealidad aportada por ReLu favorece a la conservación de características y permite obtener mejores resultados de accuracy en tareas generales como clasificación en imagenet. Esto se debe a que al realizar los saltos entre bloques ``estrechos'' perdemos rendimiento de la red, y simplemente con eliminar la última transformación no lineal del bloque, contrarrestamos este problema.
    \item ReLu6. Se mantiene una versión modificada de la original función de activación. En lugar de utilizar la tradicional función ReLu entre 0 y 1, se extiende este intervalo hasta 6, permitiendo mantener la precisión en caso de utilizar coma fija, ya que se aseguran 3 dígitos de parte entera, y el resto queda destinado a la mantisa, que se almacena de forma precisa.
\end{itemize}

    \begin{figure}[H]
	\label{mv2}
	\centering
	\includegraphics[scale = 0.25]{imagenes/mobilenetv2.png}
	\caption{Arquitectura de MobileNetV2}
\end{figure}

\subsubsection{MobileNet V3 (2019)}

En su tercera versión\cite{howard2019searching}, MobileNet incorpora métodos avanzados de diseños de redes basados en NetAdapt. Este algoritmo se basa en la transformación de modelos preentrenados para escritorio, y, en base a una serie de requisitos de potencia especificados, adapartar la arquitectura a una plataforma móvil perdiendo las mínimas capacidades posibles de la red original.  El modelo de partida empleado fue ajustado con precisión para mejorar latencias y uso de memoria, aplicando los siguientes conceptos:

\begin{itemize}
    \item Se añade la capa Squeeze-and-Excite, dentro de las conexiones residuales. Se trata de un mecanismo surgido en 2018 \cite{hu2019squeezeandexcitation}. Este estudio afirma que existen filtros de imagen con mayor importancia para el cómputo global que otros, como, por ejemplo, los bordes. Por tanto, les aporta un mayor ``peso'' durante el entrenamiento a dichos filtros haciendo uso de una serie de parámetros adicionales. Éstos añaden una carga computacional muy pequeña, por lo que se trata de una técnica eficaz. Para obtener los parámetros de relevancia, se dispone de dos módulos: squeeze, y excite. El módulo squeeze se encarga de representar cada filtro mediante un valor numérico, obtenido por average pooling de la imagen. Y por otro lado, el módulo excite se encarga de aprender los pesos que dar a cada uno de estos filtros o canales, haciendo uso de un MLP. El resultado final serán los pesos de cada canal en cuanto a su importancia, normalizados entre 0 y 1 por una función sigmoide.

    \item Se incluyeron nuevas capas al inicio y al final de la red de tipo residual invertidas.
\end{itemize}

    \begin{figure}[H]
	\label{mv2}
	\centering
	\includegraphics[scale = 0.85]{imagenes/mobilev3.png}
	\caption{Arquitectura de MobileNetV3}
\end{figure}

Debido a la gran complejidad adquirida por el modelo, los problemas de latencia y rendimiento en dispositivos de menor potencia, se opta por dividir la arquitectura en dos modelos parametrizables: MobileNet Small y Large. Mientras que la versión Large mejora los resultados de la versión 2 aumentando las prestaciones, el modelo Small otorga importancia sobre todo a la eficiencia y el uso de memoria, enfocado al hardware embebido o dispositivos de poca potencia.


\subsubsection{Aplicaciones en dermatología y cáncer de piel}

MobileNet, concretamente en su segunda versión, ha sido utilizado en la literatura para el diagnóstico de enfermedades de la piel. En \cite{Chaturvedi_2020}, es utilizado para realizar la clasificación de 7 enfermedades cutáneas extraídas del Humans against Machine, HAM10000 \cite{ham10000}, que podemos encontrar en ISIC archive \cite{isicarchive}, un repositorio web de acceso libre con enfermedades de la piel tanto beningnas como cancerosas. 

También se utilizó más recientemente para su implementación en dispositivos de IOT, \cite{mnetsqueeze}, donde se logra alcanzar el 99\% de accuracy en un pequeño conjunto extraido de ISIC, haciendo uso de la versión V3 junto a un algoritmo de Squeeze. Dicho algortimo se encarga de localizar la ubicación de los pelos y otros posibles artefactos para preprocesar la imagen y lograr una fotografía resultante libre de interferencias. 

Para ello, hace uso de un filtro black hat, que binariza la imagen y obtiene los píxeles objetivo de eliminar, que son sustituidos por los colores de los píxeles adyacentes, junto al uso del aumento de datos. Sin embargo, no queda especialmente claro la tasa de precisión del modelo para cada una de las enfermedades que se intentan diagnosticar.

\subsection{Shuffle Net}

Shuffle Net \cite{zhang2017shufflenet,shufflenetreview} surge con el objetivo ofrecer un modelo capaz de ofrecer un buen modelo con la mínima pérdida de rendimiento frente a modelos profundos. Es capaz de superar en resultados a la primera versión de MobileNet, logrando un error de aproximadamente tres puntos menos que MobileNet V1. Su mejor rendimiento se debe sobre todo al uso de Channel Shuffle para las convoluciones grupales, y creando una arquitectura basada en módulos shuffle.

La convolución en grupo mediante mezcla de canales (Channel Shuffle) surge tras el estudio del funcionamiento de las convoluciones grupales en 	AlexNet\cite{NIPS2012_c399862d} y ResNext. En ambos modelos, se uitilizan convoluciones grupales, donde cada canal de salida sólo se relaciona con los canales de entrada del que proviene. Esto podría debilitar la relación entre cada canal, y debilitar los resultados; para evitarlo, y no poner en riesgo el rendimiento con demasiadas convoluciones 1x1 para relacionarlos, se hace uso del mezclado de canales; es decir, es como si permitiésemos que cada grupo convolucional pudiera obtener información de otros grupos adyacentes, para así mejorar la relación entre ellos.

Para evitar un sistema complejo a la hora de representar dichas interconexiones, es como surge el channel shuffle \ref{shufflechannels}: se mezclan los canales de forma que los grupos ya no quedan aislados con sus respectivas entradas y salidas.

    \begin{figure}[H]
	\label{shufflechannels}
	\centering
	\includegraphics[scale = 0.2]{imagenes/shufflechannels.png}
	\caption{Channel Shuffle de ShuffleNet}
\end{figure}

Esta técnica se aplicará sobre la primera y última convolución 1x1 realizada sobre los bloques residuales de la red, que siguen una estructura parecida a la que adoptaría MobileNetV2 posteriormente. Mediante esta propuesta, podemos además aplicar una mayor capacidad de procesamiento en anchura añadiendo stride, y aplicando average pooling. El resultado, es conseguir modelos más anchos en procesamiento que no impacten negativamente en el rendimiento de los dispositivos con menor capacidad. En la figura \ref{label} , podemos apreciar la arquitectura de la red, cuyos filtros pueden ser escalados mediante el parámetro s, aunque teniendo en cuenta una penalización en la complejidad, equivalente a $s^2$ sobre Shuffle Net base, equivalente a $s=1$

\begin{figure}[H]
		\label{arquitecturashuffle}
		\centering
		\includegraphics[scale = 0.2]{imagenes/arquitecturashuffle.png}
		\caption{Arquitectura de ShuffleNet}
	\end{figure}

Sus aplicaciones han sido variadas, pero en lo que respecta a la detección de lesiones cutáneas, la cercana salida de MobileNet V2 y su mejor rendimiento provocó que ShuffleNet quedase relegada a un segundo plano, y no fuese muy utilizada para este fin. Podemos encontrar algunos trabajos \cite{shuffleapp} donde podemos observar una comparativa de este modelo frente a la completitud de los modelos del estado del arte de 2022, y podemos confirmar que  MobileNetV2 es capaz de superar su rendimiento en la mayoría de pruebas, siendo estas comparaciones en cuanto a tiempo de entrenamiento, precisión, accuracy y tamaño del conjunto de entrenamiento. Sólo consigue superar a MobileNetV2 en tiempo de entrenamiento, donde es aproximadamente 900 más rápida, pero ofrece peores resultados en promedio.

\subsection{EfficientNet}

EfficientNet \cite{Chaturvedi_2020} es un conjunto de arquitecturas de redes creadas por el departamento de investigación de Google con el fin de conseguir una familia de modelos variada que fuese capaz de adaptarse fácilmente mediante parámetros a diferentes conjuntos de imágenes, y a requisitos de hardware más o menos limitados. 

Parte de que una red convolucional sigue el siguiente esquema:
$$\mathcal{N}=\odot_{i=1...s} F_i^{L_i}(X_{(H_i, W_i,L_i)})$$


Donde se denota que la capa $F_i$ es repetida $L_i$ veces la etapa i de la red, y la dimensionalidad de la capa queda representada con ${( W_i,L_i)}$.  Fijando $F_i$, efficient net intenta dar versatilidad a sus modelos variando las dimensiones restantes, Li, Ci, Hi, Wi mediante el uso de 3 constantes de escalabilidad:

\begin{itemize}
	 \item  Profundidad, Depth (d):  Aumentar la profundidad es la tendencia habitual presente en las redes convolucionales. Pero llegar a un equilibrio es crítico, ya que aumentar demasiado la profundidad sin modificar otros parámetros puede ocasionar pérdidades rendimiento por el desvanecimiento del gradiente  a menor profundidad. 
	\item Anchura, Width (w): Aumentar la anchura suele ser beneficioso para modelos de pocos recursos donde el aumento de profundidad supone un gran aumento de la carga computacional. Permite conseguir mayor cantidad de características de grado fino, pero si la red es demasiado poco profunda, el modelo careceré de características de alto grado que permitan aprender patrones generales.
	\item Resolución, (r): al emplear tamaños de entrada mayores, damos opciones a obtener una mayor cantidad de características de grado fino, pero un exceso de resolución puede provocar grandes tiempos de ejecución y puede ser contraproducente, al reducirse la ganancia con tamaños demasiados grandes.
\end{itemize}

\begin{figure}[H]
	\label{arquitecturashuffle}
	\centering
	\includegraphics[scale = 0.2]{imagenes/efnet_scale.png}
	\caption{Parámetros de EfficientNet}
\end{figure}


Experimentalmente, estos parámetros pueden ser ajustados, y dan lugar a una serie de modelos distinto: los conocidos EfficientNetB0 - B7, denotando el valor numérico la profundidad y complejidad del modelo, siendo esta mayor a mayor valor del índice.	 Cada una de ellas fue ajustada utilizando como requisito la potencia medida en TFLOPS para su ejecución, y puede ser posteriormente ajustada con el resto de parámetros libres no fijados a las características del conjunto de entrada.

En smartphones de alta gama, los modelos B0 a B4 pueden ser ejecutados con un rendimiento aceptable para aquellas aplicaciones que no requieran un alto tiempo de respuesta, pero si necesitan dar al usuario una respuesta aceptable. Para modelos de mayor complejidad computacional, se usan las variantes lite, que estudiaremos más adelante.


\subsubsection{Aplicaciones en dermatología y cáncer de piel}

En el problema que nos concierne, esta arquitectura ha conseguido grandes resultados en el dataset del ISIC abierto al público como competición en la plataforma Kaggle, habiendo sido utilizado como parte de un ensemble de modelos, o bien como modelo único entrenado en el top 3 de ganadores de la competición. En el caso de la segunda mejor solución clasificada \cite{2ndISIC}, se menciona la utilización de EfficientNet-B6, con tamaño de entrada de 512x512, y un tamaño de batch de 64, obteniendo 0.9485 de accuracy como resultado final a la hora de emplear los datasets ISIC 2019 y 2020\\ En la primera solución, es usada en conjunto a Resnet50, y una red especializada en los metadatos de la imagen, y todos los modelos juntos someten su resultado a votación \cite{1stISIC}.

Ambos resultados han sido evaluados con computadores de alta gama, haciendo uso de múltiples tarjetas gráficas para el entreanmiento y la inferencia. Este proceso es demasiado pesado para un dispositivo móvil, por lo que de cara a este trabajo, se buscarán alternativas capaces de ahorrar en espacio y potencia como concepto de cuantización.

 \section{Cuantización de modelos}

Las arquitecturas y soluciones propuestas por la literatura que han sido analizadas en los apartados anteriores proponen métodos que ofrecen buenos resultados. Sin embargo, existe un problema: todas ellas han sido entrenadas con un computador cuya capacidad de cálculo supera incluso las características de un ordenador doméstico promedio, como en \cite{2ndISIC}, donde se emplean 4  GPU Nvidia Quadro RTX 6000 24GB. Aunque su utilización engloba sobre todo el proceso de entrenamiento, la inferencia de estos modelos también sigue siendo extremadamente costosa para un dispositivo móvil, y este proceso también es realizado a través del computador.

Esto resume a que el teléfono simplemente adquiere el papel de cliente dentro de una arquitectura cliente-servidor, donde el dispositivo host de todo el procesamiento de la imagen y de su clasificación es un ordenador de gran potencia de cálculo, y el teléfono movil únicamente debe compartir con este la imagen que desea examinar. Pero esto supone una gran desventaja: en ausencia de conexión de red, o interrupción del servicio por parte del servidor, el usuario no sería capaz de emplear la aplicación para el diagnóstico. La dificultad e interés de este proyecto se ve reforzado por este argumento, y resulta ahora de interés la posibilidad de realizar la inferencia en el teléfono, a pesar de que el entrenamiento sea realizado en un ordenador.

Existe una técnica que nos permitirá obtener un modelo optimizado a partir de uno entrenado de forma tradicional: la cuantización.

\subsection{Características}

La cuantización de modelos se centra en la simplificación y optimización de un modelo preentrenado por un computador, reduciendo algunas de sus características buscando un impacto mínimo sobre los resultados obtenidos, pero reduciendo de forma considerable el tiempo de inferencia para dispositivos de baja potencia. Aunque existen mecanismos de poda, donde la arquitectura del modelo se ve simplificada de forma directa, el método que evaluaremos será la cuantización de modelos basada en la reducción de la precisión numérica, es decir, una disminución en la profundidad en bits de la variable flotante.

Este concepto ya fue brevemente mencionado durante el desglose de características de MobileNet V3 \cite{howard2019searching}. En la arquitectura AMD64, el almacenado de variables en coma flotante emplea una precisión de 32 bits. Por tanto, al realizar operaciones de cualquier tipo, las 32 posiciones del nuevo número han de ser actualizadas al valor resultado. El tiempo empleado en realizar dicha operación en cada dígito de este número puede parecer despreciable, pero, cuando el conteo de operaciones alcanza los miles de millones, supone una diferencia significativa en el rendimiento. 

En dispositivos de baja potencia, esta precisión suele verse reducida a 8 bits, de forma que reducimos la longitud del máximo número almacenable en 4 veces menos, y reducimos también así el tiempo por operación.

Este mecanismo es empleado tanto por los frameworks de trabajo habituales para aprendizaje mediante redes convolucionales, como Pytorch y TensorFlow, y por los propios fabricantes de teléfonos móviles de forma nativa. Es el caso de Qualcomm, conocido por su gama de procesadores Snapdragon. Esta empresa realizó un estudio del impacto de la cuantización en los modelos \cite{kuzmin2024fp8} usando flotantes de 8 bits del precisión. Demuestran que el uso de números flotantes de 8bits en lugar de enteros con exponenciación para desplazar el punto ofrece un mayor rendimiento. Las consideraciones a la hora de defender la utilidad de la cuantización son las siguientes:

\begin{itemize}
	\item La operación más costosa realizada durante la inferencia es el producto de matrices. Para simplificar la complejidad de la operación, normalmente se suelen emplear matrices de números enteros reescalados a un nuevo intervalo de valores. Esto es, se aplica una transformación de rangos, donde una matriz $\mathbb{R}^{m x n}$ se cuantiza a una matriz $X^{(int)}$ asociado a un valor de reescalado s: $X^{(int)}=clip([\frac{X}{s}],x_{min},x_{max})$, donde el producto pasa a ser el valor redondeado más cercano al número tras aplicar la transformación de rango mediante escala, y la operación de clip asegura que el número es representable dentro de los valores establecidos para el extremo. Esto proporciona una simplificación contra la habitual notación IEEE-754 de 32-bit, FP32, donde se usa un bit de signo, 23
	de mantisa, y 8 bits de exponente.
	\item  La utilización de FP8 puede ser beneficiosa al ser capaz de encontrar un término medio entre la precisión y variabilidad de rango numérica del resultado, como se puede ver en \cite{kuzmin2024fp8}. Los resultados empíricos muestran que tras un ajuste entre el número de bits de mantisa y exponente, se logra un equilibrio adecuado usando valores cercanos a 3 bits de mantisa y 4 de exponente, y en caso de post training quantization, 5 de mantisa y 2 de exponente, en caso de que el aprendizaje se realice de forma lenta.
	\item Para mejorar estos resultados, se puede usar la técnica Quantization Aware Training (QAT), o entrenamiento preparado para cuantización, donde el objetivo es facilitar el paso a FP8 desde el proceso de entrenamiento teniendo en cuenta los rangos de valores adoptados durante el entrenamiento, como por ejemplo, controlando el diferencial mínimo que puede tomar el algoritmo de optimización. Esta operación muestra mejor resultado sobre valores enteros, pero también es capaz de mejorar los resultados obtenidos por PF8.
\end{itemize}

En los frameworks Pytorch y TensorFlow, podemos encontar estas optimizaciones en sus funcionalidades relacionadas con la optimización de modelos en post entrenamiento, con las dependencias de Pytorch Mobile y TensorFlow lite, respectivamente. Ambas aplican cuantización numérica de los tensores, y en caso de tratarse de modelos tradicionales de la literatura, existen arquitecturas simplificadas de forma profunda, con redes como EfficientNet o ResNet, que cuentan con variantes lite.

\subsection{Modelos cuantizados}

El framwork de TensorFlow lite contiene dos de los modelos más utilizados en la literatura en sus variantes lite: EfficientNet Lite y ResNetV2. Ambos han recibido un tratamiento de simplificación, que pasa por el uso de INT8 como modelo de representación numérica y la simplificación de la arquitectura, retirando algunas de las capas más resentidas en el proceso de optimización. En el caso de EfficientNet Lite, podemos encontrar un descripción detallada sobre aquellos cambios realizados para la mejora de rendimiento \cite{eflite2}:

\begin{itemize}
	\item Se hace uso de cuantización post entrenamiento mediante el algoritmo de cuantización empleado por Tensorflow lite. Esta cuantización se hace de forma dinámica, dependiendo del hardware en el que se ejecuta el modelo. Puede realizarse una cuantización de rango dinámico, donde el factor de escalado puede aplicarse de forma distinta dependiendo de la capa en la que se realice la operación; cuantización de enteros, para los dispositivos menos potentes; y 	por último, cuantización de FP16, de forma que se logra un término medio entre ambas soluciones, y se puede emplear en dispositivos con GPU de potencia considerable. Por defecto, es la opción de rango variable la utilizada.
	\item Se eliminan algunos módulos Squeeze and Excite presentes en capas intermedias, debido a su coste e imprecisión para dispositivos que no soportan gran precisión numérica.
	\item Las funciones de activación se reemplazan con RELU6, al igual que se realizó en MobileNetV3 \cite {howard2019searching} 
\end{itemize}

\begin{figure}[H]
	\label{eflite}
	\centering
	\includegraphics[scale = 0.325]{imagenes/eflite.png}
	\caption{Optimización de EfficientNet Lite}
\end{figure}

Este procedimiento variable, con múltiples posibles combinaciones, se realiza debido a la gran variedad del mercado actual. Existe una gran diversidad de  dispositivos: conviven dispositivos de bajo consumo con escasa potencia gráfica, así como terminales especializados con unidades de TPU para acelerar el cálculo neuronal. Y, de esta forma, con optimizaciones modulares, podemos priorizar la eficiencia en dispositivos con poca capacidad de cálculo , y aumentar la precisión en aquellos que pueden ejecutarlos. En el dispositivo de ejemplo, un Google Pixel 4, se consigue un tiempo medio de inferencia de 30 ms, una mejora de casi el 60\% sobre ResNet50.

\subsection{Conclusión de los modelos cuantizados}

A la vista de los resultados, los modelos cuantizados pueden suponer una ventaja; obtenemos resultados cercanos a los modelos complejos del estado del arte, con una pequeña penalización ganada en forma de rendimiento, y que la ejecución ofrezca tiempos de respuesta razonables. Además, este tipo de transformaciones no se han empleado para el reconocimiento y detección de enfermedades de la piel, y abren una nueva linea de investigación en la que obtener resultados.

Como puntos en contra, debemos tener en cuenta que la penalización obtenida al entrenar este tipo de modelos no es siempre la misma; dependiendo del problema a clasificar, y del modelo seleccionado, la penalización de la cuantización puede ser o no más marcada; para ello, es necesario obtener empiricamente resultados que nos permitan conocer el tipo adecuado para nuestro problema. Este conflicto para encontrar el equilibrio ya fue detectado por los desarrolladores de Google en EfficientNet Lite \cite{eflite,eflite2}, donde inicialmente, sus pruebas en ImageNet arrojaron resultados pésimos al obtener un 46\% de accuracy en clasificación TOP1 a diferencia del 75.1\% obtenido con FP32. Pero, tras ajustar el rango de amplitud de los enteros utilizados para cuantización, se consiguió una ganancia de 1.85x sobre el modelo original con tan solo un 0.7\% de pérdida en los resultados.

\begin{figure}[H]
	\label{gananciacuant}
	\centering
	\includegraphics[scale = 0.35]{imagenes/gananciacuant.png}
	\caption{Ganancia de la cuantización en EfficientNet}
\end{figure}

Como aspecto negativo, aunque la varianza de los resultados pueda verse paliada mediante la selección de la cuantización más adecuada, aún existen riesgos de resultados inesperados. Con este mismo modelo, por ejemplo, se han alcanzado resultados inesperados, como que los modelos de menor tamaño (B0,B1,B2) poseen un mejor desempeño que B3 y B4 en datasets de propósito general, como podemos observar en el artículo de Agarwal  \cite{}, pero esto puede deberse por la necesidad de una mayor cantidad de datos para realizar el entrenamiento.

Dado a los buenos resultados en general, la cuantización será la senda seguida por este TFG a la hora de optimizar los modelos de escritorio en post entrenamiento.



\section{Recursos gráficos disponibles}

La obtención de datos es un proceso fundamental en la resolución de problemas de Machine Learning. Este tipo de problemas requieren un gran número de imágenes que aporten variedad, y permitan construir un modelo general que se capaz de adaptarse a cambios de iluminación, diferentes puntos de vista y composiciones.

Es clave, por tanto, disponer de diferentes tipos de lesiones, tanto benignas como malignas, así como diferentes tonos de piel. La inexistencia de un tipo de piel en el conjunto de entrenamiento, o la inexistencia de un tipo de lesión podrían provocar resultados sesgados indeseados durante la predicción de la imagen tomada.

Podemos encontrar en la red varios datasets de acceso público que permiten su utilización de forma abierta con fines académicos. Dada a la gran cantidad de publicaciones disponibles, resulta complejo averiguar si los datos a los cuales hace referencia se encuentran disponibles públicamente, si son de acceso restringido, o bien, ya no se encuentran disponibles debido a cambios en su política o la falta de mantenimiento.

Debido a que el estudio de la evolución y el diagnóstico del cáncer de piel de forma temprana es un tema en auge, existen gran cantidad de publicaciones especializadas únicamente en el análisis de los conjuntos de datos públicamente accesibles, como es el caso de la lista propuesta por M. Goyal et Al. [16], o el reciente estudio realizado por Sana Nazari y Rafael García (2023)[30].
Basados en su modelo de estudio y las referencias recomendadas por sus artículos, se ha elaborado el siguiente plan de búsqueda para saber qué tipo de datos utilizar y cuáles descartar:

\begin{figure}[H]
	\centering
	\includegraphics[scale=0.75]{imagenes/DiagramaBusqueda.png}
	\caption{Proceso de búsqueda seguido}
\end{figure}

Siguiendo dicho procedimiento, se han encontrado 6 datasets diferentes, cuyo origen son instituciones públicas que han cedido datos con fines académicos y de investigación.\\

Un factor determinante para la elección de estos conjuntos es la diversidad. Es indispensable, para este proyecto, encontrar datos lo suficientemente variados como para distinguir lesiones cancerígenas y no cancerígenas, y disponer de diferentes tonos de piel para entrenar. Aunque los tonos de piel más oscuras sufren lesiones de tipo cancerígeno en menor proporción gracias a su protección natural, también pueden sufrir este tipo de patologías, y es clave ser capaces de detectarlas en cualquier posible paciente.

\subsubsection{ISIC Data}
El repositorio ISIC (International Skin Imaging Collaboration [1]), contiene imágenes demoscópicas de lesiones principalmente cancerosas.  Existe una gran cantidad de publicaciones acerca de este conjunto de datos, debido a su utilización anual durante los años 2016-2020 para la realización de un reto virtual de Machine Learning descrito en [1]. El objetivo, consiste en identificar los melanomas frente a lesiones no cancerosas (ISIC Challenge-2020[2]) o bien, identificar diferentes subtipos de lesiones cancerosas frente a lesiones benignas (ISIC Challenge 2019, [3]). \\

En el estado del arte, destacan las soluciones que hacen uso de métodos de DeepLearning, como el planteado por Ian Pan [4], finalista de la competición para el año 2020. El ganador de la competición del año 2020, cuyo análisis podemos encontrar en [5], realiza por su parte un enfoque híbrido entre el uso de DeepLearning para la clasificación de los datos de tipo imagen, y clasificación mediante Machine Learning para los metadatos, y así obtener un resultado más preciso. \\

Debido a que cada uno de estos subconjuntos de datos anuales podían ser pequeños, normalmente se recurría a la reutilización de los conjuntos anteriores para enriquecer el conjunto de entrenamiento.  Este método es bastante recomendable, ya que, a mayor conjunto de entrenamiento, más cercanos se encontrarán los parámetros de interés del problema general a solucionar. Sin embargo, hay que realizar dicha fusión con especial cuidado, ya que existen volúmenes de datos considerables que se repitieron en las competiciones de cada año para aumentar el tamaño del dataset, y si se realizase una simple concatenación de los datos, estaríamos desperdiciando esfuerzo computacional en clasificar imágenes redundantes (comportamiento para nada deseable al trabajar sobre entornos móviles de menor potencia.).\\

Un análisis extenso de los datos asociados a cada Challenge podemos encontrarlo en [6]. En él, se recogen otros modelos del estado del arte utilizables para este fin, así como una forma de tratar los datos duplicados. Los datos pueden ser separados en los siguientes subconjuntos:

\begin{table}[H]
	\centering
	\begin{tabular}{|l|l|l|l|l|}
		\hline
		\textbf{Challenge Dataset Year} & \textbf{Train} & \textbf{Test} & \textbf{Total} & \textbf{Tipo de problema } \\ \hline
		ISIC 2016 & 900 & 379 & 1279 & Clas. binaria  \\ \hline
		ISIC 2017 & 2000 & 600 & 2600 & Clas. Multiclase  \\ \hline
		ISIC 2018 & 10015 & 1512 & 11527 & Clas. Multiclase  \\ \hline
		ISIC 2019 & 25331 & 8238 & 33569 & Clas. Multiclase  \\ \hline
		ISIC 2020 & 33126 & 10982 & 44108 & Clas. Binaria  \\ \hline
	\end{tabular}
\end{table}

\begin{itemize}
	
	\item ISIC 2016 [7]:  Es el dataset de menor tamaño de todos los propuestos. Hace distinción únicamente de los casos malignos y benignos. Contiene imágenes dermoscópicas anotadas con información acerca de la localización de la mancha, y la edad del paciente. Contiene información adicional para la segmentación de la mancha pigmentada de interés (máscaras).
	\item ISIC 2017 [8] Es un conjunto de mayor tamaño al anterior, y hace alusión a 4 clases diferentes: melanomas, nevus, y seborrheic keratosis. Contiene también información acerca de la edad del paciente, y otros metadatos de interés. La escasa cantidad de datos provoca que normalmente en la literatura este dataset se utilice también como clasificación binaria entre nevus y keratosis, u otros enfoques similares. 
	\item ISIC 2018 [9]. Este dataset contiene un número de imágenes considerables, siendo un total de 10015 imágenes para entrenamiento, y 1512 para test. En este caso, se realiza subclasificación de tipos, a través de las clases melanocytic nevus, basal cell carcinoma, actinic keratosis, benign keratosis, dermatofibroma y lesiones vasculares. Es de especial interés destacar que este dataset proviene, a su vez, de HAM10000 (Human against machine, [11] )  y MSK Dataset [12]. El challenge original comprendía, de nuevo, la clasificación de los diferentes tipos realizando previamente una discriminación de la mancha en cuestión mediante segmentación. Existen gran cantidad de publicaciones que tratan este conjunto de datos, como [13], donde se emplea este dataset para demostrar mejores resultados al emplear transformaciones polares de la imagen y aumentar la invarianza.
	
	\item ISIC 2019 [2].  Se trata del mayor conjunto de datos para clasificación multiclase propuesto por ISIC [1]. Se trata del mismo dataset que el año 2018 [9], con la adición de BCN\_20000 Dataset [14], cuyos datos provienen del Hospital Clínic de Barcelona [14]. Las clases a clasificar se amplían hasta 9, encontrando subtipos de melanomas en el conjunto.
	\item ISIC 2020 [3]. El último dataset propuesto públicamente, contiene únicamente datos binarios acordes a melanomas y no malignos. 
\end{itemize}

Todos estos datos pueden ser acoplados entre sí para dar un dataset global de ISIC [6], donde obtendríamos las siguientes clases: 


\begin{table}[H]
	\centering
	\begin{tabular}{|c|c|c|c|c|}
		\hline
		\textbf{Clase} & \textbf{2017} & \textbf{2018} & \textbf{2019} & \textbf{2020} \\ \hline
		\textbf{Melanoma} & 374 & 1113 & 4522 & 584 \\ \hline
		\textbf{Atypical melanocytic proliferation} & - & - & - & 1 \\ \hline
		\textbf{Cafe-au-lait macule} & - & - & - & 1 \\ \hline
		\textbf{Lentigo NOS} & - & - & - & 44 \\ \hline
		\textbf{Lichenoid keratosis} & - & - & - & 37 \\ \hline
		\textbf{Nevus} & - & - & - & 5193 \\ \hline
		\textbf{Seborrheic keratosis} & 254 & - & - & 135 \\ \hline
		\textbf{Solar lentigo} & - & - & - & 7 \\ \hline
		\textbf{Melanocytic nevus} & - & 6705 & 12.875 & - \\ \hline
		\textbf{Basal cell carcinoma} & - & 514 & 3323 & - \\ \hline
		\textbf{Actinic keratosis} & - & 327 & 867 & - \\ \hline
		\textbf{Benign keratosis} & - & 1099 & 2624 & - \\ \hline
		\textbf{Dermatofibroma} & - & 115 & 239 & - \\ \hline
		\textbf{Vascular lesion} & - & 142 & 253 & - \\ \hline
		\textbf{Squamous cell carcinoma} & - & - & 628 & - \\ \hline
		\textbf{Other / Unknown} & 1372 & - & - & 27.124 \\ \hline
		\textbf{Total} & 2000 & 10.015 & 25.331 & 33.126 \\ \hline
	\end{tabular}
\end{table}

Sin embargo, sería necesario tener en cuenta la eliminación de imágenes repetidas, debido a que durante cada edición de ISIC, un número considerable de imágenes han sido incluidos en varios años. Este procedimiento engloba:

\begin{enumerate}
	
	
	\item Eliminar las imágenes idénticas por hash. Todas las imágenes de ISIC están numeradas de forma única para facilitar la identificación de cada una de ellas. Si unimos todos los datatesets, y tomamos las repeticiones, podemos remover:
	
	\begin{table}[H]
		\centering
		\begin{tabular}{|c|c|c|c|c|c|}
			\hline
			\textbf{} & \textbf{2016} & \textbf{2017} & \textbf{2018} & \textbf{2019} & \textbf{2020} \\ \hline
			\textbf{Train} & 291 & 1283 & 0 & 0 & 0 \\ \hline
			\textbf{Test} & 95 & 594 & 0 & 0 & 0 \\ \hline
		\end{tabular}
		\caption{Número de imágenes duplicadas recogidas por [6]}
	\end{table}
	
	
	\item 	Eliminación del ISIC 2018. Como éste se encuentra contenido en la composición para el año 2019, puede prescindirse totalmente de él a favor de la versión de 2019.
	\item 	Eliminación de imágenes “downsampled” del conjunto. En los años 2019 y 2020, se añadieron imágenes de challenges anteriores con una reducción en resolución. Para ahorar en espacio y tiempo de cómputo, pueden eliminarse las imágenes reducidas para quedarnos con una única copia de mayor calidad de la lesión, y luego realizarles manualmente un reescalado en caso de que sea necesario.
	
	Atendiendo de nuevo a los resultados propuestos por [6], obtenemos el siguiente conjunto: 
	
	\begin{table}[H]
		\centering
		\begin{tabular}{|c|c|c|c|}
			\hline
			\textbf{Year} & \textbf{Task No.} & \textbf{Images Removed} & \textbf{Images Remaining} \\ \hline
			\textbf{2016} & 3 & 826 & 74 \\ \hline
			\textbf{2017} & 3 & 801 & 1199 \\ \hline
			\textbf{2018} & 3 & 10,015 & 0 \\ \hline
			\textbf{2019} & 1 & 2235 & 23,096 \\ \hline
			\textbf{2020} & - & 433 & 32,693 \\ \hline
			\textbf{Total} & - & 14,310 & 57,0621 \\ \hline
		\end{tabular}
		\caption{Tabla de imágenes únicas extraída de [6]. En este caso, el autor descarta el uso del dataset de 2016 por su baja aportación}
	\end{table}
	
\end{enumerate}

Obtendríamos un total de 57000 imágenes, los cuales podrían clasificarse, con sus respectivas clases extraídas de los metadatos. Componen, en resumen, un conjunto de datos robusto que puede formar parte del dataset de entrenamiento de este trabajo.
\begin{figure}[H]
	\centering
	\includegraphics[scale = 0.5]{imagenes/Ejemplo2020.png}
	\caption{Ejemplo de imágenes de ISIC 2017 [23]}
	\label{fig:enter-label}
\end{figure}

\subsubsection{ASAN Dataset}

ASAN (Seung Seog Han 2018)[15][18][19] es un conjunto de datos de origen surcoreano compuesto por lesiones malignas y benignas de la piel. Nos permite obtener un mayor grado de variedad de las imágenes, ya que el repositorio ISIC se centra sobre todo en lesiones de piel de población europea. 

\begin{table}[H]
	\centering
	\begin{tabular}{|c|c|}
		\hline
		\textbf{Tipo de lesión} & \textbf{Número de ejemplares} \\ \hline
		{Actinic keratoses and intraepithelial carcinoma (AKIEC)} & 651 \\ \hline
		{Basal Cell Carcinoma (BCC)} & 1082 \\ \hline
		{Dermatofibroma (DF)} & 1247 \\ \hline
		{Hereditary angioedema (HAO)} & 2715 \\ \hline
		{Intraepithelial Carcinoma (IC)} & 918 \\ \hline
		{Lentigo (LEN)} & 1193 \\ \hline
		{Melanoma (ML)} & 599 \\ \hline
		{Nevus (NV)} & 2706 \\ \hline
		{Pyogenic Granuloma (PG)}  & 375 \\ \hline
		{Squamous Cell Carcinoma (SCC)} & 1231 \\ \hline
		{Seborrhoeic Keratosis (SK)}  & 1423 \\ \hline
		{Wart} & 2985 \\ \hline
		\textbf{Total} & \textbf{17125} \\ \hline
	\end{tabular}
	\caption{Distribución de clases de ASAN dataset}
\end{table}

Tal y como se describe en [17] (M Goyal 2019), este dataset tiene 12 tipos de enfermedades, sumando un total de 17125 imágenes clínicas. Estas imágenes están compuestas en su mayoría por imágenes en miniatura, pero existe un repositorio con imágenes de mayor tamaño, donde sería necesario realizar tareas de segmentación- Sin embargo, dichas imágenes son de acceso restringido, y se requieren permisos especiales del hospital para acceder a ellos. Por ese motivo, tendremos únicamente en cuenta loas 17125 miniaturas.

Adicionalmente, podemos encontrar también imágenes proporcionadas por Hallym, un dataset complementario de 125 imágenes pertenecientes a lesiones de tipo melanoma cancerosas.

Las clases más destacadas de este dataset en su conjunto son la presencia de lesiones benignas de la piel, detalle que no encontramos en ISIC, y que permiten así contrastar información de la piel con lesiones benignas con la piel cancerosa. Podemos encontrar 4 clases benignas: lentigos (manchas solares fruto del envejecimiento y la exposición prolongada al sol), nevus (lunares comunes), verrugas y granulomas benignos. 

\begin{figure}[H]
	\centering
	\includegraphics[scale = 0.6]{imagenes/ASAN.png}
	\caption{Ejemplo de lunares beningnos en ASAN (Nevus)}
\end{figure}

Los resultados han sido confirmados por expertos dermatólogos y los resultados verificados en su mayoría mediante biopsia, por lo que las etiquetas asociadas a cada lesión están completamente verificadas.\\

El formato de las imágenes es una disposición matricial de las miniaturas, donde cada fichero que contiene las subimágenes representa en su conjunto una clase. Por desgracia, no se aporta otro tipo de información adicional más allá de la etiqueta por motivos de privacidad.

\subsubsection{Dermnetz}

Podemos encontrar extraer este dataset de un atlas online de enfermedades cutáneas recogidas de pacientes alrededor de todo el mundo. Contiene tanto lesiones benignas como malignas, existiendo además manchas y lesiones vinculadas a enfermedades infecciosas y hongos. 

Existen gran cantidad de herramientas para realizar esta extracción de datos, como la que podemos encontrar en [21]. En total, se pueden obtener hasta 23000 imágenes, existiendo un total de 23 clases no balanceadas. Podemos encontrar lesiones de tipo alérgico, así como acné, dermatitis severa o celulitis.

Carece de metadatos asociados, ya que dicha información es de carácter reservado por su mantenedor.  El dataset no se encuentra listo para usar de forma inmediata, ya que desde 2019, las imágenes deben ser extraídas de la propia web, pues dispone un índice donde se pueden acceder a las enfermedades de interés. El fichero contenedor del dataset fue retirado en 2019 del libre acceso, junto a sus metadatos. Es necesario solicitar su acceso y aportar una cantidad económica.


\subsubsection{ PH2}
El conjunto de datos PH2 [22] es un conjunto de 200 imágenes obtenidas gracias al hospital Pedro Hispano de Portugal. Está compuesto por imágenes de alta resolución que contienen 3 posibles casos de lesiones:
\begin{itemize}
	\item Lunar común (Common Nevus), 80 ejemplares
	\item Lunar atípico (Atypical Nevus), 80 ejemplares
	\item Melanomas, 40 ejemplares.
\end{itemize}

Además de las 200 imágenes, podemos encontrar metadatos asociados a cada una de ellas, como el color, su extensión, textura, forma del borde, localización, entre otros.
Su acceso es libre para fines académicos desde su página oficial [22], que contiene las imágenes en formato jpg, y varios ficheros .csv con la información de la imagen y su clasificación.

\begin{figure}[H]
	\centering
	\includegraphics[scale = 0.6]{imagenes/PH2.png}
	\caption{Nevus maligno y benigno en PH2}
\end{figure}

\subsubsection{PAD-UFES 20}

PAD-UFES-20 [25] se trata de un conjunto de datos recopilado de diferentes poblaciones, que contiene diagnósticos para 1.641 lesiones cutáneas únicas recopiladas, comprendiendo un total de 2.298 imágenes.\\

Entre sus clases, podemos encontrar tres enfermedades y tres cánceres de piel.  Todos estos datos han sido recogidos y verificados mediante biopsia en un 100\% de los casos cancerosos, por lo que su diagnóstico está totalmente verificado.\\

Podemos encontrar, además del diagnóstico, metadatos acerca de:
\begin{itemize}
	\item ID de paciente
	\item ID de lesión, 
	\item ID de imagen
	\item Si la lesión benigna fue o no probada por biopsia.
	\item Información del paciente: fumador o no, localización de la lesión, edad, exposición a químicos, historial cancerígeno, etc.
\end{itemize}
Los datos han sido recogidos mediante teléfonos móviles en formato PNG, siendo las imágenes validadas por el Hospital Pathological Anatomy Unit of the University Hospital Cassiano Antȳnio Moraes (HUCAM) de la Federal University of Espírito Santo (Brasil). 
En su publicación original [25], podemos encontrar un resumen de su contenido de forma más específica:

\begin{table}[!ht]
	\centering
	\begin{tabular}{|c|c|c|}
		\hline
		\textbf{Diagnostico} & \textbf{Ejemplares} & \textbf{\% biopsied} \\ \hline
		Actinic Keratosis (ACK) & 730 & 24.4\% \\ \hline
		Basal Cell Carcinoma of skin (BCC) & 845 & 100\% \\ \hline
		Malignant Melanoma (MEL) & 52 & 100\% \\ \hline
		Melanocytic Nevus of Skin (NEV) & 244 & 24.6\% \\ \hline
		Squamous Cell Carcinoma (SCC) & 192 & 100\% \\ \hline
		\textbf{Total} & \textbf{2298} & \textbf{58.4\%} \\ \hline
	\end{tabular}
	\caption{Tabla de casos diagnosticados en PAD-UFES20}
\end{table}

Donde podemos apreciar que todos los casos de enfermedades cancerígenas han sido probados mediante biopsia, y el cáncer de célula basal se trata del tipo de enfermedad más frecuente.

\begin{figure}[H]
	\centering
	\includegraphics[scale = 0.45]{imagenes/PAD-UFES.png}
	\caption{Batch de ejemplo de PAD-UFES 20 [25]}
\end{figure}

\subsubsection{Severance}
Se trata de un conjunto de imágenes de lesiones cutáneas [24] recopiladas de pacientes de Corea del Sur. Recibe dicho nombre a que los datos recopilados cuentan con la colaboración del Hospital Severance, en el mismo país. \\

En su variante A, que es la única disponible públicamente, podemos encontrar el diagnóstico y otra información asociada sobre 10426 imágenes, cuya valoración se encuentra entre las 38 posibles clases que contiene este conjunto de datos. \\

Seleccionando las 6 clases más comunes contenidas en este dataset, encontraremos que comprenden aproximadamente el 75\% del conjunto. Está compuesto por actinickeratosis (22.5\%), angiofibromas (14.4\%), angiokeratomas(13.8\%), cáncer de tipo basal cell (8.1\%), Becker nevus (7.5\%), bluenevus (6.2\%), y la enfermedad de Bowen (carcinomas)(6.1\%).

El interés en este dataset se debe a que algunas de estas clases mayoritarias, como los nevus azules y de Becker, son condiciones benignas que suelen ser retirados únicamente con fines estéticos, permitiendo complementar con el resto de los diagnósticos negativos. Este tipo de lunares son los más complejos de diagnosticar, debido a sus colores similares a un melanoma, y suelen requerir una biopsia, por lo que su diagnóstico suele alargarse.

Las imágenes se encuentran en formato matriz, por lo que es necesario proceder a su separación previo a su utilización con fines de Deep Learning: 

\begin{figure}[H]
	\centering
	\includegraphics[scale = 0.5]{imagenes/Severance.png}
	\caption{Imágenes de ejemplo provenientes del dataset Severance   }
\end{figure}

\subsubsection{Otros datasets}
Existen otros datasets ampliamente referenciados que son de acceso público. Sin embargo, en los últimos años, éstos han sido retirados y han quedado inaccesibles. Es el caso de DermQuest, un atlas virtual que contenía lesiones cutáneas y otras patologías. Ese dataset fue contenido posteriormente por Derm101, pero ambas versiones fueron retiradas para su descarga. Alternativamente, podemos encontrar algunas de sus imágenes en los datasets SD-198 y SD-260 [26][29], pero únicamente permanece en activo el primero de ellos, bajo solicitud. En total, SD-198 contiene más de 6500 imágenes, mientras que SD-260 alcanzaba las 20000 imágenes.

En el estado del arte actual, podemos encontrar otros datasets ampliamente utilizados, como el caso de DermIS [27], un atlas online de patologías de la piel. También existen publicaciones reciente sobre nuevos conjuntos de datos utilizados de uso restringido, que permiten observar que la tendencia de investigación de este campo sigue en alza; es el caso del estudio propuesto por Papadakis et Al (2021) [28], que recoge datos sobre pacientes con melanoma de grado 3 para estudiar su evolución durante un período de 3 años, para estimar su crecimiento y potencial grosor del tumor.

Debido a las restricciones de acceso, ninguno de estos datasets será empleado como parte del entrenamiento del modelo diseñado para este estudio.

\subsubsection{Conjunto resultado}
Una vez examinados todos los conjuntos mencionados anteriormente, podemos llevar a cabo la unión de todos los datos en un único subconjunto. Esto nos permitirá conseguir un dataset completo y variado con diferentes tipos de piel y diferentes lesiones que nos permitirán identificar multitud de tipos de patologías, siendo posible ajustar el grado de granularidad en función de la agrupación o no de posibles subclases.

Inicialmente, el conjunto de datos construido contendrá todos los subtipos de lesiones cutáneas vistos, pero dispondrán de una segunda etiqueta que indicará si se trata de un caso canceroso o no, atendiendo a su subclase que lo etiqueta. Si agrupamos por lesiones benignas, cancerosas, y potencialmente cancerosas, obtenemos:


\begin{figure}[H]
	\centering
	\includegraphics[scale = 0.65]{imagenes/datasetfinal.png}
	\caption{Distribución de clases}
\end{figure}

Se puede observar cómo la mayoría de imágenes disponibles engloban problemas de piel no cancerosos, mientras que el segundo tipo más común de lesión si es la cancerosa. Si atendemos a clasificar las subclases de cada tipo de patología, encontramos 52 posibles etiquetas.

%\section{Procesado de imágenes cutáneas}
%\subsection{Técnicas de reducción de ruido}
%\subsection{Normalización}
%\subsection{Extracción de características}



	
	\chapter{Preprocesado de datos}


Una vez examinados todos los conjuntos mencionados anteriormente, podemos llevar a cabo la unión de todos los datos en un único subconjunto. Esto nos permitirá conseguir un dataset completo y variado con diferentes tipos de piel y diferentes lesiones que nos permitirán identificar multitud de tipos de patologías, siendo posible ajustar el grado de granularidad en función de la agrupación o no de posibles subclases.

Inicialmente, el conjunto de datos construido contendrá todos los subtipos de lesiones cutáneas vistos, pero dispondrán de una segunda etiqueta que indicará si se trata de un caso canceroso o no, atendiendo a su subclase que lo etiqueta. Si agrupamos por lesiones benignas, cancerosas, y potencialmente cancerosas, podemos obtener el siguiente diagrama de sectores de la figura \ref{tartabinaria}, donde se puede observar cómo la mayoría de imágenes disponibles engloban problemas de piel no cancerosos, mientras que el segundo tipo más común de lesión si es la cancerosa. Si atendemos a clasificar las subclases de cada tipo de patología, encontramos 52 posibles etiquetas, las cuales iremos examinando a medida que se preprocese cada uno de los subconjuntos.


\begin{figure}[H]
	\centering
	\label {tartabinaria}
	\includegraphics[scale = 0.7]{imagenes/datasetfinal.png}
	\caption{Distribución de clases}
\end{figure}

En el caso de las lesiones potencialmente cancerosas, como se trata de condiciones de la piel no cancerosas con posible evolución a cancerosas, se tendrán en cuenta como imágenes benignas, ya que la condición de malignidad sólo podría aparecer en el futuro, el cual sigue siendo desconocido.

El preprocesado de datos es una fase indispensable para el correcto aprendizaje de los algoritmos de deep learning. Se ha demostrado empíricamente que una correcta preparación y normalización de los datos permiten hallar soluciones más cercanas a la optima que con datos no procesados.

Es importante tener en cuenta que no existe una metodología de preprocesado única, y que es necesario adaptarse al tipo de dato que estamos tratando. Para este proyecto, además, existe una dificultad adicional, y es la existencia de diferentes procedencias para los datos, pues en total se dispone de 5 datasets distintos, cada uno recopilado con diferentes metodologías e instrumentación. Por tanto, será clave adaptarse a cada uno de los destinos, y realizar la partición final de forma estratificada para evitar sesgos que perturben el resutado.

a continuación, se describe la estrategia seguida para el procesado global de los datos, y los ajustes necesarios para cada uno de los conjuntos empleados.

\section{Estrategia de preprocesado para la fusión}

En el punto de partida, antes del preprocesado, contamos con 5 datasets muy diferentes entre sí. Cada uno ha sido documentado y organizado siguiendo unos criterios no estándares que nos afectan en gran medida a la hora de emplear estos datos para el aprendizaje. Antes de proceder con el desarrollo de los modelos, debemos de estadarizarlos a un formato común para evitar que existan clases con el mismo diagnóstico que, por cuestiones de formato, se consideren etiquetadas como clases distintas, por usar criterios distintos de escritura, como ausencia de espacio, mayúsculas o one hot encoding. 

Concretamente, los datos recopilados poseen el siguiente formato de etiquetado:
\begin{itemize}
	\item ISIC: etiquetas escritas a formato completo, como nombre de carpeta, con la primera letra de la enfermedad en mayúscula.
	\item ASAN: nombres escritos en el nombre de la fotografía, haciendo uso de caracteres especiales, y de abreviaturas.
	\item Severance: etiquetas escritas en la propia imagen, la cuales habrá que etiquetar y organizar manualmente, debido a que su csv está incompleto.
	\item PH2: fichero csv, con diagnósticos en formato one hot encoding. Es decir, una fila de ceros y unos, siendo uno la clase a la que pertenece, y 0, el resto.
	\item PAD UFES 20: fichero csv, con los nombres de diagnóstico escritos en minúsculas, sin espacios.
\end{itemize}

Podemos observar la gran variedad de formatos de registro empleados, y que por tanto, es completamente obligatorio y necesario realizar una transformación para hacerlo homogéneo. En este caso, por decisión propia, he considerado adecuado realizar una transformación de las etiquetas al siguiente formato: Uso únicamente de minúsculas, con ausencia de caracteres especiales, nombres sin abreviaturas, y evitando el uso de espacios, con el uso de la barra baja como carácter sustitutivo. Para la clasificación binaria, se utilizará one hot encoding, denotando como 0 los casos negativos y como unos, los positivos.\\


De esta forma, se obtiene un fichero .csv donde encontrar los valores necesarios para entrenar los modelos. Para poder localizar cada imagen, se mantendrá el arbol de directorios por defecto de cada dataset, y se anotará su directorio en un nuevo fichero .csv, que contendrá las etiquetas estandarizadas y los nombres de los ficheros de imagen con y sin el directorio. En resumen, contará con los campos:

\begin{itemize}
	\item image: nombre la imagen, sin el path en su nombre, y con la extensión de formato
	\item dir: directorio donde se aloja la imagen, respetando la estructura de carpetas original seguida por el dataset de origen
	\item label: etiqueta con el diagnóstico de la lesión, siguiendo las pautas indicadas anteriormente
	\item dataset: columna que indica el dataset de procedencia de la imagen, por si fuese necesario utilizar solo un subconjunto de todos los datos.
	\item bin: columna para la etiqueta que indica si se trata de clase Benigna (0) o Maligna (1)
\end{itemize}

\begin{figure}[H]
	\centering
	\label {formatocsv}
	\includegraphics[scale = 0.55]{imagenes/formatocsv.png}
	\caption{Formato del fichero csv normalizado}
\end{figure}

Una vez establecido el formato común, podemos pasar al análisis y adaptación propia de cada conjunto.

\subsection{ISIC Dataset}

El dataset ISIC es el mayoritario de la lista, ya que posee casi 60.000 imágenes de alta resolución, del total de casi 108.000 imágenes de las que disponemos. Para su descarga, se han empleado la galería de la web oficial \cite{isicarchive}, donde podemos filtrar cómodamente las enfermedades que queremos descargar. Como criterio de descarga, se han tenido en cuenta únicamente aquellas fotografías correctamente diagnosticadas, ya que existe un total de 27896 imágenesno etiquetadas dentro del repositiorio, las cuales descartaremos. El problema se centrará en resolver un problema de aprendizaje supervisado, por lo que las imágenes no etiquetadas suponen una complejidad adicional y un ruido para el modelo.

Cada clase descargada, además, se ha sometido a un proceso de filtro, sobre todo por cuestiones numéricas; existen nuevas clases, con escasas cantidades de datos, las cuales poseen menos de 10 imágenes, cantidad insuficiente a la hora de clasificar frente a clases como lunares comunes, que tienen en total 32697 ejemplares. De esta forma, nos queda el siguiente conjunto de clases:
\begin{itemize}
	\item Nevus
	\item Seborreic keratosis            
	 \item Actinic keratosis            
	\item Pigmented benign keratosis         
	\item Solar lentigo                           
	\item Dermatofibroma                          
	\item Vascular lesion                         
	\item LIchenoid keratosis                     
	\item Acrochordon                             
	\item Lentigo NOS                             
	\item Atypical melanocytic proliferation       
	\item Aimp                                     
	\item 	Wart                                     
	\item Angioma                                  
	\item Lentigo simplex                          
	\item 	Neurofibroma                             
	 \item 	Scar 
\end{itemize}

Estas clases se almacenan en ficheros zip cada una, así que tras ser descargadas, deben ser extraídas y añadidas al fichero csv que definimos anteriormente. Al tratarase del primer subconjunto que se añadirá, será la parte del código encargada de crear el fichero y establecer las columnas mencionadas. Además, se realizará la transformación de las etiquetas, dispuestas en el formato de la enumeración anterior, a notación snake case. Numerando el proceso, se ha creado un script de python que realiza las siguientes tareas:

\begin{enumerate}
	\item Extraer las imágenes mediante uzip en una carpeta con el mismo nombre de la clase a la que pertenecen+
	\item Crear un fichero .csv, denominado preprocessedData.csv, donde se alojarán las 5 columnas: images, dir, label, dataset, bin.
	\item Recorrer cada carpeta creada, y añadir los 4 primeros campos
	\item Una vez añadidas todas las imágenes, se renombran las etiquetas a camel case mediante las funciones upper(), lower() y replace() de la clase string de python.
\end{enumerate}

Para facilitar el procesado, el rellenado de los datos se realiza sobre una estructura tabular de pandas, para así transformar la columna label fácilmente.
En cuanto a la quinta columna, la clase binaria, dicha tarea se realizará cuando todos los datasets estén añadidos al .csv, de forma que el recorrido de los datos sólo se realice una vez, cuando tengamos disponible todas las clases. 

En cuanto al estudio estadístico de los datos, este se realizará una vez dividido los datos en los conjuntos de entrenamiento y test, definido en entradas posteriores.

\subsection{ASAN}

ASAN es uno de los dos datasets cuyo formato de entrega de los datos consistía en una matriz de imágenes en un canvas de gran resolución. En el caso de este dataset, tenemos un total de 32 imágenes de este tipo, cuya etiqueta se encuentra escrita en el nombre del fichero.
El procesado de este dataset será más complejo que el anterior, ya que debemos recortar cada una de las imágenes, evitando que queden bordes blancos que puedan perturbar la predicción, y sesgar el aprendizaje.

Podríamos idear una solución codificada de forma estricta en la cual la imagen se subdivida en n filas y m columnas para extraer las fotografías; sin embargo, cada uno de los canvas del datasets tiene un número filas y columnas concreto que dificultaría esta tarea de forma automática. En su lugar, se ha medido mediante una herramienta de recorte fotográfico el tamaño de una de las miniaturas, siendo este de 98 píxeles, y será el valor que utilizaremos a la hora de realizar el recortado.

No se debe pasar por alto que las imágenes se encuentran separadas vertical y horizontalmente por espacios en blanco, cuyo grosor es de 8 píxeles entre imagen, y 12 en los bordes exteriores. El proceso que debemos seguir para extraer las miniaturas es:

\begin{enumerate}
	\item Eliminar 4 píxeles en blanco de los extremos para que todas las bandas blancas queden del mismo grosor
	\item Hallar el numero de imágenes por fila y columana teniendo en cuenta el tamaño de miniatura y el borde.
	\item Recortar la imagen usando el método findContours() de openCV. Este método binaria la imagen transformándola a blanco y negro, y trazado con técnicas de detección de puntos de interés en imágenes los bordes de cada una de las miniaturas, y devolviendo las coordenadas de sus equinas en un vector multidimensional. 
	\item Para cada imagen, obtenemos la esquina superior izquierda de la imagen, y mediante el ancho y alto de la imagen, recortamos dicha sección de la imagen y se almacena en una nueva variable.
	\item Se recortan los bordes de dicha imagen y se almacena el resultado en disco, en una carpeta que posee el mismo nombre que la imagen de la que fue extraída.
	\item Se repite el paso 3-6 para cada imagen de la matriz, pasando a abrir la siguiente matriz hasta que no quede ninguna por recortar.
\end{enumerate}

Una vez finalizado el proceso, el proceso a aplicar es similar a ISIC; pero, en este caso, en lugar de simplemente convertir a camelcase, debemos de cambiar los nombres por completo para no usar el formato por abreviaturas original, y poder hacer merge de las clases de este dataset con ISIC que sean de la misma enfermedad. Para ello, simplemente se crea un diccionario clave-valor, donde la clave es el nombre que deseamos cambiar, y el valor, el nuevo nombre. Mediante pandas, el proceso de sustitución se puede hacer de forma inmediata mediante la función replace.

Es importante destacar que el dataset Hallym, también será incluido en el conjunto final, siendo el procedimiento de preprocesado a aplicar exactamente el mismo al descrito en este punto.

\subsection{PAD UFES 20}

PAD UFES 20, como ya describimos en el apartado de Estado del arte, se trata de un dataset diseñado para el entramiento de sistemas de asistencia en diagnóstico computado, donde el experto dermatólogo puede utilizarlo como un medio de apoyo. Contiene 6 enfermedades distintas, siendo 3 cancerosas (células basales, células escamosas o melanoma maligno) y 3 benignas (actinic keratosis, nevus, keratosis seborreica).

La estructura de presentación de los datos es más sencilla que ASAN, pues las imágenes son individuales, y cuentan con un fichero .csv donde se describen las etiquetas y otros metadatos asociados a las imágenes. La única modificación necesaria es actualizar el path de cada imagen y la nomenclatura del diagnóstico de la enfermedad, teniendo en cuenta que debemos de tranformar de abreveviatura a camel case:
\begin{itemize}
	\item NEV $\rightarrow$ nevus
	\item SEK  $\rightarrow$ seborreic\_keratosis 
	\item ACK $\rightarrow$ actinic\_keratosis  
	\item BCC $\rightarrow$  basal\_cell\_carcinoma           
	\item SCC $\rightarrow$ squamous\_cell\_carcinoma
	\item MEL $\rightarrow$ melanoma
	         
\end{itemize}

De esta forma, podemos simplemente hacer fusión de las nuevas filas con el fichero anterior en modo de apertura ``append''.


	
	\chapter{Deep learning: modelos y entorno de trabajo}

Una vez disponemos de los datos correctamente preprocesados, podemos pasar a la fase de construcción del modelo. En este proyecto, se usará la cuantización de modelos tras el entrenamiento (Post Training Quantization), por lo que el diseño del modelo , su arquitectura y su forma de entrenamiento se realizan de la forma habitual, con la particularidad del proceso posterior. Sin embargo, existe una serie de factores a tener en cuenta durante la creación del modelo:

\begin{itemize}
	\item La memoria y potencia del dispositivo móvil son limitadas. Aunque la cuantización consigue reducir el tiempo de inferencia y el espacio ocupado por el modelo, la complejidad del mismo sigue siendo proporcional, por lo que un modelo excesivamente profundo o pesado a nivel espacial puede suponer un problema.
	\item La resolución de las imágenes de entrada es variable, ya que depende del terminal en el que se ejecute y su cámara. Es posible que se requiera un reescalado de la entrada (resize) o la posibilidad de disponer de un tamaño de entrada sin fijar. Esto es posible mediante la no utilización de capas totalmente conectadas, o de pooling.
\end{itemize}

En los siguientes apartados, entraremos en detalle en cada uno de estos aspectos, y describiremos los modelos a probar: MobileNet, y las familias de redes EfficientNet y ResNet.

\section{Conceptos previos}

Habitualmente, para el estudio y creación de modelos cuya información de entrada son imágenes, se hace uso de redes convolucionales. Estas redes, a diferencia de las redes neuronales habituales, donde tenemos una serie de neuronas conectadas entre sí formando distintas capas,  tenemos un conjunto de capas de procesado local, que permiten aplicar transformaciones sobre las imágenes para simplificar su estructura y destacar elementos característicos que permitan obtener características útiles y distintivas para realizar clasificación de la información que muestran.

Los pesos que se aprenden son, precisamente, los coeficientes y parámetros de transformaciones (convoluciones) a realizar sobre la imagen , con la peculiaridad de que estos se aprenden automáticamente dentro de la red, y no es necesario de especificarlos manualmente. La propia red, en base a su función de pérdida, intentará maximizar los resultados entre el valor predicho y real, y encontrará los parámetros más adecuados. Únicamente, debemos indificar las capas que compondrán la red, y la dimensionalidad de entrada y salida para que sean correctamente interconectables.

Por tanto, el nombre convolucional proviene de la capacidad de aprender los valores para los filtros que queremos aplicar para destacar propiedades. Para una convolución, exiten las siguientes modificaciones y variantes:

\begin{itemize}

	\item  Padding. Consiste en el relleno de bordes auxiliares en la imagen para obtener una imagen resultante con la misma dimensionalidad que la entrada de una capa. Habitualmente, dicho relleno se realiza con ceros (zero-padding).
	\item  Stride. Indica el desplazamiento que realiza el filtro sobre la imagen. por defecto, dicho valor es 1, por lo que la imagen se recorre píxel a píxel. Pero puede ser aumentado para reducir la dimensionalidad del problema.
	\item Dilation: espacio entre los valores con los que opera el kernel. Por defecto, su valor es 1, es decir, los kernels operan con los valores adyacentes al píxel sobre el que se trabaja, pero puede ser una distancia mayor. Por ejemplo, con padding de 2, el área de influencia de los píxeles del entorno alejados dos píxeles de su centro adquieren mayor presencia en el resultado de la convolución.

\end{itemize}



Habitualmente, encontramos en ellas dos tipos de capas para el aprendizaje de pesos. Se encuentran en la mayoría de modelos del estado del arte:

\begin{itemize}
	\item Capa totalmente conectada. Son las capas hasta ahora vistas en Aprendizaje automático, donde cada neurona de un nivel está conectada a todas las del siguiente. Son costosas computacionalmente, y requieren un formato de entrada concreto, especificando de antemano su tamaño de entrada. En este problema, las imágenes tomadas con el teléfono movil pueden variar en tamaño, debido al recorte de las imágenes o a la calidad variable de la cámara que la capta. Supone un problema, el cual que podemos solucionar mediante una capa previa de Average Pooling: es una operación convolucional capaz de reducir la dimensionalidad de la imagen de entrada, aprendiendo de la información resumida para extrear de características. Es una de las capas que habitualmente se usa cuando tomamos modelos ya entrenados sobre un tamaño de entrada distinto al que requerimos usar.
	\item Capa convolucional. Son el resultado de un dot-product sobre una región concreta del volumen de entrada, operando a nivel local. Gracias a la localidad, no es necesario que cada neurona está conectada a todas del siguiente nivel, por lo que se produce un ahorro computacional considerable. En resumen: se trata de una matriz filtro que se desplaza a lo largo y alto de la imagen, realizando una operación de producto.
\end{itemize}

En las redes neuronales, nos interesa procesar los datos con capas convolucionales que estimen los parámetros para las traformaciones adecuadas de la imagen, mientras que las capas fully-connected serán más útiles en capas finales, ya que al estar conectadas totalmente, tienen acceso a todos los valores de entrada, y permitirán obtener de forma condensada los resultados para la clasificación de las imágenes. Es recomendable el uso esta capa cuando la imagen es suficientemente reducida por las convoluciones.\\

El proceso de entrenamiento de estas redes es generalmente costoso, debido al entrenamiento de una gran cantidad de parámetros, que habitualmente superan los varios millones. Normalmente, los modelos convolucionales empleados para la resolución de problemas suelen ser preentrenados, modelos ya configurados entrenados con millones de datos durante períodos de tiempo prolongados, de forma que adquieren capacidades generales de filtro para un espectro muy amplio de imágenes.\\

Cuando se requiere su aplicación para una tarea específica, que en este caso se trata de la piel, basta con realizar un ajuste de los parámetros mediante el entrenamiento del modelo durante una serie de epocas reducidas: a esto se le conoce como transfer learning, y será la técnica que aplicaremos para la creación de los modelos mediante MobileNet, EfficientNet y Resnet.

\section{Modelos preentrenados escogidos}

A cotinuación, se detallan lar arquitecturas seleccionadas para el entrenamiento del modelo, teniendo en cuenta aquellos modelos del estado del arte que no fueron existosos. Los 3 modelos serán sometidos a procesos de entrenamiento similares, de forma que el de mejor desempeño de los 3 muestre será el elegido para ser parte de la aplicación móvil.\\

Al tratarse de modelos ya entrenados con un dataset de gran tamaño, es necesario adaptar su salida para que ésta se adecúe a nuestras necesidades; dicha información la veremos posteriormente, ya que la capa de adaptación será muy similar para los 3 modelos seleccionados.

\subsection{ResNet 50}

La familia de redes preentrenadas Resnet \cite{he2015deep} es bastante amplia. Dispone de sus versiones 18, 34, 50, 101 y 153. Cada una de estas redes hace uso de la misma configuración de capas, pero replicando esta con mayor o menor profundidad. Para nuestro problema, las versiones de 18 y 153 quedan descartadas, ya que 18 unidades de profundidad son insuficientes para la complejidad y variedad de nuestro problema. No serían capas suficientes para extraer todas las características necesarias del entrenamiento.

En cuanto a Resnet153, su excesiva profunidad requiere grandes cantidades de datos para ajustar las capas más profunda de la red, ya que tras cada capa supera, el gradiente que regula la optimización del modelo hacia el óptimo decae, y provoca que los ajuste de las últimas capas sean mínimos. A este fenómemo se le llama desfallecimiento de gradientes, y es un suceso bastante común en redes profundas como esta.

A pesar de que Resnet se caracteriza por la aplicación de conexiones residuales, que permiten interconectar distintas capas distanciadas entre sí sin necesidad de transcurrir sobre las unidades intermedias, la cantidad de datos requerida, y el tiempo de inferencia necesario para el modelo en un dispositivo móvil son excesivos.\\

En la literatura, se encuentran algunos casos de utilización de esta arquitectura, pero no se tienen datos de la aplicación de este modelo sobre teléfonos móviles mediante cuantización. Teniendo en cuenta este factor, considero este modelo como parte de los candidatos.


\begin{table}[H]
	\centering
	\label{fig:tablaresnet}
	\begin{tabular}{|c|c|c|c|c|c|c|}
		\hline
		\textbf{Model} & \textbf{ Input} & \textbf {Size (model)} & \textbf{Size (feat.)} & \textbf{Est. FLOPS} \\ \hline
		resnet18 & 224 x 224 & 45 MB & 23 MB & 2 GFLOPs  \\ \hline
		resnet34 & 224 x 224 & 83 MB & 35 MB & 4 GFLOPs   \\ \hline
		resnet-50 & 224 x 224 & 98 MB & 103 MB & 4 GFLOPs   \\ \hline
		resnet-101 & 224 x 224 & 170 MB & 155 MB & 8 GFLOPs  \\ \hline
		resnet-152 & 224 x 224 & 230 MB & 219 MB & 11 GFLOPs   \\ \hline
	\end{tabular}
	\caption{Rendimiento de ResNet en \textbf{Imagenet} \cite{resnetspecs}}
\end{table}

Concretamente, se procede a escoger la versión ResNet50 del modelo, cuya característica clave es ser el punto de equilibrio entre modelos de pequeño y gran tamaño por sus requisitos de memoria y tiempo de inferencia. Podemos apreciar estos valores en la tabla \ref{fig:tablaresnet}



\subsection{MobileNet V2}

Mobile Net es una red convolucional especializada en su uso directo en dispositivos móviles. Este modelo ya fue descrito en su totalidad en el capítulo \ref{cap:mobile}. Entre sus modelos preentrenados que podemos encontrar, debemos destacar la existencia de los modelos V1 y V2:

\begin{table}[!ht]
	\centering
	\begin{tabular}{|c|c|c|c|c|c|}
		\hline
		\textbf{Model} & \textbf{Input} & \textbf{Size (model)} & \textbf{Est. FLOPS}  \\ \hline
		MobileNet-V1 & 224 x 224 & 16.9 MB & 0.569 GFLOPs \\ \hline
		MobileNet-V2 & 224 x 224 & 14.0 MB & 0.3 GFLOPs	  \\ \hline
	\end{tabular}
		\caption{Rendimiento de MobileNet en \textbf{Imagenet} \cite{mobilespecs}}
\end{table}

Ambos modelos son bastante ligeros, por lo que no requieren cuantización para obtener buen rendimiento. En este caso, se elige la versión V2, debido a su mejor eficiencia computacional sin pérdida de resultados. En cuanto ala profundidad y ancho de la red, parámetros clave, se utilizan profunidad = 1 y anchura = 1, es decir, no se realizará ninguna reducción sobre el modelo, de forma que pueda tener profundidad suficiente para aprender todas las clases del conjunto de entrenamiento. Se descarta el uso del modelo V3 por su especialización en segmentación, ya que el problema que estamos tratando es clasificación.\\

Sin embargo, tal y como veremos en el estudio de resultados posteriomente, los resultados ofrecidos no estarán finalmente a la altura de la aplicación.


\subsection{EfficientNet B5}

EfficientNet, modelo descrito en el capítulo \ref{efnetcap},  se trata de otro conjunto de arquitectura de red cuyo funcionamiento en problemas de clasificación de enfermedades cutáneas es positivo, tal y como demuestran los resultados ganadores de la competición de ISIC Challenge \cite{1stISIC, 2ndISIC}.

Como ya detallamos anteriormente, estos modelos son demasiado costosos para entrenar y ejecutar en sus versiones de gran tamaño, como las utilizadas en dichas soluciones. En este caso, emplearé de nuevo el modelo intermedio, EfficientNet B5, el cual logra un equilibrio suficiente entre tamaño y calidad del resultado. Además, por limitación del hardware disponible, no es posible utilizar sus variantes superiores, ya que ni el entrenamiento e inferencia de los mismos es posible.

\begin{table}[!ht]
	\centering
	\begin{tabular}{|c|c|c|}
		\hline
		\textbf{Model} & \textbf{\# of Parameters} & \textbf{Est. FLOPs} \\ \hline
		EfficientNet-B0 & 5.3M & 0.39B \\ \hline
		EfficientNet-B1 & 7.8M & 0.70B \\ \hline
		EfficientNet-B2 & 9.2M & 1.0B \\ \hline
		EfficientNet-B3 & 12M & 1.8B \\ \hline
		EfficientNet-B4 & 19M & 4.2B \\ \hline
		EfficientNet-B5 & 30M & 9.9B \\ \hline
		EfficientNet-B6 & 43M & 19B \\ \hline
		EfficientNet-B7 & 66M & 37B \\ \hline
	\end{tabular}
	\caption{Tamaño de las versiones de Efficient Net \cite{tan2020efficientnet}}
\end{table}

El tamaño del model final no es calculable debido a la amplia variedad de parámetros a configurar en cuanto a su arquitectura. En el caso de la competición antes mencionada, este rondaba entre 300 y 400MB para la versión B7, motivo por el cual se ha decidido seleccionar la versión B5, y obtener un tamaño estimado de 150-200MB.

\section{Función de pérdida}

El modelo a entrenar requiere el uso de una función que nos permita transformar la salida paramétrica del modelo, en un valor numérico legible, que nos permita conocer su progreso en el entrenamiento.  Es decir, nos interesaría saber tanto el error $E_{in}$ como $E_{val}$, como un valor de pérdida, que queremos minimizar para ajustarnos lo máximo posible a la distribución real. Mientras que la probabilidad de pertenencia a cada clase nos la ofrece la función Softamax, la función de pérdida puede ser elegida entre varias alternativas. Comúnmente se utiliza CrossEntropy loss, pero para este problema, utilizaré también FocalLoss, especializada en el ajuste de modelos con clases desbalanceadas, como es nuestro caso.

\subsection{Cross entropy}

La entropía cruzada es una función de pérdida utilizada comúnmente en problemas de clasificación. Su imagen se utiliza como valor a minimizar, y representa la calidad de ajuste del modelo.

Se apoya en las salidas de la función softmax. Conocemos que Softmax proporciona como salida, en un problema de clasificación, la probabilidad de que el ejemplar que estamos clasificando pertenezca a una clase concreta. La entropía cruzada se basa en calcular la distancia existente entre las salidas probabilísticas de la función softmax, y el valor real de su etiqueta. Se calcula cuánto divergen los valores entre sí de los valores predichos contra los valores reales.

Como su resultado es nuestra representación del error, queremos que sea lo más pequeño posible. Por eso, se intenta minimizar esta función para obtener mejores resultados.

Su expresión analítica es la siguiente:

$$H(P,Q)= -\sum_{x \in X}p(x)log(q(x))$$

Donde p es el valor real, y q es el valor estimado por nuestro modelo para cada ejemplo del conjunto X de entrenamiento.

Esta función es adecuada para la mayoría de casos, aunque puede ver perjudicado su rendimiento cuando las clases son dispares en número entre sí.

\subsection{Focal Loss}

Focal loss \cite{lin2018focal} es una función de perdida que modula el aprendizaje en conjuntos de datos desbalanceados, para fomentar el aprendizaje de ejemplos infrarepresentados, o más complejos, sobre aquellos más sencillos.

Para ello, parte de las expresiones de CrossEntropy loss, aplicando el efecto de una valor regularizador, al que llamaremos $\alpha$. Este peso aportará más valor a los casos complejos, mediante la siguiente expresión:

$$FL = \left\{ \begin{array}{lcc} -\alpha(1-p)^{\gamma}log(p)& y = 1 \\ 
	\\  -(1-\alpha)p^{\gamma}log(1-p) & otherwise  \\ \end{array} \right.$$

Donde $log(p)$ hace alusión a CrossEntropy evaluado a un único elemento, y el valor $\alpha$es aquel que modula el efecto regularizador.

Los casos díficiles hacen referencia a los Falsos negativos' mientras que los sencillos son los correctamente clasificados, en especial los Verdaderos negativos. Cuanto más valor se aporte a $\gamma$, mayor será la importancia de los objetos no clasificados correctamente, y con $\alpha$, cuanto mayor, mayor es la importancia del conjunto no clasificado en completo.\\

Esta función de pérdida es muy útil para el problema, ya que el dataset sufre de un problema de desbalanceo importante en sus clases, tanto a nivel de clasificación benigno-maligno, como a nivel de enfermedades de cada tipo. Por tanto, esta función de pérdida se hace ideal para dar más importancia a las clases menos representadas, y que por ende, tenderán a ser marginales para el modelo si no se ajusta adecuadamente. Su Su configuración por defecto, con $\gamma = 2$ es la adecuada para la mayoría de problemas, y será la configuración empleada para este dataset de entrenamiento.

\section{Especificaciones y framework de trabajo}

La reaización de este proyecto usará el lenguaje Python, debido a la gran extesión de las librerías y frameworks de Deep learning desarrollados en la última decada. Concretamente, se basará en el uso de la librería de aprendizaje Pytorch\cite{paszke2019pytorch}, con el apoyo del framework de alto nivel FastAI\cite{howard2018fastai}.

Se ha elegido Pytorch por su tendencia creciente en las publicaciones científicas, así como su interfaz simplificada frente a Tensorflow. Aunque ambas soluciones son equivalentes, se ha elegido la primera por su reciente incorporación de capacidades para modelos Android, capaz de superar levemente en rendimiento promedio a Tensorflow lite.\\

FastAI, por su parte, esta basado en Pytorch, y permite la implementación de modelos y  la organización de los datasets de forma simplificada mediante un mayor grado de abstracción. Permitirá simplificar la implementación en aquellas tareas que funcionen adecuadamente con las configuraciones ofrecidas.\\

Para las ténicas de aumento de datos, se emplearán dos librerias:
\begin{itemize}
	\item Imgaug: librería de aumento de datos, específicamente diseñada para tareas de aprendizaje profundo, y agnóstica en cuanto al framework de desarrollo empleado. Esto permite su utilización de forma independiente, de forma previa al aprendizaje, para realizar otras tareas, como el oversampling de clases minoritarias, procedimiento el cual aplicaremos para equilibrar el entrenamiento en clasificación binaria (benigno vs maligno).
	
	 \item Albumentations: se trata de otra librería de aumento de datos, que se integra de forma nativa con el formato datablock de Fastai. Nos permitirá realizar aumento de datos de la forma habitual, cuando no se necesite aplicar oversampling a los datos de entrenamiento.
\end{itemize}

En cuanto a las características del sistema, se usarán las dos configuraciones siguientes:

\begin{itemize}
	\item Intel Core i7 12700K, Nvidia RTX 3080, 32GB de RAM
	\item Intel Xeon Silver 4216, Nvidia Quadro RTX 8000 32GB, 20GB RAM
\end{itemize}

Una vez definido el framework y las especificaciones a utilizar, podemos pasar a detallar el proceso de aprendizaje completo y sus resultados.


	
	 \chapter{Proceso de aprendizaje}

Tras las descripción de los modelos a utilizar, y las funciones de pérdida a emplear, comienza el proceso de entrenamiento de los modelos. Este proceso es uno de los más lentos de todo el desarrollo del proyecto, debido a la cantidad de horas necesarias a esperar para recopilar los resultados del modelo. Se estudiará la eficiacia de cada modelo a la hora de evaluar el problema, y las decisiones tomadas para su mejora y elección.\\

Para resolver este problema, se ha tomado un enfoque de dos niveles; en primer lugar, distinguiremos si la enfermedad que se recibe como entrada se trata de un ejemplar de enfermedad maligna o benigna, y posteriormente, su salida indicará un segundo modelo a aplicar sobre la enfermedad, especializado únicamente en uno de los dos tipos de enfermedades. Así, conseguimos un conjunto de modelos especialziados que permiten otorgar el diagnóstico con mayor precisión y seguridad.\\

Un aspecto clave de este proceso será la eficiencia, pero sobre todo el correcto diagnóstico de las enfermedades, por lo que usaremos como métricas de selección la precisión, el recall y el accuracy balanceado:

\begin{itemize}
	\item Precision = TP / (TP + FP). Proporción de valores bien clasificados dentro de una clase teniendo en cuenta verdaderos y falsos positivos.
	\item Recall = TP / P. Valores correctamente identificados como positivos respecto al total de elementos positivos.
	\item Accuracy balanceado: combina sensitivity (TP/(TP+FN)) y specificity (TN/(TN+FP)). La primera, devuelve el valor real de proporción de valores correctamente clasificados entre el total de casos positivos que tenemos (contando predicciones positivas y falsos negativos), mientras que la especifidad, nide el caso dual: la proporción de casos negativos bien clasificados respecto al total de datos negativos tanto bien clasificados como mal clasificados, y que son identificados como falsos positivos. Esto nos permite calcular el accuracy de forma proporcional al porcentaje de presencia de las clases, e intentamos así paliar el efecto del desequilibrio de los datos:
	
$$ Balanced\ accuracy: \frac{Sensitivity + Specificity}{2}$$
	
	En base a esta expresión, podemos saber que si el valor es aproximadamente 1/númeroClases, es posible que gran parte de las clases no estén siendo correctamente clasificadas en caso de ser minoritarias. Valores mayores nos harán conocer que el resultado de clasificar todas las clases es satisfactorio.
\end{itemize}

\section{Clasificación binaria}

El problema de la clasificación binaria consiste en la distinción de las enfermedades malignas de aquellas que son benignas. Se trata del primer problema a resolver antes de especificar el tipo de enfermedad que posee el paciente, y así poder especializar los modelos segundo nivel.

\subsection{Equilibrio de los datos}

El primer inconveniente que nos encontramos para este problema es el inmenso desbalaceo entre la clase benigna y maligna. Si redibujamos el gráfico mostrado durante la fase de preprocesamiento de datos, únicamente con los datos de entrenamiento, obtenemos lo siguiente: 

\begin{figure}[H]
	\centering
	\label {desequilibriototal}
	\includegraphics[scale = 0.4]{imagenes/desequilibriototal.png}
	\caption{Desequilibrio de datos}
\end{figure}

Se puede observar la alta disparidad existente entre ambas, siendo 35664 casos benignos y 9752 malignos. Es una diferencia de 3.65 veces más casos benignos que malignos, lo que se traduce en un equilibrio de  79\% y 21\% respectivamente. Para paliarlo, podemos efectuar dos estrategias: aplicar técnicas de oversampling (también conocido como sobremuestreo sintético), donwsampling de la clase mayoritaria, o bien, la aplicación de penalziaciones sobre la clase minoritaria para que su incorrecta clasificación tenga mayores consecuencias (haciendo uso de FocalLoss).\\

\subsubsection{Técnicas descartadas}

La ténica de downsampling de la clase mayoritaria queda descartada, debido a que dentro de esa clase, podemos encontrar etiquetas de segundo nivel asociadas a las enfermedades que se pueden diagnosticar. Como algunas de estas clases son muy escasas en número, si realizamos downsampling de forma aleatoria sin ningún tipo de restriccióin, podría darse el caso límite en la que una clase completa desapareciese del cojunto. Y esto conllevaría a la obtención de errores al evaluación de test, pues tendríamos una clase presente en test no estudiada en entrenamiento, que sólo provocaría un aumento de las métricas de error. \\

El uso de la función de pérdida FocalLoss podría ayudar a paliar este efecto. Sin embargo, debido al desajuste 79-21 del que disponemos, y la gran variedad de imágenes, se trata de una solución demasiado arriesgada. Para comprobarlo, se sometíó a evaluación con los valores por defecto recomendados de Resnet50. A priori, los resultados podrían parecer correctos en validación \ref{tab:resultsfl}; sin embargo, observamos que la clase minoritaria, la maligna, obtiene resultados que no están a la altura de la clase mayoritaria. Observamos un valor 0.41 en recall, valor preocupante frente al 0.95 de la clase benigna. Al existir un recall bajo, quiere decir que esta clase no se está detectando adecuadamente como maligna, y casi el 60\% de sus casos son calificados como benignos. Esto es un grave problema, ya que los casos malignos son los que más riesgo conllevan para la salud. Por tanto, se descarta también esta estrategia.


\begin{table}[!ht]
	\centering
	\begin{tabular}{|l|c|c|c|c|}
		\hline
		& Precision & Recall & F1-score & Support \\
		\hline
		Benign & 0.86 & 0.95 & 0.90 & 15338 \\
		Malignant & 0.68 & 0.41 & 0.51 & 4126 \\
		\hline
		Accuracy &  &  & 0.83 & 19464 \\
		Macro avg & 0.77& 0.68& 0.71&19464\\
		Weighted avg&0.82&0.83&0.82&19464\\
		\hline
	\end{tabular}
	\caption{Informe de clasificación}
	\label{tab:resultsfl}
\end{table}

\subsubsection{Propuesta: oversampling}

Sólo nos queda disponible la estategia de oversampling: la ampliación sintética de la clase minoritaria. Este proceso es muy delicado, ya que se basa en realizar alteraciones sobre las imágenes disponibles para dotarlas al conjunto de una mayor variabilidad, hasta equiparar el número de ejemplares con el de la clase mayoritaria. Existen distintas técnicas a aplicar, pero en este caso, utilizaré la estrategia de realizar alteraciones sobre las imágenes mediante una biblioteca de transformaciones: imgaug.

Se aplicarán una serie de transformaciones la imagen que alteren su estructura, pero con cuidado de no generar imágenes demasiado modificadas que se alejen de la distribución original de la clase maligna. Para realizar las transformaciones, he contruido un pipeline, una secuencia de transformaciones que aplica, en orden aleatorio, y con cierta probabilidad, las siguientes variaciones:

\begin{itemize}
	\item Fliplr(p=0.5). Se trata de un efecto de espejo de la imagen, respecto a su eje horizontal. Es equivalente a realizar una rotación de la imagen de 180º sobre su eje teórico horizontal que transcurre por la mitad de la imagen. Esta alteración se realziará con un 50\% de probabilidad.
	\item Flipud(p=0.5). Se trata de un efecto de espejo de la imagen, respecto a su eje vertical. Es un caso de efectos similares al anterior, aplicado con un 50\% de probabilidad.
	\item Gaussian blur (p=0.5). Se aplicará filtro gaussiano en el 50\% de los casos, especificándose un valor de sigma muy pequeño de 0.5.
	\item Modificación de contraste  (p=0.25). Se aplicará un aumento o decremento del contraste del 20\% en el 25\% de los casos. Esto emula posibles cambios de luminosidad ambiente y de profundidad de color de la cámara
	\item Ruido Gaussiano aditivo  (p=0.5). Se aplica en caso de que el ruido gaussiano estándar no sea aplicado. Aplica un filtro gaussiano a nivel de canal, de forma que solo parte de la imagen se encuentra difuminada.  Se aplica también en un 50\% de los casos
	\item Transformaciones afines  (p=0.2): se trata de una alteración compuesta de escala y zoom, y una posterior traslación o rotación de la imagen. Se trata de una opcionalidad ofrecida por imgaug de forma no configurable directamente.
\end{itemize}

Estos valores han sido escogidos tomando como inspiración las transformaciones realizadas por el equipo ganador de la competición de ISIC \cite{1stISIC}, debido a sus buenos resultados.

Es importante destacar que dichas transformaciones únicamente se han realizado sobre los datos de entrenamiento; los datos de validación y test permanecen inalterados de forma que éstos sigan siendo un estimador no sesgado del rendimiento del modelo. Poir tanto, la extracción del 30\% de validación se realiza antes del sobremuestreom y los dos subconjuntos se guardan en carpetas distintas, a las que llamaremos train y test.

\subsection{Construcción del modelo}

Con los datos preparados, configurar mediante Fastai y Pytorch los modelos que deseamos probar. Los 3 modelos siguen la misma configuración, a excepción de la importación del modelo, por lo  que será explicado de forma paramétrica. Tenemos 3 tareas diferenciables: la creación del datablock para la gestión de los datos, el entrenamiento del modelo, y la transformación del mismo al modelo cuantizado empleado en el dispositivo Android.

\subsubsection{Creación del datablock}

FastAI ofrece el tipo de objeto Datablock para la creación de una estructura que organiza los datos de entrenamiento. Permite construir un bloque con todos los datos, de forma que cualquier consulta o verificación pueda realizarse sobre un mismo objeto. Consta de varios parámetros que podemos ajustar para normalizar las imágenes, adaptar el tamaño de entrada, o realizar la separación train-validación. Desglosadando los parámetros más importantes, tenemos:

\begin{itemize}
	\item blocks. Nos permite establecer el formato de entrada de datos, y la salida que queremos obtener de los mismos. En problemas de clasificación, habitualmente, tendremos como entrada un bloque de imágenes, y como salida, queremos obtener un conjunto de categorías asociados al bloque.
	\item get\_items. Establece la fuente de la cual obtener los datos para la construcción del DataBlock. Se trata de una función preconfigurada para la lectura de datos dado un repositorio o dirección de disco. En nuestro caso, como las imágenes no contienen información compleja, como plantillas de segmentación, se importan directamente de esta forma.
	\item splitter. Permite especificar divisiones de los datos. Habitualmente, se utiliza la función RandomSplitter, que nos permite separar los datos en entrenamiento y validación. Como parámetros de entrada, recibe el porcentaje de datos para validación que queremos obtener, y opcionalmente una semilla, por si deseamos repetibilidad de los experimentos. Como en este caso, queremos que el conjunto de entrenamiento sea el conjunto sobremuestreado, y validación, sea la carpeta valid, se ha creado una función que indica 1 si la imagen pertenece a la carpeta valid, y 0, si se encuentra en train. \ref{fig:particiones}
	\item get\_y. Parámetro al cual se asocian las etiquetas de los elementos de entrada. Puede recibir una función o bien una lista con las etiquetas asociadas a los elementos del dataset. FastAI toma las etiquetas por defecto del nombre de carpeta, por lo que tendremos que especificarle una función que tome la etiqueta del dataset \ref{fig:etiquetadobinario}.
	\item item\_tfms. Establece modificaciones sobre las imágenes de forma previa a la ejecución del modelo. En este caso, como ya hemos realizado oversampling, no realizaremos ninguna otra modificación sobre las imágenes para no transformar en exceso los datos.
	\item batch\_tfms. Permite realizar transformaciones a nivel de batch. Permite aplicar operaciones como la normalización de las imágenes. Este procedimiento favorece la convergencia del modelo, aunque el cálculo de la media y desviación típica es costoso. En su lugar, utilizaré la normalización que provee ImageNet, donde ya se incluyen los valores estadísticos de media y desviación para los datos en la variable *image\_stats provenientes de decenas de millones de imágenes.
	
\end{itemize}


 \begin{algorithm}[H]
	\caption{Función para la distinción de entrenamiento y validación}
	\label{fig:particiones}
	\begin{algorithmic}[1]
		
		\Procedure{ val\_splitter}{fname}
		\Comment{Comprueba si la carpeta contenedora es validiación}
		\State var pertenece: boolean = False
		\If{Path(fname).parent.name = valid}
			\State pertenece = True
		\EndIf \\
		\Return pertenece
		\EndProcedure
		
	\end{algorithmic}
\end{algorithm}

\begin{algorithm}[H]
	\caption{Función para la consulta de etiquetas}
	\label{fig:etiquetadobinario}
	\begin{algorithmic}[1]
		
		\Procedure{ binary\_label}{fname, df\_data: Dataframe}
		\State coincidence = $df\_data[i]$ where $df\_data[i] = fname, i \in Integer$
		\State var label = coincidence.label\\
		\Return label
		\EndProcedure
		
	\end{algorithmic}
\end{algorithm}

Una vez definido, podremos instanciar el datablock, configurando uno de los parámetros más críticos del entrenamiento: el tamaño de batch.

Cuando entrenamos una red, estamos ajustando una serie de pesos según recorremos los datos. Éstos, por tanto, pueden actualizarse en diferentes momentos: tras procesar cada imagen, tras procesarlas todas, y antes de volver a empezar, o cada n imágenes. A dicho tamaño n, se le conoce como tamaño de batch: es el tamaño del conjunto de imágenes que han de procesarse para realizar una actualización de pesos del modelo. 

Este número es un hiperparámetro clave, ya que tiene relación directa con la convergencia: a mayor tamaño, mayor velocidad de convergencia, pero menor precisión; y el caso inverso ocurre con el caso extremo, si, por ejemplo, actualizamos con cada ejemplar procesado. En la literatura, existen gran cantidad de estudios acerca de este valor, y establecen el valor adecuado en un intervalo $[2, 32]$,  compuesto únicamente por las potencias de 2   \cite{masters2018revisiting}. Debido a limitaciones de memoria, y la demostración de su optimalidad, emplearé tamaño 32 para los 3 modelos, aunque otros tamaños, como 64 o 128, también son viables.

A partir de ahora, denominaremos época al proceso de recorrer cada uno de los batches del datablock.

\subsubsection{Definición del modelo. Transfer learning}

Con los datos correctamente definidos, podemos construir el modelo propiamente dicho para el entrenamiento. En Fastai, la construcción de modelos puede hacerse de forma modular, mediante la especificación de las capas que componen la red, ya que la librería dispone de los módulos ya configurados a excepción de los canales de entrada y salida. A pesar de ello, la creación de un modelo desde cero es muy costoso computacionalmente, ya que se parte el entrenamiento de valores aleatorios. Por este motivo, normalmente se tiende a emplear modelos preentrenados, y adaptar su salida mediante el uso de transfer learning.

El transfer-learning es una métodología que permite utilizar complejas redes convolucionadas preeentrenadas con datasets de gran tamaño que son adaptadas para utilizarse con otros datasets diferentes y obtener buenos resultados, sin necesidad de realizar un entrenamiento desde cero del modelo.\\

De forma intuitiva, el proceso a seguir es el siguiente: partimos de las capas ocultas ya entrenadas del modelo, y sustituimos su cabecera por una nueva que se adapte a nuestro problema. Por cabecera(head), entendemos la parte final de la red, punto en el cual se elabora la salida final que la red pasa a la función SoftMax para obtener las probabilidades de pertenencia a cada clase de la imagen de entrada. Como la red fue entrenada para otro dataset, es probable que el número de clases de salidas sea diferente, y por eso, debemos de sustituirla por una nueva que se adapte a las necesidades de nuestro problema.\\

Una vez sustituida el head de la red, debemos de entrenar la nueva cabecera para que sus pesos se adapten a los datos de entrada y ofrezcan buenas métricas. Pero no nos interesa modificar el resto de capas ocultas, ya que fueron entrenadas previamente y tienen valores de W adecuados. Realizaremos un "congelado" de estas capas, para únicamente entrenar la nueva adición.

Posteriormente, para afinar el comportamiento de la red completa, ``descongelaremos'' la red para hacer un entrenamiento breve a la red completa, durante un número pequeño de épocas. Así, obtemos un modelo final adaptado a nuestro problema.\\

En esta primera fase, debemos ofrecer una salida binaria, que nos ofrezca una respuesta de 0 (benigno) y 1 (maligno). Para conseguirlo, recurriremos a retirar la capa predefinida con la que viene nuestro modelo, y para todos ellos, emplearemos la misma cabecera, la cual es creada de forma automática por fastai tras especificar el número de salidas.

\begin{table}[H]
	\centering
	\caption{Cabecera para trasfer-learning binario}
	\begin{tabular}{|c|c|c|}
		\hline
		\textbf{Capa} & \textbf{Tam. Entrada} & \textbf{Tam. Salida} \\ \hline
		Average pooling & n x n & 32 x 2048 \\ \hline
		Flatten & 32 x 2048 & 4096 \\ \hline
		Fully connected & 4096 & 512 \\ \hline
		Fully connected & 512 & 32 x 2048 \\ \hline
		Fully connected & 32 & 2 \\ \hline
		Softmax & - & - \\ \hline
	\end{tabular}
	\label{header}
\end{table}

Dicha cabecera, observable en la figura \ref{header}, recibe como tamaño de entrada n x n, siendo n el tamaño de la salida de la última capa de la red preentrenada que estemos utilizado. Mediante una oporación de average pooling, podemos convertir dicha entrada a un tamaño de salida concreto, el cual nos permitirá tener a partir de este punto una red totalmente conectada la cual no requiere ser modificada aunque modifiquemos el tamaño de entrada de la red por imágenes de mayor o menor resolución. \\

La configuración escogida tras el flattening de la imagen, que consiste en vectorizar su contenido, es una tradicional triple capa de redes totalmente conectadas: la estructura tradicional de una red neuronal donde cada neurona se encuentra conectada a las siguientes. Dicha estructura de entradas y salida sigue forma de ``embudo'', en el que el tamaño de la capa final coincide con el  número de salidas, en este caso, dos clases.

Para instanciar el modelo ya configurado, se hace uso de la clase vision\_learner, la cual recibe como entreada el modelo, expresado como cadena, un vector con los nombres de las métricas, el datablock de entrenamiento y validación creado anteriormente, y las funciones de callback. Estas últimas son funciones que serán ejecutadas cada vez que se complete una época de nuestro modelo. Como entrada, he decidido añadir un callback de salvado, que guarde el modelo tras cada época finalizada, y un monitor de Early stoping.


\begin{figure}[H]
	\centering
	\label {earlystopping}
	\includegraphics[scale = 0.25]{imagenes/earlystopping.png}
	\caption{Demostración gráfica del Early stopping}
\end{figure}

Resumiendo, early stopping es un mecanismo de parada automática del entrenamiento, el cual, especificando una paciencia, en época, si no se cumple una condición, se detiene el modelo; en este problema, he decidido que el monitor de parada supervise el error de pérdida en validación, para así evitar el sobreaprendizaje de la red: la memorización completa del conjunto de entrenamiento, conllevando el riesgo de pérdida de generalidad. Controlaremos este fenómeno gráficamente mediante la muestra de las pérdidas de entrenamiento y validación.

\subsubsection{Entrenamiento}

El entrenamiento es la otra parte clave de la creación de modelos de deep learning. Esto se debe a la existencia de usa serie de hiperparámetros que debemos regular para asegurar el correcto entrenamiento del modelo. En este apartado, trataramos los dos parámetros clave: el número de épocas de entrenamiento, y el learning rate.

El número de épocas es equivalente al número de iteraciones realizadas sobre la totalidad de los datos del modelo, Éste debe ser fijado de forma estratégica para evitar que se produzca sobreentrenamiento, donde en lugar de aprender las características generales de los datos y su distribución, pasamos a su memorización completa. Este fenómeno, como ya se ha explicado, provoca pérdida de generalidad del modelo fuera del conjunto de entrenamiento, y no es lo deseable. En lugar de establecer dicho número de forma fija, usaremos, como ya enunciamos, Early stoping. De esta forma, será el propio modelo quien se detenga sin necesidad de que establezcamos un valor manual. Concretamente, emplearemos una paciencia de 3 épocas sobre la pérdida de validación.\\

En cuanto al learning rate, éste es un valor crítico para definir el tamaño de salto que se produce a la hora de optimizar el modelo. Define el tamaño del salto (step) que se produce a través del hiperplano definido por la función que modela los datos, en búsqueda de su mínimo local (o global). AL igual que el tamaño de batch, define de manera implícita la precisión de la búsqueda del óptimo; a mayor valor, mayor velocidad de convergencia, pero menor certeza de encontrar el óptimo de forma precisa; y a menor valor, mayor tiempo de entrenamiento, pero mayores garantías de encontrar un valor cercano al óptimo de la función.

La decisión de este parámetro normalmente se realiza mediante pruebas experimentales. Sin embargo, al disponer de recursos hardware limitados, no es posible realizar experimentos en paralelo en búsqueda de este valor, cuyo entrenamiento podría conllevar semanas. Por tanto, en lugar de usar un learning rate fijo, se ha tomado la decisión de usar un learning rate adaptativos, mediante la política de entrenamiento de un solo ciclo (Fit one cycle en FastAI) propuesta por Leslie Smith \cite{smith2018disciplined}.

Con esta técnica, el valor del learning rate se ajusta de forma iterativa, partiendo de un valor pequeño, llegando a un valor máximo, y finalizando en un valor más pequeño que el inicial.

Este proceso se realiza para conseguir una buena detección de óptimos, sin necesidad de conocer el valor exacto de learning rate, que es una tarea compleja. Este se basa en el compartamiento de la teoría de learning rate cíclicos, donde el learning rate parte de un valor pequeño, llega a un máximo, y luego se decrementa de nuevo hasta un valor mínimo, de forma periódica. Pero la diferencia radica en que fit one cycle realiza un único ciclo de crecimiento y decrecimiento del valor del learning rate.

Otra gran ventaja de este método es que no se requieren gran cantidad de parámetros. Con la implementación realizada en Fastai, solo es necesario especificar el número de épocas a realizar, y un valor máximo de learning rate, clave para especificar el valor pico de la fase ascendente del lr.

Fastai también implementa un método para ayudar en la búsqueda del learning rate pico, llamado lr\_find. Este método realiza un estudio de diferentes valores de learning rate y su capacidad minimizadora de la función de pérdida, y devuelve puntos de partida razonables para el inicio del entrenamiento: puntos pico, valle , y pendiente. De todos ellos, nos quedaremos con el punto valle, ya que es el que mejores propiedades de generalización posee.\\

La decicisión, por tanto, de elegir este método, es la considerable reducción de costo computacional que se produce, así como la mejora de los resultados, tal y como se demuestra en el paper original publicado por L. Smith \cite{smith2018disciplined}. 

\subsection{Resultados}

Una vez realizado el proceso de preparación de los modelos, comienza el período de aprendizaje. Para comparar los resultados de cada uno de los modelos, se ha recopilado información acerca de las métricas ya mencionadas, las pérdidas, y, sobre todo, el accuracy balanceado, para observar la tasa de acierto de cada uno de los modelos.\\

Destacar también que, en esta parte del proyecto, la gran cantidad de imágenes y el impacto de la calidad de las mismas provocaban problemas desbordamientos de memoria, que interrumpían el entrenamiento por falta de recursos hardware. Debido a este motivo, únicamente se ha utilizado el subconjunto de imágenes que pertenecen al dataset ISIC, el cual, engloba un aproximado 60\% de los datos totales recopilados, a un tamaño máximo de entrada de 512x512. Reservaremos las imágenes del resto de datasets para la distinción de enfermedades malignas y benignas de forma más concreta.

A continuación, mostramos los resultados obtenidos por cada modelo.

\subsubsection{ResNet 50}

Los resultados con ResNet 50 son bastante satisfactorios\ref{tab:resultsbinrn50} , ya que consiguen mejorar los ya vistos anteriormente cuando se probó el entrenamiento mediante Focal Loss.

\begin{table}[!ht]
	\centering
	\begin{tabular}{|c|c|c|c|c|}
		\hline
		\textbf{} & \textbf{Precision} & \textbf{Recall} & \textbf{F1-score} & \textbf{Support} \\ \hline
		\textbf{Benign} & 0.87 & 0.91 & 0.89 & 7205 \\ \hline
		\textbf{Malignant} & 0.70 & 0.60 & 0.64 & 2460 \\ \hline
		\textbf{Accuracy} & ~ & ~ & 0.83 & 9665 \\ \hline
		\textbf{Macro avg} & 0.78 & 0.76 & 0.77 & 9665 \\ \hline
		\textbf{Weighted avg} & 0.83 & 0.83 & 0.83 & 9665 \\ \hline
	\end{tabular}
	\caption{Informe de clasificación Resnet50}
	\label{tab:resultsbinrn50}
\end{table}


Podemos apreciar que el aprendizaje de la clase minoritaria, la maligna, ha mejorado considerablemente respecto a la alternative de ajustar la función de pérdida unicamente. El aumento del accuracy se ve potenciado, sobre todo, en el aumento de la precisión y el recall. Mientras que sin hacer oversampling, obteníamos 68\%(0.68) y 41\% (0.41), respectivamente, ahora obtenemos 70 \% (0.70) y 60\% (0.60), siendo por tanto el aumento en casi 0.2 en el caso del recall. Esto significa que estamos consiguiendo mejores resultados a la hora de distinguir las lesiones malignas, y estas se diagnostican correctamente.

En cuanto a la clase benigna, hemos sacrificado parte de los resultados al realizar el equilibrado, pero sus valores siguen siendo muy buenos: entorno al 87\% de perecisión, y 91\% de recall.


Si evaluamos el modelo en cuanto a su tasa constancia en el aprendizaje, debemos examinar primero las curvas descritas por la función de perdida en entrenamiento y validación, observables en la figura \ref{fig:curvasrensetbinaria}

l\begin{figure}[H]
	\centering
	\subfigure[Fase de congelado]{\includegraphics[width=0.45\textwidth]{imagenes/bin_class_resnetfreeze.png}} 
	\subfigure[Descongelado]{\includegraphics[width=0.45\textwidth]{imagenes/bin_class_resnetunfreeze.png}} 
	\caption{Curvas del finetuning del modelo binario con Resnet50}
	\label{fig:curvasrensetbinaria}
	
\end{figure}

El aprendizaje es correcto durante la fase de congelado, pues podemos apreciar con la curva de entrenamiento y validación descienden en valor hasta la época 10, momento en el cual la pérdida asciende durante 3 épocas seguidas y provoca la interrupción por early stopping. No podemos afirmar lo mismo de la fase de descongelado, donde podemos apreciar una subida abrupta del valor de la función de pérdida de validación, lo que quiere decir que la red ya estaba correctamente entrenada y el descongelado únicamente ha provocado sobreaprendizaje.\\

En total, el modelo ha requerido 4 horas y media para el proceso de entrenamiento, y casi 1.65 horas para el proceso de oversampling de la clase minoritaria, que ha llegado a ocupar 240GB de disco.\\

Por tanto, el mecanismo de oversampling parece haber funcionado, sin apenas penalizar los resultados ya obtenidos para la clase mayoritaria, aunque con grandes requisitos de almacenamiento. Sin embargo, debemos comparar con los otros dos modelos antes de decidir cuál se empleará para cuantizar y usar de forma definitiva en la app.

\subsubsection{Efficient Net}

Efficient net B5 es la segunda arquitectura elegida para su comparación. El resultado ofrecido, siguiendo los mismos procedimientos aplicados en Resnet, son los mostrados en la tabla \ref{tab:resefnet}.

\begin{table}[H]
	\centering
	\begin{tabular}{|c|c|c|c|c|}
		\hline
		\textbf{} & \textbf{Precision} & \textbf{Recall} & \textbf{F1-score} & \textbf{Support} \\ \hline
		\textbf{Benign} & 0.89 & 0.93 & 0.91 & 7205 \\ \hline
		\textbf{Malignant} & 0.76 & 0.65 & 0.70 & 2460 \\ \hline
		\textbf{Accuracy} &  &  & 0.85 & 9665 \\ \hline
		\textbf{Macro avg} & 0.85 & 0.79 & 0.8 & 9665 \\ \hline
		\textbf{Weighted avg} & 0.83 & 0.83 & 0.86 & 9665 \\ \hline
	\end{tabular}
	\caption{Informe de clasificación de EfficientNet B5}
	\label{tab:resefnet}
\end{table}

A priori, podemos apreciar unos resultados ligeramente mejores al caso anterior, siendo de media, 0.02 superior en cada una de las métricas establecidas para la comparación. Esto nos deja con una tasa de acierto equilibrada de 85\% en ambas clases. Se trata, por lo tanto, del modelo con mejores resultados hasta la fecha. Sin embargo, si analizamos las curvas de aprendizaje \ref{fig:curvasefnetbinaria}, podemos observar un fenómeno llamativo: la curva no sigue la tendencia habitual a minimizarse, y posee picos de gran altura donde el error de validación se dispara.

l\begin{figure}[H]
	\centering
	\subfigure[Fase de congelado]{\includegraphics[width=0.45\textwidth]{imagenes/bin_class_efnetfreeze.png}} 
	\subfigure[Descongelado]{\includegraphics[width=0.45\textwidth]{imagenes/bin_class_efnetunfreeze.png}} 
	\caption{Curvas del finetuning del modelo binario con EfficientNet B5}
	\label{fig:curvasefnetbinaria}
\end{figure}

Este fenómeno puede deberse al reajuste de pesos fruto del learning rate variable, donde un subconjunto de las imágenes correctamente etiquetadas es cambiada de clase a la hora de modificar los valores para adaptarse a otro conjunto distinto no correctamente clasificado hasta el momento. También puede tratarse de un mal ajuste del learning rate, pero esto no es posible al utilzar la política de un ciclo con learning rate adaptativo.\\

De forma aislada, como ocurre en la fase de descongelado, este fenómeno no adquiere mayor importancia; pero, dada la periodicidad con la que lo encontramos, es preferible valorar sus resultados con cautela. En este modelo, el tiempo de aprendizaje ha sido de aproximademante 7.7 horas, pero ha requerido el reinicio del entrenamiento en 5 ocasiones debido a problemas de memoria durante el entrenamiento.

\subsubsection{Mobile Net}

Mobilenet es la red con menor número de operaciones de las 3 opciones, debido a que está destinada directamente al uso en dispositivos de baja potencia. Los resultados de esta red, como eran de esperar, son inferiores a los otros 2 modelos probados \ref{tab:mob}.   

\begin{table}[H]
	\centering
	\begin{tabular}{|c|c|c|c|c|}
		\hline
		\textbf{} & \textbf{Precision} & \textbf{Recall} & \textbf{F1-score} & \textbf{Support} \\ \hline
		\textbf{Benign} & 0.87 & 0.86 & 0.86 & 7205 \\ \hline
		\textbf{Malignant} & 0.60 & 0.61 & 0.60 & 2460 \\ \hline
		\textbf{Accuracy} &  &  & 0.80 & 9665 \\ \hline
		\textbf{Macro avg} & 0.73 & 0.74 & 0.73 & 9665 \\ \hline
		\textbf{Weighted avg} & 0.80& 0.80 & 0.80 & 9665 \\ \hline
	\end{tabular}
	\caption{Informe de clasificación de EfficientNet B5}
	\label{tab:mob}
\end{table}

Podemos apreciar como onde se obtienen los peores resultados es en la columna de precisión, donde la diferencia con los modelos anteriores es de 0.11 y 0.16, respectivamente. Tratándose de la clase minoritaria, es un factor clave que este valor sea lo más alto posible, por lo que el redimiento no es el deseado. Sin embargo, resulta una opción atractiva a la hora de ejecutar el modelo dispositivos de baja potencia.

En cuanto a las curvas de aprendizaje (figura \ref{fig:curvasmbin}), se puede apreciar un comportamiento prácticamente ideal de la función  de pérdida en su época de descongelado; tanto el estimador de la pérdida en la función de entrenamiento como el de validación progresan en un intervalo de valores muy similares, sin picos bruscos. Pero, se puede observar cierto aplanamiento del modelo, lo cual puede indicar que hemos alcanzado la capacidad de la red, y sería necesario un modelo más profundo para minimizar aún más la pérdida. En la fase de descongelado, se puede apreciar que no se realizó un apredizaje adecuado, y que la tendencia de validación es al alza, por lo que se trata de un caso de sobreentrenamiento.

l\begin{figure}[H]
	\centering
	\subfigure[Fase de congelado]{\includegraphics[width=0.45\textwidth]{imagenes/perdidasmobv2_freeze_equilib.png}} 
	\subfigure[Descongelado]{\includegraphics[width=0.45\textwidth]{imagenes/perdidasmobv2_unfreeze_equilib.png}} 
	\caption{Curvas del finetuning del modelo binario con Mobilenet V2}
	\label{fig:curvasmbin}
\end{figure}

En resumen, se trata de uno de los modelos más eficientes,    debido a su estructura especialmente optimizada, y requiriendo únicamente 2.26 horas de entrenamiento, pero se trata de un modelo con capacidades limitadas para el problema que estamos tratando.

\subsection{Conclusión: modelo seleccionado}

Atendiendo a los modelos entrenados, considero justificada la elección de Resnet50 para la generación del modelo cuantizado. Aunque se trata del segundo mejor modelo en lo que a accuracy balanceado se refiere, EfficientNet ha provocado demasiados problemas a nivel local, desde errores de memoria disponible de CUDA, hasta el comportamiento exhibido en el punto anterior, donde se producían picos periódicos durante el entrenamiento.

MobileNet es el peor modelo de los 3 en todas las métricas observadas, pero con la ventaja de ofrecer un menor tiempo de inferencia. Por tanto, considero suficientemente justificada la decisión de usar ResNet50 para clasificación binaria, y las pruebas que restan de los modelos especializados. En base a validación, y al comportamiento de la curva de aprendizaje, se trata del modelo más equilibrado.

\begin{figure}[H]
	\centering
	\label {fig:metricas}
	\includegraphics[scale = 0.35]{imagenes/metricas.png}
	\caption{Valores de las métricas de cada modelo}
\end{figure}

Si bien Resnet ofrece un valor cercano, el tiempo de inferencia ofrecido por este es menor. Además, en cuanto a tamaño, EfficientNet posee un número de parámetros excesivo, que provoca una necesidad de almacenamiento elevada. (figura \ref)
\begin{figure}[H]
	\centering
	\label {fig:tam}
	\includegraphics[scale = 0.6]{imagenes/tamestimado.png}
	\caption{Espacio (en MB) requerido por cada modelo}
\end{figure}

Por tanto, realizaremos el proceso de cuantizado con Resnet, ya que se tratará de un modelo más versátil para la clasificación multiclase.

\subsection{Cuantización}

Pytorch contiene un proceso de cuantización directo a partir de los modelos entrenados con FastAI. Proporciona una interfaz de métodos ya preconfigurada que permite exportar el modelo de forma directa a Android, o bien, haciendo uso de cuantización.

La optimización aplicada al realizar cuantización se centra, sobre todo, en la reducción de la precisión numérica del modelo, pero cubre los siguientes aspectos:

\begin{itemize}
	\item Fusión de las capa de convolución y normalización de batch: realiza una integración de los valores de normalización dentro de la capa convolutiva, ya que no será necesario reajustar esos valores para inferencia-
	\item Inserción y plegado de operaciones: utiliza la librería XNNPack, que se trata de un sistema de operaciones en coma flotante especialmente optimizado para ARM, que permite variar la precisión de los valores flotantes según el modelo. Fusiona las operaciones lineales y de convolución, de forma que se realicen en un menor tiempo.
	\item Fusión de la función ReLu con el conjunto de operaciones empaquetado creado con XNNPack en el paso anterior.
	\item Eliminación del dropout, ya que en tiempo de inferencia, se busca el mejor resultado si necesidad de reentreno.
	\item Optimización del grafo interno del modelo correspondiente a la convolución, para hacer que formen parte de un solo bloque raiz y mejore el tiempo de inferencia.
	
En total, la web estima una ganancia del 60\% en tiempo de inferencia de forma teórica. Examinaremos este resultado de forma práctica empleando el modelo elegido, y comparando el funcionamiento de versión optimizada con la original para verificarlo.

\subsubsection{Creación del modelo cuantizado}
\subsubsection{Evaluación de modelo original y cuantizado}
	
\end{itemize}


\section {Clasificación multiclase}

Una vez distinguida si la enfermedad de entrada es benigna o maligna, debemos de indicar de qué tipo de enfermedad se trata. Para ello, crearemos dos modelos especializados: uno en imágenes benignas, y otro especializado en malignas. De esta forma, favorecemos la especialización de modelos, y facilitamos el proceso de entrenamiento al mostrar imágenes dentro del intervalo de tamaño de cada clase; es decir, como la clase benigna es mayoritaria, si evaluamos sus subtipos frente a los malignos, los benignos seguirán teniendo sesgo a favor. Al separarlo en dos, esto podrá evitarse, con el único inconveniente de necesitar un extra de espacio en disco.

El proceso a seguir será exactamente el mismo que con clasificación binaria, pero con ajustes en la cabecera de la red para dar salida a tantas clases de salida como tipos de entrada existen.

\subsection{Modelo de enfermedades malignas}

Comenzaremos por las imágenes malignas. En este caso, al estar utilizando finalmente únicamente el conjunto de entrenamiento de ISIC, disponemos de 4 clases, con un desbalanceo considerable. La clase menos representada se trata de melanoma metastasis. Debido a que su número es muy inferior al del resto de clases,  representando solo el 0.22 \% del total de datos malignos, por riesgo a una incorrecta clasificación de esta enfermedad,  he decidido fusionar dicha clase con melanoma (figura \ref{fig:malas}). Aunque aumentar la clase mayoritaria añadiendo este conjunto de imágenes puede parecer negativo, al tratarse de la misma enfermedad en una fase más avanzada, conseguimos aumentar la variabilidad de las imágenes.

\begin{figure}[H]
	\centering
	\includegraphics[scale = 0.6]{imagenes/countmalignant.png}
	\caption{Conteo de clases malignas antes de agrupar}
	\label {fig:malas}
\end{figure}

\subsubsection{Entrenamiento}

Para el entrenamiento del modelo, hare uso de la misma configuración que la clasificación binaria: imágenes 512 x 512, tamaño de batch de 32, pero con la diferencia de no emplear oversampling, para hacer el modelo lo más fiel posible al conjunto de imágenes. Además, tras la fusión de la clase minoritaria, el desequilibrio entre clases es razonable, siendo de:

\begin{itemize}
	\item Melanoma: 53.95\%
	\item Células basales: 36\%
	\item Células escamosas: 10.5\%
 \end{itemize}
 
 Tras el entrenamiento, se han obtenido las métricas referenciadas en la tabla \ref{tab:malignometrics}. Se puede observar como los resultados son positivos, ya que se ha conseguido alcanzar un valor de accuracy cercano al 84\% de forma balanceada.  
 
\begin{table}[!ht]
	\centering
	\begin{tabular}{|l|c|c|c|c|}
		\hline
		& Precision & Recall & F1-score & Support \\
		\hline
		Basal cell & 0.81 & 0.85 & 0.83 & 908 \\
		Melanoma & 0.88 & 0.92 & 0.90 & 1311 \\
		Squamous cell & 0.66 & 0.38 & 0.48 & 240 \\
		\hline
		Accuracy &  &  & 0.84 & 19464 \\
		Macro avg & 0.79& 0.72& 0.74&2459\\
		Weighted avg&0.84&0.84&0.84&2459\\
		\hline
	\end{tabular}
	\caption{Informe de clasificación para validación maligna}
	\label{tab:malignometrics}
\end{table}

La precisión del modelo también es considerablemente buena, ya que alcanza el 74\%. Son resultados satisfactorios tratándose de un conjunto de datos desbalanceado, y con las limitaciones de tamaño del modelo requeridas.

\subsubsection{Cuantización}


\subsection{Modelo de enfermedades benignas}

Tras evaluar los resultados obtenidos en clasificación maligna, procedemos al caso dual: las enfermedades benignas, o no cancerosas. En este caso, disponemos de un conjunto de datos mucho más amplio, englobando 17 clases distintas. En este caso, el desbalanceo que se produce es un caso límite, en cual la clase mayoritaria, el lunar común, engloba el 81\% del total. Se probarán dos alternativas distintas: la aplicación de la métrica Focal Loss, y la utilización de imágenes complementarias provenientes de los datasets distintos al ISIC.

\begin{figure}[H]
	\centering
	\includegraphics[scale = 0.6]{imagenes/countbenign.png}
	\caption{Conteo de clases malignas antes de agrupar}
	\label {fig:buenas}
\end{figure}

\subsubsection{Entrenamiento}

Conocemos que el desequilibrio del conjunto de entrenamiento es considerable. Pero, antes de probar otras soluciones más sofisticadas, es razonable considerar aplicar la misma técnica que la aplicada en clasificación binaria: el uso de la una métrica que tolere adecuadamente el desbalanceo, como FocalLoss.\\

 
\begin{table}[!ht]
	\centering
	\begin{tabular}{|l|c|c|c|c|}
		\hline
		& Precision & Recall & F1-score & Support \\
		\hline
		       \textbf{NV} & 0.92 & 0.98 & 0.95 & 5864 \\ \hline
		       \textbf{SK} & 0.50 & 0.35 & 0.41 & 345 \\ \hline
		       \textbf{AK} & 0.53 & 0.58 & 0.56 & 265 \\ \hline
		       \textbf{PBK} & 0.71 & 0.74 & 0.72 & 222 \\ \hline
		       \textbf{SL} & 0.28 & 0.12 & 0.17 & 99 \\ \hline
		       \textbf{DER} & 0.75 & 0.18 & 0.29 & 83 \\ \hline
		       \textbf{VL} & 0.68 & 0.55 & 0.61 & 71 \\ \hline
		       \textbf{LK} & 0.67 & 0.07 & 0.12 & 61 \\ \hline
		       \textbf{ACR} & 0.46 & 0.24 & 0.32 & 54 \\ \hline
		       \textbf{LN} & 0.16 & 0.11 & 0.13 & 44 \\ \hline
		       \textbf{AMP} & 0.50 & 0.07 & 0.12 & 29 \\ \hline
		       \textbf{AIMP} & 0.00 & 0.00 & 0.00 & 24 \\ \hline
		       \textbf{WART} & 0.00 & 0.00 & 0.00 & 23 \\ \hline
		       \textbf{ANG} & 0.29 & 0.33 & 0.31 & 6 \\ \hline
		       \textbf{LS} & 0.00 & 0.00 & 0.00 & 5 \\ \hline
		       \textbf{NEU} & 0.00 & 0.00 & 0.00 & 5 \\ \hline
		       \textbf{SCAR} & 0.00 & 0.00 & 0.00 & 5 \\ \hline
		\hline
		Accuracy &  &  & 0.87 & 7205 \\
		Macro avg & 0.38& 0.25& 0.28&7205\\
		Weighted avg&0.84&0.87&0.85&7205\\
		\hline
	\end{tabular}
	\caption{Informe de clasificación para validación benigna (Ordenadas de mayor a menor rep.)}
	\label{tab:benignomalmetrics}
\end{table}

El resultado obtenido deja bastante que desear  (\ref{tab:benignomalmetrics}), debido a que la clase de lunares comunes es la única que ha recibido resultados excelentes; el cuanto a precisión, sin tener en cuenta al lunar común,  los mejores resultados provienen de los dermatofibromas, queratosis benigna pigmenteada,  las lesiones vasculares, y la queratosis liquenoide. Todas ellas, obtienen resultados razonables, aunque a partir de estas clases, los resultados empeoran considerable.

Se ve mayormente reflejado en las tres últimas clases: la cicatrices, los neurofibromas y los lentigo simples. Estas clases, al encontrarse en absoluta minoría, no aportan suficientes caracteristicas al modelo para aprender los filtros adecuados, y por tanto, quedan descartadas por el mismo. Esto nos lleva a tomar la decisión de descartar estas clases, o bien, tratar de equilibrarlas complementando con el resto de datasets.

No nos debemos dejar llevar por el valor del accuracy: este es del 87\%, debido a que prácticamente la completitud de la tasa de aciertos se ve aportada por la clase mayoritaria. Si atendemos a su valor medio, es del únicamente 28\%, que se trata de una cifra más representativa.\\

Antes de probar la eliminación de clases minoritarias, trataremos de dotar al modelo de penalizaciones para las clases menos reprentadas; es decir, de forma explícita, estableceremos una serie de pesos que otorguen más importancia a clasificar ejemplares de clases minoritarias sobre las mayoritarias, calculando para ello un peso inversamente proporcional al recuento de ejemplares. Basta con aplicar la siguiente operación, teniendo en cuenta el valor c como el número de elementos de la clase C:

$$p_C =(\frac{c}{\sum_{i=0}^{num\_clases} c} )^{-1}$$

De esta forma, haremos que cada ejemplar tenga un peso acorde a su proproción en el dataset. Si repetimos el experimento siguiendo este razonamiento, obtenemos los resultados de la tabla \ref{tab:benignomalmejormetrics.}


\begin{table}[!ht]
	\centering
	\begin{tabular}{|l|c|c|c|c|}
		\hline
		& Precision & Recall & F1-score & Support \\
		\hline
	\textbf{NV} & 0.99 & 0.77 & 0.86 & 5864 \\ \hline
	\textbf{DER} & 0.33 & 0.60 & 0.42 & 83 \\ \hline
	\textbf{PBK} & 0.53 & 0.59 & 0.56 & 222 \\ \hline
	\textbf{VL} & 0.37 & 0.88 & 0.52 & 71 \\ \hline
	\textbf{LK} & 0.18 & 0.46 & 0.26 & 61 \\ \hline
	\textbf{AK} & 0.31 & 0.63 & 0.41 & 265 \\ \hline
	\textbf{SK} & 0.58 & 0.83 & 0.69 & 345 \\ \hline
	\textbf{AMP} & 0.20 & 0.39 & 0.26 & 29 \\ \hline
	\textbf{ACR} & 0.23 & 0.67 & 0.34 & 54 \\ \hline
	\textbf{ANG} & 0.12 & 0.43 & 0.19 & 6 \\ \hline
	\textbf{SL} & 0.23 & 0.48 & 0.31 & 99 \\ \hline
	\textbf{LN} & 0.01 & 0.04 & 0.01 & 44 \\ \hline
	\textbf{AIMP} & 0.40 & 0.26 & 0.32 & 24 \\ \hline
	\textbf{WART} & 0.18 & 0.67 & 0.29 & 23 \\ \hline
	\textbf{LS} & 0.06 & 0.20 & 0.09 & 5 \\ \hline
	\textbf{NEU} & 0.00 & 0.00 & 0.00 & 5 \\ \hline
	\textbf{SCAR} & 0.00 & 0.00 & 0.00 & 5 \\ \hline
		\hline
		Accuracy &  &  & 0.74 & 7205 \\
		Macro avg & 0.28& 0.46& 0.33&7205\\
		Weighted avg&0.87&0.74&0.78&7205\\
		\hline
	\end{tabular}
	\caption{Informe de clasificación para validación benigna (Ordenadas de mayor a menor rep.)}
	\label{tab:benignomalmejormetrics}
\end{table}

Los resultados han mejorado, escalando hasta el 46\% de accuracy balanceado. Sin embargo, estos siguen siendo pobres, y las clases minoritarias siguen siendo un problema. Por tanto, las 3 clases de menor representación sería oportuno eliminarlas, debido a la imposibilidad de unir estas enfermedades a otras clases existenetes. En el caso de los lentigo, que se encuentran divididos en 3 clases, podemos unirlos en un una única clase para mejorar los resultados.

\subsubsection{Cuantización}


	 
	 \chapter{Cuantización}

La cuantización, como ya definimos en el apartado \ref{cap:cuantización}, se trata de un proceso por el cual tomamos un modelo entrenado para arquitecturas de escritorio de la forma habitual, y realizamos una serie de operaciones de simplificación o transformaciones del formato numérico para obtener un modelo cuyos requisitos de espacio y tiempos de inferencia se ven reducidos considerablemente. 

Este procedimiento es el núcleo del proyecto, y el objetivo real de estudio de modelos para tecnologías móviles: experimentar cuál es la diferencia en tiempo y uso de memoria de la arquitectura tradicional sobre la optimizada.

Para realizar este proceso, emplearemos los mecanismos de Pytorch\cite{paszke2019pytorch} dedicados a la transformación e implementación de modelos convolucionales en Android: Pytorch mobile  \cite{pmobile}. En este punto, analizaremos las optimizaciones realizadas por este framework, y las ventajas y desventajas de dicho procedimiento.

\section{Optimizaciones de Pytorch mobile}

Pytorch mobile ofrece una serie de optimizaciones predeterminadas bastante eficientes en la optimización de modelos para arquitecturas ARM. Para comprender en detalle su funcionamiento, profundizaremos en el formato de guardado utilizado, y las optimizaciones llevadas a cabo para conseguir las ganancias en rendimiento especificadas en \cite{pmobile}.

\subsection{Capacidades del framework}
Pytorch \cite{paszke2019pytorch}, a través de sus dependecias mobile, tiene la capacidad de transformar en unos sencillos pasos un modelo entrenado de la forma habitual a un modelo simplificado optimizado para dispositivos móviles. Incluye varios métodos relacionados con la exportación que permiten optimizar o no la salida del modelo, adecuándolo a nuestras necesidades. En su página oficial, podemos encontrar un grafo donde se describe este proceso \cite{pmobile}:

\begin{figure}[H]
	\centering
	\includegraphics[scale = 0.5]{imagenes/pytorch-mobile.png}
	\caption{Optimización de modelos con Pytorch Mobile \cite{pmobile}}
	\label {fig:mobileprocess}
\end{figure}

Todo este procedimiento se realiza sobre un lenguaje intermedio de alto rendimiento, perteneciente al núcleo de Pytorch, llamado Torchscript \cite{torchscript}. Este lenguaje permite crear modelos escalables y optimizables sin necesidad de dependencias Python en el sistema destino. Está pensado, principalmente, para su aplicación en entornos de producción, los cuales estén basados en otros lenguajes, como C o similares. La conversión del formato habitual de Pytorch, y el establecido por Torchscript es prácticamente inmediato, ya que la conversión se realiza de forma interna al aplicar el comando \textit{torch.jit.script(model)}, siendo model el modelo que queremos transformar.

Como inconvenientes, no podremos utilizar herramientas de explicabilidad de modelos como GradCam \cite{Selvaraju_2019}, debido a la pérdida de toda la información no esencial sobre los gradientes en el proceso de exportación a producción.

La portabilidad al ser independizado de Python es el principal motivo de su uso para realizar la transformación del modelo a dispositivos móviles, y puede ser ejecutado tanto en PC como en smartphones, facilitando así el proceso de comparativa entre ambos.

Si bien el lenguaje soporta Android y iOS,  este último fue soportado oficialmente tras el comienzo de la realización de este TFG. Debido a ello, este proyecto fue concebido inicialmente para ser probado en dispositivos Android, ya que el hardware de estos dispositivos es mñas accesible, y dispone de un mejor soporte de librerías y entornos de desarrollo. iOS, al tratarse de un SO limitado a dispositivos Apple, y difícilmente emulable si no se dispone de dispositivos de la marca, requiere unas condiciones adicionales que no se pueden cumplir por limitaciones económicas, pues no disponemos de dispositivos de la marca. Por este motivo, este proyecto se limita únicamente al desarrollo sobre terminales Android, más accesibles y frecuentes en el mercado.

\subsection{Optimizaciones disponibles}

Pytorch mobile dispone para Android de una lista de 5 optimizaciones posibles a realizar sobre el modelo. Todas ellas, seleccionables de forma independiente, o aplicadas en conjunto por defecto, cubren los siguientes aspectos:

\begin{itemize}
	\item Fusión de las capa de convolución y normalización de batch: realiza una integración de los valores de normalización dentro de la capa convolutiva, ya que no será necesario reajustar esos valores para inferencia-
	\item Inserción y plegado de operaciones: utiliza la librería XNNPack, que se trata de un sistema de operaciones en coma flotante especialmente optimizado para ARM, que permite variar la precisión de los valores flotantes según el modelo. Fusiona las operaciones lineales y de convolución, de forma que se realicen en un menor tiempo.
	\item Fusión de la función ReLu con el conjunto de operaciones empaquetado creado con XNNPack en el paso anterior.
	\item Eliminación del dropout, ya que en tiempo de inferencia, se busca el mejor resultado si necesidad de reentreno.
	\item Optimización del grafo interno del modelo correspondiente a la convolución, para hacer que formen parte de un solo bloque raiz y mejore el tiempo de inferencia.	
\end{itemize}

En total, la web  \cite{pmobile} estima una ganancia aproximada del 60\% en tiempo de inferencia de forma teórica. Sin embargo, este valor puede oscilar en función de los datos que se estén evaluando, la arquitectura que deseamos cuantizar, y las capacidades de cálculo del dispositivo objetivo. A continuación, estudiaremos los resultados ofrecidos, y si merece la pena la aplicación de este proceso.

\section{Creación de los modelos cuantizados}

La transformación de los modelos entrenados a su versión cuantizada puede realizarse de forma inmediata, ya que en los pasos anteriores, el modelo entrenado sobre cada uno de los casos (clasificación binaria, y multiclase para benigno y maligno) fue almacenado en un fichero de formato .pt (extensión de modelo Pytorch).

Al tratarse de una conversión post-entrenamiento, no se requiere ningún ajuste de parámetros previo durante el entrenamiento, lo cual significa que el mecanismo puede aplicarse sobre los modelos entrenados de la forma habitual, sin necesidad de emplear métodos específicos como el entrenamiento apto para cuantizado (QAT, o Quantization Aware Training  \cite{kuzmin2024fp8}), el cual es más complejo de ajustar, pero puede ofrecer algunas ventajas marginales en cuanto a precisión.

Para cuantizar el modelo tras su entrenamiento, basta con realizar los siguientes pasos:
\begin{enumerate}
	\item Cargar el modelo entrenado previamente. Como en nuestro caso, hemos realizado el salvado del modelo y procedemos a realizar la cuantización en el mismo en el mismo script, podemos directamente emplear el objeto Learner que almacena el modelo completo. En caso de que se quisiese leer de disco duro, bastaría con usar la función model.load().
	\item Activar el modo de evaluación del modelo. De esta forma, bloqueamos los parámetros libres de las capas de normalización de batches, y la aplicación de dropout.
	\item En este punto, podemos optar por dos opciones:
	\begin{enumerate}
		\item Exportar el modelo a Android sin ninguna optimización estructural, más allá de la adaptación de tipos de flotantes (función \textit{save\_for\_lite\_interpreter})
		\item Exportar el modelo aplicando todas las optimizaciones anteriores mencionadas para obtener el modelo más eficiente posible (función \textit{export\_for\_mobile})
	\end{enumerate}
	\item (Opcional) : exportar el modelo de escritorio en formato torch script. Se trata de un lenguaje de representación universal del modelo, que puede ser empleado con Python nativo, y ser transferido a otros frameworks de trabajo habituales.
		
\end{enumerate}

Para realizar la comparativa de forma lo más exhaustiva posible, tomaremos el modelo de Android, y lo evaluaremos con el conjunto completo de imágenes de test, de forma que la valoración obtenida sea completamente representativa.

El pseucódigo resultante es batante sencillo, y su explicación queda prácticamente autocontenida gracias al uso de nombres de función represetativos:

\begin{algorithm}[H]
	\label{fig:cuantizado}
	\caption{Proceso de cuantizado de modelos a Android}
	\begin{algorithmic}
		\Procedure{exportar\_modelo}{learner: Learner Model Object, savename : String}
	
		\State model = learner.model.to('cpu') : Model
		\ State model.eval()
		
		 \State scripted\_module = torch.jit.script(nn.Sequential(model, Softmax(dim=n))) \\
		\Comment{Exportación del modelo original}
	 	 \State scripted\_module.save(savename + ``Full.pt")\\  
		 \Comment{Exportación del modelo optimizado}
		 \State optimized\_scripted\_module = optimize\_for\_mobile(scripted\_module) 
		 \State optimized\_scripted\_module.\_save\_for\_lite\_interpreter(savenameFull+ ``\_androidOptimized.pt")
	
		\EndProcedure
		
	\end{algorithmic}
\end{algorithm}

Podemos observar que el proceso es trivial, a excepción del último punto, donde se realiza la definición de un objeto de tipo sencuencial. Esto se debe que, cuando exportamos el modelo a Android, existen dos posibles alternativas para interpretar la capa de salida del mismo:
\begin{enumerate}
	\item Interpretando el tensor de salida de la función Softmax por columnas, de forma que la clase predicha para cada imagen será el máximo valor softmax encontrado en la columna en cuestión. En Python, se le conoce como máximo en dimensión 0, al tratarse del máximo de todas las filas.
	\item Interprentando el tensor de salida por filas, donde una imagen de entrenada recibe como etiqueta de salida el índice de la posición de fila cuyo valor es mayor. En este caso, se considera como dimensión 1, ya que se trata del valor máximo encontrado entre las columnas.
 \end{enumerate}

Por tanto, antes de exportar el modelo, será necesario especificar la dimensión de lectura. En versiones anteriores, esta dimensión era elegida de forma implícita por la función, pero ahora, debemos especificarla explícitamente para tener un correcto funcionamiento. Como en nuestro caso, solo tenemos una dimensión de etiquetas por modelos, debemos seleccionar el valor máximo por columnas, y valor a especificar es dim=1. Basta con crear un objeto de tipo Softmax, y especificar como parámetro \textit{dim}, el valor dim = 1.

Para concatenar el modelo original, y la nueva capa específica para la función Softmax, es necesario añadir ambos pasos a un pipeline, usando para la clase Sequential de Pytorch, que no es más que la creación de un modelo por capas.\\

Por último, para el proceso de comparación de modelos, realizaremos dos tareas:
\begin{enumerate}
	\item Evaluación del conjunto de test. Se realiza la inferencia completa del conjunto de test para estudiar la pérdida de rendimiento del modelo cuantizado frente al original haciendo uso un estimador insesgado, debido a la reserva de las imágenes hasta este momento.
	\item Comparativa de tiempos de inferencia. Realizaremos una comparativa de ejecución de ambos modelos en Android para evaluar la ganancia en tiempo tras aplicar cuantización, y si esta compensa las pérdidas de prestaciones. Debido a las limitaciones de la arquitectura ARM, evaluaremos el tiempo sobre una muestra del total, debido a la generación de calor y el desgaste ocasionado por la evaluación del test, que al tratarse del 40\% del cómputo global de imágenes, incluye 21478 fotografías.

\end{enumerate}

\section{Comparativa de los modelos original y cuantizado}

En este apartado, una vez comprendido el proceso de exportación y cuantización, examinaremos el rendimiento de cada uno de los 3 modelos creados para el proyecto. Para ello, evaluaremos el conjunto de test, intacto hasta esta fase, para conocer el rendimiento real de los modelos originales, y los compararemos con cada uno de los modelos cuantizados.

\subsection{Cuantización del modelo binario}

El modelo de clasificación binaria, que decide si la enfermedad se trata de un caso benigno o maligno, es la capa más importante de la arquitectura de dos niveles implementada. Su rendimiento es crítico dentro del funcionamiento del modelo, debido a que la confusión entre subtipos de enfermedades tiene un coste relativamente pequeño, pero un error entre una enfermedad cancerígena o benigna tiene un elevado coste para el paciente en consecuencias. Por tanto, es clave evaluar el modelo con el conjunto de test para conocer la bonanza del mismo, y estudiar si las ganancias en espacio y tiempo de inferencia justifican la cuantización.\\

Evaluaremos el conjunto de test bajo las mismas condiciones, sin ningún tipo de preprocesado, más allá de la normalización de las imágenes y el reescalado a 512 x 512. Es muy importante destacar que estos valores son los mismos que los empleados en la fase de entrenamiento, ya que recalcular la media y desviación típica podría crear mejores resultados pero no ser una prueba representativa, ya que al realizar una inferencia de un diagnóstico, esta normalmente se realiza con una única imagen, y no habría datos suficientes para hallar la media y desviación típica. Debemos trabajar siempre con los parámetros de entrenamiento para que la distribución sea la misma, verificando así las condiciones ya vista acerca de disponer de un estimador no sesgado con test.

A continuación, comparamos la inferencia de ambos modelos y la ganancia en tiempo de inferencia entre ellos.

\subsubsection{Métricas de inferencia en test}

En primer lugar, comenzamos por evaluar la pérdida de rendimiento del modelo cuantizado. Para ello, ejecutamos el proceso de inferencia con ambos modelos, y obtenemos la tabla de resultados para ambos.
Los resultados de ambos modelos, original y cuantizado, podemos apreciarlos en las tablas \ref{tab:bintestorig} y \ref{tab:bintestquant} respectivamente.

\begin{table}[!ht]
	\centering
	\begin{tabular}{|c|c|c|c|c|}
		\hline
		~ & Precision & Recall & F1-score & Support \\ \hline
		Benign & 0.90 & 0.89 & 0.89 & 16010 \\ 
		Malignant & 0.68 & 0.70 & 0.69 & 5468 \\ \hline
		~ & ~ & accuracy & 0.84 & 21478 \\ \hline
		Macro avg & 0.79 & 0.80 & 0.79 & 21478 \\ 
		Weighted avg & 0.84 & 0.84 & 0.84 & 21478 \\ \hline
	\end{tabular}
	\caption{Resultados de inferencia en test para el modelo original}
	\label{tab:bintestorig}
\end{table}



\begin{table}[!ht]
	\centering
	\begin{tabular}{|c|c|c|c|c|}
		\hline
		~ & Precision & Recall & F1-score & Support \\ \hline
		Benign & 0.90 & 0.89 & 0.89 & 16010 \\ 
		Malignant & 0.68 & 0.70 & 0.69 & 5468 \\ \hline
		~ & ~ & accuracy & 0.84 & 21478 \\ \hline
		Macro avg & 0.79 & 0.79 & 0.79 & 21478 \\ 
		Weighted avg & 0.84 & 0.84 & 0.84 & 21478 \\ \hline
	\end{tabular}
	\caption{Resultados de inferencia en test para el modelo cuantizado}
	\label{tab:bintestquant}
\end{table}

Podemos apreciar que el resultado obtenido en el modelo original es muy similar a los obtenidos con la partición de validación, por lo que el modelo ha funcionado correctamente con datos los cuales no ha visto hasta el momento. Concretamente, en media, el modelo original en validación nos ofrecía el mismo resultado:  0.79 de precisión, 0.79 de recall,  0.79  de F1 Score y 0.84 de accuracy (\ref{tab:resultsbinrn50}). Si comparamos en detalle, hemos perdido recall y precisión en la clase maligna a favor de un ligero aumento de dichas métricas en la clase benigna, arrojando el mismo resultado medio. 

Si ahora estudiamos el caso optimizado y cuantizado para Android, observamos que sorprendentemente, no se han perdido prestaciones: las métricas son prácticamente idénticas a las obtenidas en el modelo original sin optimizar, a diferencia del recall medio, una centésima menor, lo cual es posible que se deba a algún ejemplo cuya valoración se ha visto erróneamente clasificada, pero cuyo impacto global ha sido. Podríamos afirmar que para este modelo, el impacto de optimización no ha afectado al rendimiento del modelo, el cual ha conservado las mismas prestaciones, ya que la diferencia en recall es prácticamente despreciable.

Esto ha sido posible gracias a las capacidades del framework desarrollado por Pytorch, las cuales han funcionado adecuadamente, al haber elegido un modelo totalmente compatible con el proceso de cuantización y optimización \cite{comptquant}.\\

En conclusión, para el modelo de clasificación binaria, obtenemos un modelo optimizado sin pérdida de rendimiento.

\subsubsection{Tiempo de inferencia}

Tras obtener unos buenos resultados en las métricas del modelo, podemos evaluar el tiempo medio de inferencia del modelo cuantizado frente al original en Android. Este factor es clave, ya que si el modelo cuantizado no ofrece resultados más eficientes en tiempo, el proceso de cuantizado no supone ninguna ventaja sobre el modelo original, sobre todo en este modelo, el cual ha demostrado no perder prestaciones en cuanto a métricas.

Los resultados se han tomado teniendo en cuenta únicamente la inferencia de 200 imágenes, debido a que la evaluación de las 21478 imágenes en un dispositivo móvil produce un sobrecalentamiento y desgaste considerable sobre los componentes. El terminal empleado hace uso de un Snapdragon 855, y 8GB de RAM, siendo estos componentes del año 2019, y arroja los resultados de la tabla \ref{infbintmp}.

\begin{table}[H]
	\centering
	\begin{tabular}{|c|c|c|}
		\hline
		Modelo & Tiempo total (ms) & Tiempo medio (ms) \\ \hline
		Original & 114342 & 571,71 \\ \hline
		Cuantizado & 72352 & 361,76 \\ \hline
	\end{tabular}
	\caption{Tiempos de inferencia para caso binario en Android}
	\label{infbintmp}
\end{table}

Tras la evaluación del modelo mediante 200 imágenes, podemos apreciar una ganancia significativa en tiempo, superior a 210ms por imagen. Si calculamos la ganancia en velocidad del modelo cuantizado sobre el original, obtenemos:

$$Ganancia = \frac{t\ original}{t\ cuantizado} = \frac{571,71}{361,76} = 1.58$$

Es decir, el modelo optimizado nos aporta una ganancia de 1.58, lo cual supone aproximadamente un 40\% menos del tiempo de inferencia. Este resultado evidencia que el modelo cuantizado ofrece unas mejores prestaciones que el original. Teniendo en cuenta que el modelo cuantizado no pierde bondad de métricas al ser evaluado con el conjunto de test frente al original, queda demostrado que el modelo cuantizado supone una mejora considerable y debe ser el modelo a utilizar en un dispositivo móvil. La penalización obtenida es nula, y las ventajas, cuantiosas.

\subsection{Cuantización del modelo multiclase maligno}

El modelo de clasificación de enfermedades malignas se del primero de los casos de clasificación multiclase. A diferencia del anterior, este modelo aporta una salida más compleja, al deber distinguir entre melanomas, cáncer de células basales y cáncer de células escamosas. Esto nos deja un con vector de salida de 3 componentes, una por cada clase, que el modelo deberá aportar tanto en el modelo original como el cuantizado.

De la misma manera que con el caso de clasificación binaria, compararemos los dos modelos mediante la evaluación del conjunto de test, teniendo en cuenta que las imágenes ha de ser preprocesadas con los mismos parámetros y configuraciones del conjunto de test, y no recalcular ninguno de los valores. Este preprocesado, al igual que en entrenamiento, incluye:

\begin{enumerate}
	\item Renombrado de etiquetas. Al igual que en el modelo de entrenamiento decidimos fusionar las clases ``melanoma'' y ``melanoma metastasis'', debemos de aplicar la misma transformación en test. Si no lo hiciésemos, el modelo funcionaría adecuadamente, pero no podríamos evaluar correctamente las métricas que nos permiten ver si el rendimiento es el esperado.
	 \item Normalización y resize: se utilizan la media y desviación típica de ImageNet\cite{5206848}, empleadas en entrenamiento, y se aplica un resize de 512 x 512 px siguiendo la misma distribución.
\end{enumerate}

Una vez realizado este preprocesado, podemos calcular la inferencia de los dos modelos.

\subsubsection{Métricas de inferencia en test}
Para evaluar la calidad del modelo cuantizado, se tomarán tanto el modelo original como el cuantizado, ambos exportados durante el aprendizaje realizado en el capitulo 5, y obtendremos la tabla resumen de métricas para realizar la comparativa.
Los resultados  podemos apreciarlos en las tablas \ref{tab:maltestorig} (original) y \ref{tab:maltestquant}(cuantizado).

\begin{table}[!ht]
	\centering
	\begin{tabular}{|c|c|c|c|c|}
		\hline
		~ & Precision & Recall & F1-score & Support \\ \hline
		Basal Cell C. & 0.80 & 0.84 & 0.82 & 1969 \\
		Melanoma & 0.87 & 0.92 & 0.89 & 2950 \\
		Squamous Cell C. & 0.65 & 0.37 & 0.47 & 549 \\ \hline
		Accuracy & ~ & ~ & 0.83 & 5468 \\ \hline
		Macro avg & 0.78 & 0.71 & 0.73 & 5468 \\
		Weighted avg & 0.83 & 0.83 & 0.83 & 5468 \\ \hline
	\end{tabular}
	\caption{Resultados de inferencia en test para el modelo original}
	\label{tab:maltestorig}
\end{table}

\begin{table}[!ht]
	\centering
	\begin{tabular}{|c|c|c|c|c|}
		\hline
		~ & Precision & Recall & F1-score & Support \\ \hline
		Basal Cell C. & 0.80 & 0.84 & 0.82 & 1969 \\
		Melanoma & 0.87 & 0.92 & 0.89 & 2950 \\
		Squamous Cell C. & 0.65 & 0.37 & 0.47 & 549 \\ \hline
		Accuracy & ~ & ~ & 0.83 & 5468 \\ \hline
		Macro avg & 0.78 & 0.71 & 0.73 & 5468 \\
		Weighted avg & 0.83 & 0.83 & 0.83 & 5468 \\ \hline
	\end{tabular}
	\caption{Resultados de inferencia en test para el modelo cuantizado}
	\label{tab:maltestquant}
\end{table}

En este subproblema, encontramos que ambos modelos arrojan exactamente el mismo resultado, por lo que el proceso de optimización no ha conllevado ninguna pérdida para el rendimiento del modelo, lo cual es un caso totalmente ideal. Para ver su rendimiento real, podemos comparar el modelo con el resultado arrojado por el conjunto de validación:  0.79 de precisión,  0.72 de recall, y  0.74 de F1 Score (tabla \ref{tab:malignometrics}).  Podemos apreciar que, por tanto, el conjunto de validación fue un correcto estimador de las funciones de pérdida y métricas fuera de la muestra.

Por tanto, el modelo cuantizado, y a su vez, el original, ofrecen un muy buen resultado, acorde a lo estimado durante el entrenamiento, y sin pérdida de precisión de clasificación en el caso del modelo cuantizado. 

\subsubsection{Tiempo de inferencia}

Tras observar que el modelo cuantizado no supone pérdidas frente el modelo original, un caso ideal, nos disponemos a evaluar el tiempo de inferencia por imagen, de la misma manera que se realizó con el modelo binario.

Se usará el mismo dispositivo, empleando un subconjunto para la inferencia de 200 imágenes. Los resultados se pueden apreciar en la tabla \ref{infmal}

\begin{table}[H]
	\centering
	\begin{tabular}{|c|c|c|}
		\hline
		Modelo & Tiempo total (ms) & Tiempo medio (ms) \\ \hline
		Original & 215693 & 1078,465	 \\ \hline
		Cuantizado & 136912 & 684,56 \\ \hline
	\end{tabular}
	\caption{Tiempos de inferencia para caso multiclase maligno en Android}
	\label{infmal}
\end{table}

A la vista de los resultados, podemos apreciar una ganancia significativa en tiempo, de alrededor de 394ms por imagen. A priori, parece que el ahorro en tiempo es mayor que en el caso binario, pero si calculamos la ganancia en velocidad, obtenemos:

$$Ganancia = \frac{t\ original}{t\ cuantizado} = \frac{1078,465}{684,56} = 1.575$$

Es decir, el modelo optimizado nos aporta una ganancia de 1.575, lo cual es proporción es prácticamente la misma ganancia que la conseguida con la cuantización en caso binario. De nuevo, supone aproximadamente un 37\% menos de tiempo de inferencia sobre el modelo original. 
En conclusión, la arquitectura final vuelve a ser la versión cuantizada, ya que nos permite mejorar considerablemente el rendimiento sin pérdida de prestaciones.

\subsection{Cuantización del modelo multiclase benigno}

Tras el estudio de los modelos para el caso de clasificación binaria y el caso multiclase maligno, es el turno del modelo para la detección de enfermedades benignas. Este modelo es el más complejo de los 3, ya que requiere dar como salida un grado de granularidad elevado en cuanto a diagnósticos, debiendo distinguir entre 18 enfermedades distintas.

Los resultados de esta modelo no fueron tan prometedores debido a la alta similitud de las clases, y el uso de un modelo poco profundo para acometer la clasificación. Por tanto, los resultados del modelo cuantizado podrían verse más afectados que en los dos casos anteriores.

Al igual que en la preparación de datos del modelo maligno, debemos realizar el mismo preprocesado que con las clases de entrenada de entrenamiento:

\begin{enumerate}
	\item Renombrado de etiquetas. Realizar el ajuste de etiquetas a valores numéricos para una mayor facilidad de cálculo de las métricas de evaluación.
	\item Normalización y reescalado: utilización de la media y desviación típica de ImageNet\cite{5206848}, y reescalado a 512 x 512 px.
\end{enumerate}

Tras dichos ajustes, podemos proceder a la medición de las métricas de test y el tiempo de inferencia en el dispositivo Android.

\subsubsection{Métricas de inferencia en test}

De la misma forma que los casos anteriores, procedemos a la evaluación del conjunto de test, filtrando únicamente aquellas imágenes con lesiones de naturaleza benigna. Los resultados de los modelos original y cuantizado quedan reflejados en las tablas \ref{tab:testbenorig} y \ref{tab:testbenquant}, respectivamente:

\begin{table}[!ht]
	\centering
	\begin{tabular}{|c|c|c|c|c|}
		\hline
		~ & Precision & Recall & F1-score & Support \\ \hline
		NV & 0.99 & 0.78 & 0.87 & 13067 \\ \hline
		DER & 0.31 & 0.61 & 0.41 & 770 \\ \hline
		PBK & 0.53 & 0.59 & 0.56 & 547 \\ \hline
		VL & 0.39 & 0.87 & 0.54 & 536 \\ \hline
		LK & 0.20 & 0.47 & 0.28 & 225 \\ \hline
		AK & 0.28 & 0.61 & 0.38 & 168 \\ \hline
		SK & 0.59 & 0.83 & 0.69 & 139 \\ \hline
		AMP & 0.20 & 0.43 & 0.27 & 125 \\ \hline
		ACR & 0.29 & 0.72 & 0.42 & 120 \\ \hline
		ANG & 0.11 & 0.36 & 0.17 & 96 \\ \hline
		SL & 0.04 & 0.07 & 0.05 & 56 \\ \hline
		LN & 0.02 & 0.08 & 0.03 & 48 \\ \hline
		AIMP & 0.27 & 0.23 & 0.25 & 48 \\ \hline
		WART & 0.35 & 0.50 & 0.41 & 28 \\ \hline
		LS & 0.19 & 0.53 & 0.28 & 17 \\ \hline
		NEU & 0.00 & 0.00 & 0.00 & 10 \\ \hline
		SCAR & 0.00 & 0.00 & 0.00 & 10 \\ \hline
		Accuracy & ~ & ~ & 0.75 & 16010 \\ \hline
		Macro avg & 0.28 & 0.45 & 0.33 & 16010 \\ \hline
		Weighted avg & 0.87 & 0.75 & 0.79 & 16010 \\ \hline
	\end{tabular}
	\caption{Resultado obtenido al inferir el conjunto de test (sin cuantizado)}
	\label{tab:testbenorig}
\end{table}

\begin{table}[H]
	\centering
	\begin{tabular}{|c|c|c|c|c|}
		\hline
		~ & Precision & Recall & F1-score & Support \\ \hline
		NV & 0.99 & 0.78 & 0.87 & 13067 \\ \hline
		DER & 0.31 & 0.60 & 0.41 & 770 \\ \hline
		PBK & 0.53 & 0.59 & 0.56 & 547 \\ \hline
		VL & 0.39 & 0.87 & 0.54 & 536 \\ \hline
		LK & 0.21 & 0.48 & 0.29 & 225 \\ \hline
		AK & 0.27 & 0.61 & 0.38 & 168 \\ \hline
		SK & 0.59 & 0.83 & 0.69 & 139 \\ \hline
		AMP & 0.21 & 0.45 & 0.28 & 125 \\ \hline
		ACR & 0.30 & 0.72 & 0.42 & 120 \\ \hline
		ANG & 0.11 & 0.36 & 0.17 & 96 \\ \hline
		SL & 0.04 & 0.07 & 0.05 & 56 \\ \hline
		LN & 0.02 & 0.08 & 0.03 & 48 \\ \hline
		AIMP & 0.28 & 0.23 & 0.25 & 48 \\ \hline
		WART & 0.35 & 0.50 & 0.41 & 28 \\ \hline
		LS & 0.19 & 0.53 & 0.28 & 17 \\ \hline
		NEU & 0.00 & 0.00 & 0.00 & 10 \\ \hline
		SCAR & 0.00 & 0.00 & 0.00 & 10 \\ \hline
		Accuracy & ~ & ~ & 0.75 & 16010 \\ \hline
		Macro avg & 0.28 & 0.45 & 0.33 & 16010 \\ \hline
		Weighted avg & 0.87 & 0.75 & 0.79 & 16010 \\ \hline
	\end{tabular}
	\caption{Resultado obtenido al inferir el conjunto de test (cuantizado)}
	\label{tab:testbenquant}
\end{table}

Los resultados son de nuevo positivos, ya que en lo que corresponde a los valores medios, no existe pérdida destacable. Si bajamos de nivel, y comparamos clase a clase, podemos ver que sí que existe alguna diferencia, como, una precisión una centésima menor en el diagnóstico de la queratosis actínica (abreviada como AK). Sin embargo, dicha pérdida de precisión es compensada por otros valores que, sorprendentemente, adquieren un mejor resultado en el modelo cuantizado que el modelo original, aunque menor a una centésima.

A grandes rasgos, y en promedio, ambos modelos son equivalentes en características en lo que a métricas se refiere. El diagnóstico, aunque ruidoso por ser un modelo más simple de lo ideal, es muy cercano al valor visto en validación (tabla \ref{tab:benignomalmejormetrics}), por lo que podemos confirmar que el aprendizaje se realizó de forma satisfactoria.

\subsubsection{Tiempo de inferencia}
Una vez evaluado el rendimiento de ambos modelos, podemos verificar las ganancias en rendimiento obtenidas con el procedimiento de cuantización. Al tratarse de un problema con mayor número de salidas, dicha reducción será menos notable. Empleando nuestro criterio de evaluar el promedio de 200 imágenes, obtenemos el resultado mostrado en la tabla \ref{tab:infbin}.
\begin{table}[H]
	\centering
	\begin{tabular}{|c|c|c|}
		\hline
		Modelo & Tiempo total (ms) & Tiempo medio (ms) \\ \hline
		Original & 114972 & 574,86	 \\ \hline
		Cuantizado & 73774 & 368,87 \\ \hline
	\end{tabular}
	\caption{Tiempos de inferencia para caso binario en Android}
	\label{tab:infbin}
\end{table}

A la vista de los resultados, podemos apreciar una ganancia significativa en tiempo, de alrededor de 206ms por imagen. Si calculamos la ganancia en velocidad, obtenemos:

$$Ganancia = \frac{t\ original}{t\ cuantizado} = \frac{574,86}{368,87} = 1,558$$

Por tanto, el modelo optimizado nos aporta una ganancia de 1,558, que si bien es ligeramente inferior a los dos modelos analizados anteriormente, ofrece una ventaja muy similar en rendimiento, con la ventaja de perder precisión en sus salidas de forma apreciable.


\section{Conclusión de la cuantización}

Una vez evaluados los 3 modelos, podemos dar el veredicto final de los resultados
\begin{enumerate}
	\item La generalización de los 3 modelos al realizar la inferencia del conjunto de test ha sido adecuada, ya que las métricas obtenidas son prácticamente equivalentes a las obtenidas en validación por cada uno de los modelos. Esto demuestra la efectividad del método seguido, realizando las particiones de entrenamiento, validación y test.
	\item El proceso de optimización y cuantización ofrecido por Pytorch mobile \cite{pmobile} ha demostrado ser efectivo: la pérdida de rendimiento es mínima, siendo inferior a $10^{-3}$ en todos los casos, ya que no es apreciable en el resultado de dos decimales de precisión.
	\item Las ganancias en tiempo de inferencia son considerables: en los 3 casos, el factor de ganancia ha sido superior a 1.5 (visible gráficamente en la figura \ref{fig:gananciasquant}), lo cual indica que el modelo cuantizado sería capaz de inferir aproximadamente 3 imágenes en el tiempo en el que el modelo original infiere 2. 
	\end{enumerate}
	
	Las ganancia en rendimiento de inferencia, sumado a la inexistencia de pérdida de precisión en los casos estudiados, convierte a la cuantización en la solución ideal para la implementación del modelo en dispositivos móviles.
	
	\begin{figure}[H]
		\centering
		\includegraphics[scale = 0.55]{imagenes/tiempoinferencia.png}
		\caption{Comparativa de ganancias al emplear  modelos cuantizados}
		\label{fig:gananciasquant}
	\end{figure}

	
    \chapter{Diseño de la aplicación}

Una vez han sido creados, entrenados y verificados los modelos cuantizados para teléfono móvil, podemos proceder con el diseño e implementación de la aplicación. Su objetivo será, mostrar de forma visual y cómoda para el usuario habitual, la utilidad de los modelos entrenados y optimizados en los capítulos 5 y 6 de este trabajo.

Al tratarse de una aplicación para uso local del dispositivo, y de funcionalidades concretas y bien delimitadas, como lo es la prueba de los modelos en casos reales, esta no será excesivamente compleja. Aun así, en este capítulo, abordaremos su diseño desde el punto de vista de la ingeniería del software, para detallar correctamente las funcionalidades necesarias y los requisitos que debe cumplir para que el resultado sea el esperado.

\section{Descripción del problema. Requisitos y definiciones.}

Comenzaremos por analizar la descripción de la aplicación que queremos conseguir, teniendo en cuenta los términos relevantes del texto y extrayendo los requisitos a cumplir por la misma.

\subsection{Proceso de obtención de requisitos}

En  primer lugar, se describe el problema a resolver, y los requisitos del mismo.

\subsubsection{Descripción}
Tras el entrenamiento de tres modelos de aprendizaje profundo orientados a la prognosis y el diagnóstico de enfermedades cancerígenas de la piel, es necesario crear una plataforma simple capaz de demostrar al usuario la versatilidad y precisión de los modelos haciendo uso de imágenes que contienen lesiones cutáneas, de manera que los modelos sean capaces de diagnosticar si la imagen se trata de un caso positivo o negativo de una enfermedad cancerígena, y clasificar su posible diagnóstico empleando dichos modelos.

El diagnóstico de imágenes debe realizarse teniendo en cuenta la existencia de un usuario, que puede tomar fotografías de este tipo de lesiones, y que debe recibir el resultado del examen, para que este pueda tomar medidas en caso de tratarse de enfermedades sospechosas y visitar al dermatólogo cuanto antes. Para ello, debe clasificarse, en primer lugar, si se trata de una enfermedad maligna o benigna, y posteriormente, un posible subtipo de enfermedad.

De forma detallada, el proceso sería el siguiente:

\begin{enumerate}
	\item El usuario ingresa en el sistema, y requieren al mismo la realización de un diagnóstico por medio de imagen.
	\item El usuario obtiene la imagen de la enfermedad que desea analizar empleando para ello la cámara del dispositivo, o bien, una fotografía ya realizada en el pasado que desee clasificar.
	\item A continuación, se realiza el diagnóstico empleando dos niveles de análisis: 
	\begin{enumerate}
		\item Primero, se realiza una clasificación más general para identificar si se trata de una enfermedad benigna, o de un tumor maligno.
		\item En función del resultado de la prueba, se procede al empleo de uno de los modelos para identificar el posible tipo de enfermedad que puede ser, atendiendo a los casos más probables a nivel clínico.
	\end{enumerate}
	\item Una vez terminada la aplicación de los modelos de aprendizaje para la clasificación, se asignará de forma definitiva un diagnóstico, que será obtenido en base a los resultados probabilísticos más altos de la salida de los modelos evaluados. Esta información, junto con la miniatura de la imagen, y una descripción de la enfermedad, será mostrada al usuario en pantalla, no tardando más de dos segundos y empleando ventanas flotantes para una mayor legibilidad.
\end{enumerate}

\subsubsection{Restricciones a tener en cuenta}

Para la realización del producto final, han de tenerse en cuenta los siguientes aspectos:

\begin{itemize}

	\item El usuario debe ser capaz de aportar la imagen para el diagnóstico empleando su galería o bien la cámara de su dispositivo. Opcionalmente, puede ofrecerse la posibilidad de rotar o recortar la imagen objetivo.
	\item Se debe incluir información de uso para facilitar al usuario el proceso de aportación de la información.
	\item Se debe tener en cuenta que el cliente no disponga de fotografías existentes, y deba realizar una nueva para continuar el proceso de diagnóstico.
	\item Los diagnósticos realizados deben ser visibles ordenados de mayor a menor antigüedad en una sección dedicada, de forma que el usuario sea capaz de revisitar los casos anteriores y realizar un seguimiento de los mismos.
	\item La implementación se ha de realizar en Java o Kotlin, de manera que la app sea compatible con dispositivos Android.
	\item El sistema ha de tener un comportamiento fluido y dinámico para el usuario, no tardando más de dos segundos por imagen para realizar el diagnóstico.
	\item La interfaz de usuario ha de ser fluida y eficiente, con el menor tiempo de carga posible entre funcionalidades.
	\item El almacenado del histórico ha de realizarse de forma local para velar por la privacidad del usuario.
	\item Los modelos de aprendizaje empleados han de ser compatibles con el estándar Torchscript/Pytorch.
\end{itemize}

\subsection{Requisitos}

Una vez analizado el problema, podemos identificar una serie de requisitos que debemos cumplir. Estos pueden clasificarse en tres tipos:

\begin{itemize}
	\item Requisitos funcionales. Describen la interacción entre el sistema y el entorno, indicando como ha de reaccionar la app ante ciertos estímulos.
	\item Requisitos no funcionales. Especifican restricciones del sistema no relacionadas directamente con su comportamiento funcional, pero que son clave durante el diseño e implementación del sistema.
	\item Requisitos de información. Describe las necesidades de almacenamiento del producto software que se está desarrollando.
\end{itemize}

Podemos encontrar los requisitos siguientes para el problema descrito.

\subsubsection{Requisitos funcionales}

\begin{itemize}
	\item \textbf{RF-1}. El usuario debe ser capaz de realizar el diagnóstico de una fotografía empleando el modelo de inteligencia artificial subyacente.
	\item \textbf{RF-2}. El sistema debe proporcionar la funcionalidad al usuario de leer la imagen de galería o mediante nueva fotografía.
	\item \textbf{RF-3}. El usuario debe poder recortar la imagen y seleccionar el área de interés.
	\item \textbf{RF-4}. El sistema debe ofrecer la posibilidad al usuario de mostrar el historial de fotografías examinadas
	\item\textbf{RF-5}. Cada fotografía debe ser correctamente ordenada por su fecha y hora de toma, en orden descendente.
	\item \textbf{RF-6}. El sistema de diagnóstico debe ofrecer al usuario retroalimentación sobre la enfermedad, como su nombre y descripción de la gravedad.
	\item \textbf{RF-7}. Tras el diagnóstico, el usuario debe ser capaz de observar la información asociada al resultado.
\end{itemize}

\subsubsection{Requisitos no funcionales}
	\begin{itemize}
		\item \textbf{RNF-1}. El tiempo de respuesta del modelo de aprendizaje debe ser inferior a 2 segundos.
		\item \textbf{RNF-2}. El lenguaje de programación a emplear debe ser Java o Kotlin
		\item \textbf{RNF-3}. La interfaz de usuario ha de ser implementada mediante ficheros XML, y siguiendo los requisitos de diseño de Material Design de Google para cumplir con las especificaciones de interfaz de Android.
		\item \textbf{RNF-4} La inferencia del modelo de aprendizaje ha de realizarse empleando imágenes tipo bitmap de 512 x 512
		\item \textbf{RNF-5} La implementación del proceso de inferencia debe hacer uso de la libería Pytorch Mobile.
		\item \textbf{RNF-6} El recorte de la imagen debe realizarse a tiempo real.
		\item \textbf{RNF-7} El menú de histórico de diagnósticos ha de ser sencillo, con opción de scroll, e implementado con un ViewPager de Java para poder ordenar los casos de mayor a menor antigüedad.
	\end{itemize}
\subsubsection{Requisitos de información}

\begin{itemize}
	 \item \textbf{RI-1: Historial}. El historial debe almacenar la información de cada diagnóstico realizado en el tiempo. Debe almacenarse de forma local en la memoria secundaria del dispositivo.
	 \begin{itemize}
	 	\item Contenido: imagen miniaturizada del diagnóstico, nombre de la enfermedad diagnosticada y si se trata de una patología benigna o maligna. 
	 	\item Requisitos asociados: RF-4, RF-5, RF-6.
	 \end{itemize}
	\item  \textbf{RI-2: Descripción de las enfermedades}. Almancena un breve resumen para cada enfermedad, codificada por su nombre de etiqueta.
		 \begin{itemize}
		\item Contenido: nombre de la enfermedad, descripción.
		\item Requisitos asociados: RF-4, RF-5, RF-6, RF-7.
	\end{itemize}
\end{itemize}

\subsection{Glosario de términos}
Las palabras clave de la descripción se concentra en la siguiente lista:
\begin{itemize}
	\item Diagnóstico: proceso mediante el cual se identifica la enfermedad de un paciente.
	\item Prognosis: acción de pronosticar, o juzgar médicamente una condición de un paciente.
	\item Lesión cutánea: mancha o tumor de características sospechosas originado en la piel.
	\item Modelo de aprendizaje profundo: algoritmos basados habitualmente en el uso de arquitecturas neuronales, empleados para la clasificación o segmentación de imágenes de entrada con el fin de dar como salida una clasificación o característica de relevancia sobre la entrada.
	\item Ventana flotante: interfaz de usuario que se superpone de forma total o parcial sobre una ventaja ya existente. En Android, esta funcionalidad es implementada por los fragmentos, o bien, el uso de otros componentes específicos como ``viewpager''.
\end{itemize}
	
	
	\input{capitulos/08_Implementación}
	
	\input{capitulos/09_PruebasUso}
	
	\input{capitulos/10_ConclusionesFuturo}
	
	
	%\chapter{Conclusiones y Trabajos Futuros}
	%
	%
	\nocite{}
	\bibliography{bibliografia/bibliografia}
	\bibliographystyle{plain}
	
	\appendix
	%\input{apendices/manual_usuario/manual_usuario}
	%\input{apendices/paper/paper}
	%%\input{glosario/entradas_glosario}
	% \addcontentsline{toc}{chapter}{Glosario}
%	 \printglossary
	%\chapter*{}
	\thispagestyle{empty}

\end{document}
