\chapter{Introducción}

\section{Motivación}

El cáncer es una de las causas de muerte principales en el mundo. Su gran agresividad, así como su dificultad de diagnóstico, debido a su gran variedad de ubicaciones y manifestaciones, provoca que un alto porcentaje de casos no sean diagnosticados a tiempo correctamente. Tan solo en 2023, aproximadamente se registraron 20 millones de nuevos casos de cáncer a nivel mundial, y produciéndose algo menos de 10 millones de defunciones.\\

Estos registros provocan una gran inquietud en la población y entre los expertos de la materia; debido al aumento que se produce cada año, se espera que para el año 2050, el número de nuevos casos sea un 70\% mayor.  Por desgracia, no existen formas de prevención claras para este tipo de enfermedad, ni un tratamiento efectivo que permita al paciente recuperarse fácilmente. \\

La única opción probada es la realización de pruebas rutinarias a colectivos de riesgo, para así acelerar la detección de posibles tumores, y aumentar la esperanza de supervivencia. Esto se ve reflejado en las cifras de los dos tipos de cáncer más frecuentes: el cáncer de mama, y el cáncer colorrectal. Si se detectan en fases iniciales, la correcta recuperación del colon podría aumentarse hasta el 90\%, mientras que en el cáncer de mama, podría reducirse su mortalidad entre un 25-31\%. Gracias a la existencia de pruebas rutinarias programadas por servicio de salud, se puede reducir la mortalidad.\\

El gran problema de estos tipos de cánceres son la escasa visibilidad y síntomas de los mismos; cuando muestran señales, es probable que la tasa de supervivencia sea mucho menor, sobre todo en el colon. Pero existe otro tipo de cáncer que sí se manifiesta de forma más visible y que puede alarmar al paciente de forma más temprana: el cáncer de piel.\\

Este tipo de tumores puede manifestarse en las diferentes capas de la dermis, y su origen se atribuye a la exposición prolongada a la luz solar sin hacer uso de protección. Debido a los daños que sufre la capa de ozono, y otros factores ambientales, la cantidad de rayos ultravioleta que llegan hasta la superficie ascendió desde que se tienen registros. Si bien la capa de ozono parece recuperarse, debemos ser cautos, y tener cuidado de nuestra piel; los rayos ultravioleta pueden dañar células de la misma, y provocar alteraciones en su material genéctico. Son las que dan lugar al crecimiento incontrolado de células, son las que forman los tumores cancerígenos en la piel.\\

Se estima que en el mundo, los tumores de la piel representan un tercio de los casos de cáncer diagnosticados. Esta distribución sigue valores parecidos en España, y al igual que las cifras de otros tipos de tumores, los casos diagnosticados aumentan año tras año. Las muertes debidas a esta enfermedad son principalmente, por ser indentificadas en fases tardías de su evolución. Debido a que la piel es el órgano más grande del cuerpo humano, y que está en contacto con todos los capilares sanguineos y el sistema linfático, las células cancerosas se pueden extender por ellos hacia otros lugares del cuerpo.\\

Aunque este cáncer puede ser identificado de forma más sencilla por su portador, la escasa información acerca del tema, y la confusión con otras lesiones benignas de la piel como verrugas o lunares, provoca una disminución en las posibilidades de supervivencia. Por ello, el objetivo de este trabajo es aportar una nueva forma de diagnóstico que permita a los usuarios obtener una orientación acerca de qué posible lesión están experimentando en la piel, y sirvan como complemento del experto. O bien, ayudar a los expertos a tomar la decisión, acortando los tiempos de diagnóstico para aumentar las posibilidades de supervivencia. Esta tarea será realizada gracias al uso de uno de las herramientas en auge en la actualidad: la inteligencia articificial, y concretamente, el uso de DeepLearning para visión por computador.\\
 
 Mediante una nueva arquitectura, el propósito es conseguir un buen modelo, capaz de segmentar las manchas de interés en la piel que estén recogidas en una fotografía. Dicha fotografía será capturada con el telefóno movil del usuario, retirando así la necesidad de disponer de dispositivos especializados. Y posteriormente, clasificar dichas manchas para ofrecer al usuario final una respuesta sólida acerca del posible tipo de lesión de piel que sufre.\\

\newpage
\section{El cáncer de piel}

Prosiguiendo con el cáncer de piel, su diagnóstico si dificulta, sobre todo, por su amplia variedad de formas, tamaños, texturas y manifestaciones. Aunque su visibilidad pueda parecer evidente, (ya que es observable a nivel macroscópico) puede ser confundido fácilmente con lesiones benignas. Normalmente, suele dividirse entre dos tipos diferentes:
\begin{itemize}
	\item \textbf{Melanomas de la piel}. Son la variante más peligrosa. Su origen se encuentra en los melanocitos, las células encargadas de dar el color bronceado a la piel.  Éstas pueden comenzar a crecer sin control originando tumores, los cuales crecen y se diseminan rápidamente hacia otras regiones del organismo, provocando la metástasis, una extensión a nivel total del organismo. Es el más grave de los diagnósticos. Puede identificarse como una mancha oscura en la piel, formando tumores de color café oscuro. Sin embargo, debido a la gran variedad de reacciones, pueden darse de color rosado si dejan de producir melanina. Este aspecto dificulta su diagnóstico, por lo que el papel de las herramientas de visión por computador pueden ayudar a su identificación.
	\item \textbf{Cánceres no melanomas}. Este tipo de cánceres no se ubican en los melanocitos, y pueden ser tratados mediante otras ténicas menos agresivas debido a su rara probabilidad de expansión. Los más comunes, son los tumores de células basales y los de células escamosas:
	\begin{itemize}
		\item Células basales. Componen la capa inferior de la piel, y son las células encargadas de sustituir aquellas que componen la capa más externa de la piel. Se encuentran , por tanto, en constante reproducción para cubrir aquellas que mueren en la superificie. Si experimentan alguna mutación, producen tumores de color similar al de piel del paciente, con la posibilidad de aparecer en colores como negro brillante en las pieles más oscuras.
		
		\item Células escamosas. Son las células externas de la piel, con forma plana. Se regeneran constantemente gracias a las células basales, que producen estas células las cuales se aplanan a medida que ascienden hacia la capa externa. Es frecuente, de nuevo, en zonas expuestas al sol, sobre todo la cara. Normalmente, se encuentran bien localizados, y puede procederse a su extirpación. En casos en los que se haya extendido, se hace uso de radioterapia.
		
	\end{itemize}
\end{itemize}

Aunque en base a su descripción parezcan distinguibles, son fácilmente confudidos por su variedad con otros tumores benignos de la piel, como:

\begin{itemize}
	\item \textbf{Lunares(nevus)}: hiperpigmentación benigna en la piel.
	\item \textbf{Verrugas}: tumores benignos de piel, frecuente debido a virus como el del papiloma humano.
	\item \textbf{Lesiones vasculares}: varices, derrames, y otro tipo de problemas circulatorios.
	\item \textbf{Lipomas}: tumores de tacto blando, debido a su contenido en lípidos (grasa).
	\item \textbf{Queratosis seborreica}: son manchas cerosas, comúnmente desarrolladas en la espalda. De aspecto oscuro y gran relieve, no suponen ninguna amenaza más allá de posible incomodidad al roce o estética.
\end{itemize}

El uso de aprendizaje profundo para este fin resulta interesante como forma de mejora del diagnóstico ante casos malignos y benignos de gran similitud, los cuales pueden confunir y dificultar la labor incluso a expertos dermatólogos.
