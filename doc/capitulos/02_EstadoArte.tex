\chapter{Estado del arte}

En este apartado, evaluaremos los modelos usados para la clasificación de tumores y lesiones cutáneas en la literatura, para así comprender las ventajas que ofrece cada modelo, y los puntos en contra, y obtener una referencia de la tendencia general.

\section{Aprendizaje profundo en dispositivos móviles}
\subsection{Vertientes actuales}
Una de las tendencias actuales en auge consiste en la integración de modelos de aprendizaje en aplicaciones de uso diario para mejorar la predictibilidad de ciertos fenómenos, o bien adecuarse a los hábitos y la vida diaria del usuario. Esto requiere una buena capacidad de cómputo en los dispositivos móviles, cuya principal limitación son las restricciones de espacio y consumo. Por este motivo, existen numerosas vertientes de investigación para conseguir arquitecturas convolucionales preentrenadas, a semejanza de arquitecturas como ResNet o VGGNet, pero con un menor número de parámetros para reducir la cantidad de operaciones aritméticas necesarias para el aprendizaje y la inferencia de los resultados.\\

Empresas de alto nivel, como Google, apuestan por este tipo de arquitecturas, como MobileNet. Ésta se trata de 
\subsection{Comparativa de uso entre Deep Learning y Machine learning}
\subsection{Recursos disponibles}


\section{Procesado de imágenes cutáneas}
\subsection{Técnicas de reducción de ruido}
\subsection{Normalización}
\subsection{Extracción de características}


